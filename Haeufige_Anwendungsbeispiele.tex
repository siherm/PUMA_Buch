\section{Häufige Anwendungsbeispiele}
\textit{Für wen ist PUMA geeignet und wie? Ein kleiner Überblick über die unterschiedlichen Zielgruppen und Funktionen.}
\subsection{Materialsammlung für die Bachelorarbeit}
Mit dem Festlegen des Themas ist der erste Schritt gemacht und die Literatursuche kann beginnen. Bei der Literatursuche im Internet und in diversen Bibliothekskatalogen ist PUMA für jedermann geeignet. Mit Hilfe von Bookmarklet-Buttons werden mit einem Klick Lesezeichen und Publikationen in die eigenen PUMA-Sammlung eingetragen. So kann gespeichert, gemerkt und gesammelt werden, ohne dass die Literatursuche unterbrochen werden muss. Dies ist einfach und spart Zeit. 
 \newline
Es kommt nicht selten vor, dass einem bei der Menge an gesammelten Publikationen und Lesezeichen der Überblick verloren geht. Um Chaos zu vermeiden arbeitet PUMA mit Tags. Der Nutzer gibt diese Tags beim Eintragen der Publikation/des Lesezeichen mit an. Es gibt keine Beschränkung was die Anzahl der Tags betrifft. Die Vergabe der Tags erleichtert einem später das Finden der passenden Literatur zu einem Thema in der eigenen Sammlung. Es wird nach einem Tag gesucht und PUMA zeigt die ganze Literatur mit diesem Tag an. So verhindert PUMA das Entstehen einer chaotischen Literaturliste.
 \newline
Das Wichtigste am Ende einer Bachelorarbeit ist das Literaturverzeichnis. Auch in diesem Abschnitt der Arbeit steht PUMA den Nutzern tatkräftig zur Seite. Mit einem einzigen Klick können, die in der Bachelorarbeit verwendeten Publikationen, in die Ablage übertragen werden. Alle Publikationen, die sich in der Ablage befinden, können anschließend im gewünschten Zitationsstil  angezeigt werden. Es entsteht so in wenigen Klicks eine komplettes Literaturverzeichnis und die  Bachelorarbeit ist fertig.
 
\subsection{Liste mit eigenen Publikationen auf einer Mitarbeiterhomepage}
Die eigene Publikation ist veröffentlicht, doch wie wird nun weiter vorangegangen? \newline Das Verwalten und Sortieren der eigenen Publikationen gestaltet sich mit PUMA einfach. Eigenen  Publikation werden, wie alle Publikationen, in PUMA eingetragen. Der einzige Unterschied besteht darin, dass der Nutzer sich als Autor der Publikation zu erkennen gibt, indem ein Häkchen bei \textit{Ich bin (Mit-)Autor} gesetzt wird. PUMA setzt sofort automatisch den Tag \textit{myown}. Durch den Tag wird das Suchen nach den eigenen Veröffentlichungen vereinfacht und ein Überblick ist schnell hergestellt.\newline
Natürlich sollen Außenstehende und Nicht-PUMA-Nutzer ebenfalls auf die eigenen neuen Publikationen aufmerksam gemacht werden. Das Veröffentlichen einer Literaturliste auf der eigenen Homepage ist hierbei eine der gängigsten Formen. Mit Hilfe des OpenCMS-Plugins von PUMA ist dies in wenigen Schritten möglich. Im erste Schritt werden die eigenen Publikationen in PUMA eingetragen und mit \textit{myown} getaggt. Im nächsten Schritt wird das Element \enquote{Publikationsliste (aus BibSonomy/PUMA)} in die Homepage eingefügt und die  entsprechenden Felder ausgefüllt. Dabei ist nicht zu vergessen, dass in das Feld \enquote{Tags} \textit{myown} eingegeben wird. Durch einen Klick auf \enquote{Veröffentlichen} wird nun eine Literaturliste aus den eigenen Publikationen auf der Homepage angezeigt.  
Ab sofort können sowohl PUMA-Nutzer, als auch Externe, auf die Veröffentlichungen aufmerksam werden.


\subsection{Institutspublikationsliste}
In einem Institut mit vielen Mitarbeitern wird oft gleichzeitig an mehreren Publikationen gearbeitet. Um den Überblick über alle Institutspublikationen zu behalten, hilft PUMA beim Sammeln und Erstellen einer eigenen Institutspublikationsliste. \newline
Das Institut für Systemtheorie und Regelungstechnik (ist) der Universität Stuttgart arbeitet seit einiger Zeit mit PUMA. Die Mitarbeiter des Instituts nutzen das System zur Erstellung der Institutspublikationsliste, die auf der eignen Homepage\footnote{\url{http://www.ist.uni-stuttgart.de/forschung/publikationen/index.html}} angezeigt wird. So können sich Studenten und Externe jederzeit einen Überblick über die Institutspublikationen verschaffen.\newline
Für die Sammlung der Institutspublikationen gibt es mehrere Möglichkeiten. Die erste Möglichkeit besteht darin, dass das Institut einen Verantwortlichen für  die Institutspublikationen festlegt. Der Verantwortliche sammelt über seinen persönlichen Account die Publikationen und trägt diese in PUMA ein. Aus seiner eigenen Sammlung erstellt er anschließend die Institutspublikationsliste. Eine weitere Möglichkeit wäre das Erstellen eines Funktionskontos (fn-Konto). Dieses Konto muss beim TIK der Universität Stuttgart beantragt werden. Es funktioniert wie das st- oder ac-Konto. Über dieses Konto können nun alle Institutsmitarbeiter oder ein Verantwortlicher die eigenen Veröffentlichungen/ Institutsveröffentlichungen eintragen. Aus der Sammlung kann eine gemeinsame Institutspublikationsliste generiert werden. Der Vorteil bei dieser Möglichkeit ist, dass das Konto an keinen bestimmten Mitarbeiter gebunden ist und übertragen werden kann. Die letzte Möglichkeit wäre das Erstellen einer Gruppe für das Institut. Die Mitarbeiter sind Mitglieder der Gruppe und können ihre eigenen Publikationen für die Gruppe sichtbar machen und diese in die Sammlung der Gruppe übertragen. Aus dieser  Sammlung kann wiederum eine gemeinsame Institutspublikationsliste generiert werden.\newline
PUMA vereinfacht so das Arbeiten und bietet den Instituten unterschiedliche Wege die gemeinsame Institutspublikationsliste zu erstellen.
\newline \newline
\textbf{Ein typischer Anwendungsfall:}\newline
- Pflege der Publikationsdaten durch Hilfskräfte und/oder Mitarbeiter des Instituts.\newline\newline
- Veröffentlichung einer Institutspublikationsliste auf der Institutshomepage, sortiert nach Datum in absteigender Reihenfolge und/oder im ausgewählten Zitationsstil.\newline\newline
- Institutsmitarbeiter können Publikationslisten auf Ihrer eigenen Website veröffentlichen.
\newline\newline
\textbf{Umsetzung mit PUMA:}
\begin{enumerate}
\item Anmeldung auf \url{https://puma.ub.uni-stuttgart.de} 
\item Eine Gruppe anlegen
\item Hilfskräfte und/oder Institutsmitarbeiter tragen die Publikationen mit ihren PUMA-Konten ein und taggen diese mit \textit{for:Gruppenname}.
\item Im OpenCMS wird in der Freitextseite aus dem Typenktalog der Typ \enquote{Publikationsliste aus BibSonomy/PUMA} platziert und ausgefüllt.
\end{enumerate}

\subsection{Was passiert mit dem Benutzerkonto, wenn Mitarbeiter aus der Uni ausscheiden?}
Scheidet ein Institutsmitarbeiter der Universität Stuttgart aus, bleibt sein Benutzerkonten bei PUMA auch nach Ausscheiden erhalten, aber eine Authentifizierung ist nicht mehr möglich. Die öffentlich geteilten Einträge des Ausgeschiedenen können von anderen Nutzern dennoch weiter genutzt werden.  Somit müssen die Konten nicht umgestellt werden, wenn der Auszuscheidende seine Daten nicht löschen möchte.\newline\newline
Eine weitere Möglichkeit wäre, dass andere Nutzer seine Einträge in die eigene Sammlung kopieren und einen möglichen Verlust der Daten verhindern. Der beste Weg besteht aber darin, für eine Institutspublikationsliste,
die Gruppenfunktion zu nutzen:
\begin{enumerate}
\item Legen Sie eine Gruppe an und geben Sie ihr einen Namen, z.B. das Institutskürzel.
\item Legen Sie in Ihrem Benutzerkonto im Reiter \enquote{Einstellungen} in der Rubrik \enquote{Gruppen} fest, wer Mitglied der Gruppe sein soll. dies kann durch eine Einladung oder aktive Anfrage der Mitarbeiter erfolgen. 
\item Die Administrator-Rechte werden festgelegt.
\item Alle relevanten Einträge werden mit dem Systemtag \textit{for:gruppenname} getaggt.
\item Alle Gruppen-Administratoren können nun die Einträge mit dem Benutzer
\textit{@gruppenname} verwalten.
\item Die Institutspublikationsliste wird über die Gruppe erzeugt, beispielsweise über das OpenCMS-Plugin. 
\end{enumerate}
Mit dem Systemtag \textit{for:gruppenname} legen Sie im übertragenen Sinn einen neuen Benutzer an, den Sie gemeinsam über die Gruppenfunktion verwalten können.
