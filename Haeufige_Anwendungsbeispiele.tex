\chapter{Typische Anwendungsbeispiele}
\label{ch:typischeAnwendungsbeispiele}

\section{Institutspublikationsliste}
\label{sec:institutspublikationsliste}
Für die Veröffentlichung der Institutspublikationsliste (\autoref{subsec:opencms}) gibt es mehrere Möglichkeiten:
\begin{itemize}
\item Über die Gruppenfunktion (\autoref{sec:gruppen}) von PUMA: Diese Funktion ermöglicht, gemeinsam ein PUMA-Gruppenkonto zu verwalten. Gruppenadmins können Einträge bearbeiten, die Mitglieder der Gruppe an diese per Systemtag \textit{for:Gruppenname} (\autoref{systemtag}) geschickt haben. Damit ist der Eintrag im Besitz der Gruppe (\textit{@Gruppenname}) und des Gruppenmitglieds. Dieser Weg hat mehrere Vorteile: Zum einen können mehrere Personen Einträge bearbeiten. Zum anderen können in der Gruppe gleichzeitig auch Volltexte geteilt und Dubletten schneller erkannt werden. Aus dieser  Sammlung kann wiederum eine gemeinsame Institutspublikationsliste generiert werden.
\item Über ein Funktionskonto (fn-Konto): Das fn-Konto muss beim TIK über den Benutzerverwaltungsadministrator des Instituts beantragt werden. Dieses Konto funktioniert wie das st- oder ac-Konto, es hat den Vorteil, dass es nicht an eine Person gebunden ist und kann von verschiedenen Personen bedient werden. Aus der Sammlung kann eine gemeinsame Institutspublikationsliste generiert werden.
\item Über ein persönliches Konto werden die Publikationen gesammelt und in PUMA eingetragen. Aus der eigenen Sammlung wird dann die Institutspublikationsliste erstellt.
\end{itemize}
Scheidet ein Institutsmitarbeiter der Universität Stuttgart aus, bleibt sein Benutzerkonto bei PUMA auch nach Ausscheiden (\autoref{subsec:kontoaufloesen}) erhalten. Eine Authentifizierung ist nicht mehr möglich. Damit können diese Einträge vom Nutzer nicht mehr bearbeitet werden. Öffentlich geteilte Einträge des Ausgeschiedenen bleiben weiterhin nutzbar.\\
PUMA-Nutzer können ihr Konto jederzeit löschen (\autoref{subsec:kontoloeschen}). Damit sind ihre Einträge ebenfalls gelöscht. Wenn die Publikationen am Institut erhalten bleiben sollen, kann ein anderer Nutzer des Instituts die Einträge vorher in die eigene Sammlung kopieren oder der Nutzer sendet per Systemtag \textit{for:Gruppenname} (\autoref{systemtag}) seine Publikationseinträge an eine Institutsgruppe. Die Pflege einer Institutspublikationsliste sollte aber möglichst von vorneherein über die Gruppenfunktion oder ein fn-Konto erfolgen.

\section{Eigene Publikationslisten verwalten und veröffentlichen}
\label{sec:eigenePublistenVerwalten}
Das Verwalten und Sortieren der eigenen Publikationen gestaltet sich mit PUMA einfach. Eigene  Publikationen können in PUMA gekennzeichnet werden. Nutzer können sich als Autoren der Publikation zu erkennen geben, indem beim Publikationseintrag ein Häkchen bei \textit{Ich bin (Mit-)Autor} gesetzt wird. Dies erzeugt automatisiert den Systemtag \textit{myown}. Systemtags sind \enquote{Tags}, die einheitlich definiert sind und weitere Funktionen beinhalten (\autoref{systemtag}).\\
Mit Hilfe des Plugins \enquote{Publikationsliste (aus BibSonomy/PUMA)} für OpenCms (\autoref{subsec:opencms}) kann die Publikationsliste auf einer Internetseite der Universität Stuttgart veröffentlicht werden. Das Plugin wird über das Zauberstab-Menü im Seiteneditor ausgewählt. Durch Verbindung mit dem PUMA-Benutzerkonto über den API-Schlüssel und Auswahl des eigenen Benutzerkontos werden alle Einträge, die mit diesem Konto bibliografiert wurden, angezeigt. Mit der Eingabe des \enquote{Tags} \textit{myown} wird nur die eigene Publikationsliste veröffentlicht.



\section{Materialsammlung für die Bachelorarbeit}
\label{sec:materialsammlungBachelorarbeit}
Um den Überblick über die Materialrecherche nicht zu verlieren, empfiehlt es sich spätestens bei der Bachelorarbeit den Umgang mit Literaturverwaltungsprogrammen zu üben. Bei der Literatursuche im Internet und in Bibliothekskatalogen unterstützt PUMA im Browser das Sammeln von Lesezeichen und Publikationen, ohne dass die Suche unterbrochen werden muss. Mit Unterstützung des PUMA-Add-ons für den Firefox (siehe \autoref{sec:addon}) oder Bookmarklet-Buttons (siehe \autoref{sec:button}) können diese in die eigenen PUMA-Sammlung eingetragen werden.\\
Es kommt nicht selten vor, dass einem bei der Menge an gesammelten Publikationen und Lesezeichen der Überblick verloren geht. In PUMA werden Materialsammlungen mit \enquote{Tags} (\autoref{subsec:tags}) strukturiert. Der Nutzer gibt diese \enquote{Tags} beim Eintragen mit an. Es gibt keine Beschränkung was die Anzahl der Tags betrifft. Die Vergabe der Tags erleichtert einem später das Finden der passenden Literatur zu einem Thema in der eigenen Sammlung. Wird nach einem Tag gesucht, zeigt PUMA alle Einträge mit diesem \enquote{Tag} an. So unterstützt PUMA das Entstehen einer strukturierten Literaturliste.\\
Im nächsten Schritt wird die Literatursammlung in Word importiert (siehe \autoref{importWord}). LaTeX-Nutzer können eine BibTeX-Datei aus PUMA exportiert. Beim Schreiben kann die Literatur dann referenziert und ein Literaturverzeichnis erzeugt werden.
