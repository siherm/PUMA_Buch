\section{PUMA als Forschungsprojekt}
\textit{PUMA wächst und wächst. Wie die Geschichte von PUMA begann und was die Zukunft bringt.}
\newline
\newline
Die Web-Anwendung PUMA ist ein Schwester-System von BibSonomy.  PUMA wurde durch die Deutsche Forschungsgemeinschaft (DFG) von Anfang August 2009 bis Juli 2011 gefördert. Die Universitätsbibliothek Kassel und das Fachgebiet Wissensverarbeitung der Universität Kassel waren an der PUMA-Entwicklung beteiligt. Nach der ersten Projektphase wurden das System als Open-Source-Software veröffentlicht. PUMA profitiert von der Weiterentwicklung seines Schwester-Systems BibSonomy\index{BibSonomy} und wird immer wieder aktualisiert. In einer zweiten Förderphase der DFG von 2013 bis 2015 trat die DMIR-Gruppe (Data Mining \& Information Retrieval Group) von der Universität Würzburg dem PUMA-Team bei. In ihrem Rahmen wurde die Zusammenarbeit mit Fremdsystemen ausgebaut, wie zum Beispiel die Integration von PUMA im Discovery Service des HeBIS-Verbundes, dem die Open-Source-Software Vufind zugrunde liegt. Ergebnis ist ein Modul, das die Merklistenfunktion dahingehend erweitert, dass die Einträge automatisch in PUMA gespeichert und die Inhalte der Merkliste aus PUMA geholt werden. Bis zum derzeitigen Stand gibt es PUMA an insgesamt 13 Universitäten und Universitätsbibliotheken in Deutschland. 
\newline\newline
\textbf{Weiterführende Literaturhinweise}\newline
\begin{itemize}
\item \url{http://blog.ub.uni-stuttgart.de/category/puma/}
\item \url{http://www.academic-puma.de/}
\item Benz D. et al. (2010): Academic Publication Management with PUMA – Collect, Organize and Share Publications. In: Lalmas M., Jose J., Rauber A., Sebastiani F., Frommholz I. (eds) Research and Advanced Technology for Digital Libraries. ECDL 2010. Lecture Notes in Computer Science, vol 6273. Springer, Berlin, Heidelberg
\item Universität Kassel (o.J.) Publications of the Knowledge \& Data Engineering Group. Internet: \url{http://www.kde.cs.uni-kassel.de/pub}
\item \url{http://www.kde.cs.uni-kassel.de/bibsonomy/dumps/}
\item Benz, D.; Hotho, A.; Jäschke, R.; Krause, B.; Mitzlaff, F.; Schmitz, C. \& Stumme, G. (2010):The Social Bookmark and Publication Management System BibSonomy, The VLDB Journal 19 (6), S. 849-875. 
\end{itemize}