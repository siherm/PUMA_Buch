\section{Schnelleinstieg}
\textit{Sammeln, Verwalten und teilen- so lautet die Devise.}
\subsection{Sammeln}
Bookmarks, bibliografische Daten und Volltexte kostenlos online speichern.\newline\newline
\textbf{Login}
\begin{itemize}
\item Sichere Verbindung: \url{https://puma.ub.uni-stuttgart.de/}
\item Angehörige der Universität Stuttgart: st- oder ac-Konto
\item Benutzername frei wählbar
\item Keine Lizenzgebühr
\end{itemize}
\textbf{Die ersten Schritte}
\begin{itemize}
\item Publikationen manuell eintragen oder per ISBN/ DOI automatisch aus Datenbanken abrufen
\item Vergabe von Schlagwörtern (\enquote{Tags})
\item Mindestens ein \enquote{Tag} ist obligatorisch
\item Sichtbarkeit einstellen: privat, öffentlich oder für eine Gruppe
\item Ausführliche Anleitung: PUMA-Hilfe im Benutzermenü
\item Schulung für Einsteiger in der Universitätsbibliothek Stuttgart: \url{www.ub.uni-stuttgart.de/puma}
\end{itemize}
\textbf{Schlagworte (\enquote{Tags}) nutzen}
\begin{itemize}
\item Ausgabe von eigenen Publikationslisten per Systemtag \textit{myown}
\item Beliebige Sortiermöglichkeiten über Vergabe von Tags
\item Kombination von Tags zur Eingrenzung von Abfragen
\end{itemize}
\subsection{Verwalten}
Literaturlisten in unterschiedlichen Zitationsstilen erzeugen und exportieren.\newline\newline
\textbf{Import und Export von Daten}
\begin{itemize}
\item Vorhandene Publikationslisten importieren (BibTex oder Endnote-Formate)
\item Duplikatserkennung
\item Ausgabe von Literaturlisten in rund 7.500 Zitationsstilen
\item Oder individuelle Anpassung per Citation Style Language (CSL)
\item Export der Daten in Standardformate wie BibTex, Endnote oder XML
\item Viele Sortiermöglichkeiten nach Autor, Jahr, Dokumententyp u.a.
\end{itemize}
\subsection{Teilen}
Einträge per Plugin auf der Website veröffentlichen oder in Gruppen Zugänglich machen.\newline\newline
\textbf{Soziale Funktion}
\begin{itemize}
\item Gruppen anlegen und Mitglieder veralten
\item Sichtbarkeit auf die Gruppe beschränken
\item Gemeinsam Einträge bearbeiten (Community-Posts)
\item Freunden folgen und Einträge senden
\end{itemize}
\textbf{Plugin für OpenCMS der Universität Stuttgart}
\begin{itemize}
\item Verbindung mit dem PUMA-Benutzerkonto
\item Eigene Publikationslisten aus PUMA laden
\item Dokumentation im Typkatalog des TIK: \url{http://www.tik.uni-stuttgart.de/dienste/opencms/typkatalog/typ/PumaPublicationList/}
\end{itemize} 
\textbf{Typo3 Plugin}
\begin{itemize}
\item Verbindung mit dem PUMA-Benutzerkonto (Funktionsweise wie bei OpenCMS)
\item Eigene Publikationen aus PUMA laden und auf der eigenen Homepage veröffentlichen
\item Einfügen von eigenen CSL-Styles möglich
\end{itemize}