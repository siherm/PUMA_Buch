\chapter{Schnelleinstieg}
\label{ch:schnelleinstieg}
% hier noch Sprungmarken zu den entsprechenden Kapiteln setzen!
Sammeln, Verwalten und Teilen von Publikationen und Bookmarks

\section{Die ersten Schritte}
\label{sec:ersteSchritte}
\begin{itemize}
\item PUMA-Homepage: \url{https://puma.ub.uni-stuttgart.de/}
\item Anmelden: mit st- oder ac-Konto (\autoref{sec:anmeldenBeiPuma})
\item Erstanmeldung: Es muss ein Benutzername vergeben werden.
\end{itemize}

\section{Sammeln}
\label{sec:sammeln}
\begin{itemize}
\item Eintragen: Über Menü \enquote{Eintragen} $\to$ \enquote{Publikation hinzufügen}
\item Verschiedene Eintragsmöglichkeiten (\autoref{sec:publikationen}): manuell, per Listenimport (BibTeX oder EndNote) oder per ISBN/DOI automatisch Daten importieren
\item Kopieren: Über das Doppelblattsymbol, an vorhandenen Einträgen, können Publikationen in die eigene Sammlung übernommen werden. 
\item Strukturieren (\autoref{subsec:tags}): Vergabe von mindestens einem Schlagwort (\enquote{Tag}), bei eigenen Publikationen \enquote{Ich bin (Mit-)Autor} auswählen (Systemtag \textit{myown} wird vergeben) 
\item Sichtbarkeit einstellen: privat, öffentlich oder für eine Gruppe
\item Volltext (\autoref{subsec:privateDateien}): Möglichkeit eine Dateien hochzuladen, die in einer Gruppe geteilt werden kann
\end{itemize}

\section{Verwalten}
\label{sec:verwalten}
\begin{itemize}
\item Eigene Sammlung aufrufen: über Menü \enquote{Mein Puma} (\autoref{subsec:meinPuma})
\item Bearbeiten: Über das Stiftsymbol am Eintrag können diese bearbeitet werden.
\item Publikationen auswählen: gewünschten \enquote{Tag} in den Suchschlitz (\autoref{sec:suchleiste}) eingeben, für eigene Publikationen \textit{myown}
\item Publikationsliste (\autoref{subsec:lvZusammenstellen}): über Publikationsmenü \enquote{Exportoptionen für angezeigte Einträge}  (rechtes Pfeilsymbol) $\to$ \enquote{mehr} auswählen
\item Ausgabe (\autoref{subsec:lvExportieren}): Format auswählen (viele Zitationsstile bzw. Layout-Möglichkeiten stehen zur Auswahl)
\item OpenCms (\autoref{subsec:opencms}): Dynamische Ausgabe der Publikationsliste auf OpenCms-Seiten über Plugin (Typ) \enquote{Publikationsliste (aus BibSonomy/PUMA)}
\end{itemize}

\section{Teilen}
\label{sec:teilen}
\begin{itemize}
\item Über URL: Verschicken der Publikationsliste (öffentliche Einträge) als Link möglich (für den Abruf muss keine Anmeldung am System erfolgen)
\item Gruppenfunktion (\autoref{sec:gruppen}): Möglichkeit gemeinsam ein PUMA-Konto zu verwalten (Systemtag \textit{for:Gruppenname}, \autoref{sec:systemtag}). Gruppenadmins können diese Einträge bearbeiten.
\item Sichtbarkeit: einschränkbar auf Gruppe oder Freunde (\autoref{sec:freunde})
\item Soziale Funktionen: Befreundete Nutzer können sich Einträge senden (Systemtag \textit{send:Benutzername}, \autoref{sec:systemtag}), die im Eingang des anderen in die eigene Sammlung übernommen werden können (auf Benutzername klicken und auf der rechten Seite \enquote{als Freund hinzufügen} anwählen).
\end{itemize}
\newpage
\section{Plugin für OpenCms der Universität Stuttgart}
\label{sec:pluginOpencms}
\begin{itemize}
\item Auswahl des Plugins über das Zauberstab-Menü im Seiteneditor von OpenCms
\item Verbindung mit dem PUMA-Benutzerkonto über API-Schlüssel (\mbox{Menü} \enquote{Einstellungen} $\to$ \enquote{Einstellungen})
\item Auswahl, ob aus dem eigenen Benutzerkonto (User), einer Gruppe oder aus allen öffentlich Einträgen Publikationen angezeigt werden sollen (weitere Auswahl über \enquote{Tags} möglich).
\item Dokumentation im Typkatalog des TIK: \newline \url{http://www.tik.uni-stuttgart.de/dienste/opencms/typkatalog/typ/PumaPublicationList/}
\end{itemize} 
