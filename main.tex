\documentclass[b5paper,11pt,twoside]{scrbook} %176x250mm
\usepackage[utf8]{inputenc}  
\usepackage[T1]{fontenc}      % Unterstützung für Europäische-Zeichen-Ausgabe
%\usepackage{ae}               % verbesserte Unterstützung für Umlaute
\usepackage[ngerman]{babel}% deutsche Übersetzungen und Wortumbrüche
\usepackage[autostyle=true,german=quotes]{csquotes}
\usepackage[scaled=.90]{helvet}  % schönere Schriftart: Helvetica
\usepackage{graphicx}              % Unterstützung für Graphiken
\usepackage[                
   pdftex,                  % Ausgabe-Medium: PDF
   colorlinks=true,         % farbige Links in der Bildschirm-Version?
   linkcolor=blue,          % Farbe für Querverweise
   citecolor=black,         % Farbe für Zitierungen
   urlcolor=blue,           % Farbe für Links
   bookmarks=true
   ]{hyperref}              % Paket für Links im PDF
\usepackage{listings,makeidx} \makeindex
\usepackage{array,longtable} 
\usepackage{framed,xcolor} \colorlet[named]{shadecolor}{black!15}
\newsavebox\MBox
\newenvironment{tipp}{\par\begin{lrbox}{\MBox}
%\newcommand*{}{}
%\begin{minipage}[t][][b]{0.2\textwidth}}
%T
%I
%P
%P
%\end{minipage}
\begin{minipage}[t][][b]{0.8\textwidth}}
{\end{minipage}\end{lrbox}%
%\vline width \dimexpr\fboxsep\relax
\colorbox{pumablue!40}{\usebox\MBox}\par}
\usepackage{makecell}
%\usepackage[nonumberlist]{glossaries} \makeglossaries
%\loadglsentries{Glossar.tex} 
\usepackage{microtype}
\usepackage[framemethod]{mdframed}
\mdtheorem[linecolor=blue]{thmbox}{Definition}
\mdfdefinestyle{tipp}{frametitle={\texttt{\textbf{TIPP}}}, backgroundcolor=pumablue!60, roundcorner=8pt, linecolor=black, frametitlerule=true}%{outerlinewidth=5pt, innerlinewidth=0pt, outerlinecolor=black, roundcorner=5pt, frametitle={\texttt{\textbf{TIPP}}}, backgroundcolor=pumablue!40}

\usepackage{hyperref}
\usepackage[anythingbreaks]{breakurl}
\usepackage{wrapfig,graphicx}
\usepackage{tabu}
%\renewcommand{\arraystretch}{1.15}
%\makeatletter 
%\renewcommand*\thesection{\@arabic\c@section.}
%\renewcommand*\thesubsection{\thesection\@arabic\c@subsection.}
%\renewcommand*\thesubsubsection{\thesubsection\@arabic\c@subsubsection.}
%\renewcommand*\theparagraph{\thesubsubsection\@arabic\c@paragraph.}
%\renewcommand*\thesubparagraph{\theparagraph\@arabic\c@subparagraph.}
%\makeatother   
\definecolor{pumablue}{RGB}{0,81,158}
\sloppy
\newcommand{\tag}{$\langle$Tag$\rangle$}
\newcommand{\tags}{$\langle$Tags$\rangle$}
\lstdefinelanguage{JavaScript}{
keywords={typeof, new, true, false, catch, function, return, null, catch, switch, var, if, in, while, do, else, case, break},
keywordstyle=\color{pumablue}\bfseries,
ndkeywords={class, export, boolean, throw, implements, import, this},
ndkeywordstyle=\color{darkgray}\bfseries,
identifierstyle=\color{black},
sensitive=false,
comment=[l]{//},
morecomment=[s]{/*}{*/},
commentstyle=\color{purple}\ttfamily,
stringstyle=\color{red}\ttfamily,
morestring=[b]',
morestring=[b]"
}
 
\lstset{
language=JavaScript,
extendedchars=true,
basicstyle=\footnotesize\ttfamily,
showstringspaces=false,
showspaces=false,
numbers=left,
numberstyle=\footnotesize,
numbersep=9pt,
tabsize=2,
breaklines=true,
showtabs=false,
captionpos=b
}


\begin{document} 
%\begin{center}
    %\vspace*{1cm}
    \title{\Huge Literaturverwaltung mit PUMA}
    %\vspace{0.5cm}
    \subtitle{\Large Ein umfassendes Handbuch}
    %\vspace{1.5cm}
    \author{\textbf{Selina Eckert}}
    \date{2016}
    %\vfill
    %\includegraphics
    %[scale=0.25]
    %{USt_logo3_07_klein.jpg}\newline
    %\vspace{1cm}
%\end{center}

\maketitle
%\setcounter{tocdepth}{6}
\newpage
%\center{
\textit{Mit herzlichsten Dank an Sibylle Hermann und Stefan Drößler, ohne sie wären einige Latex-Error-Meldungen nie gelöst worden.}\newline\newline
\textit{Dieses Buch ist dem Entwickler von KOMA-Script, Markus Kohm, gewidmet.}%}
\pagenumbering{Roman} 
\clearpage
\tableofcontents 
\setcounter{secnumdepth}{3} 
\setcounter{tocdepth}{3} 
\newpage
\pagenumbering{arabic}
\pagestyle{headings}
\chapter{PUMA - der digitale Zettelkasten}
\textit{Publikationen und Lesezeichen sammeln, verwalten und teilen, mit PUMA ein Kinderspiel.}\newline
\newline
Das Akademische Publikationsmanagement\index{Akademische 
Publikationsmanagement} (PUMA) ist ein System zum Sammeln, Verwalten, Teilen 
und Entdecken von Lesezeichen und Publikationen.\newline
Es ist zu vergleichen mit einem riesigen digitalen 
Zettelkasten, der für alle möglichen Quellen und Medien einsetzbar ist. PUMA 
ermöglicht Struktur und Ordnung für gesammelte Publikationen. Gespeicherte 
Publikationen und Lesezeichen lassen sich schnell wieder finden. Gleichzeitig 
bietet PUMA Platz für Notizen und Anmerkungen sowie eine Zusammenarbeit mit 
anderen PUMA-Nutzern. 
Die Software steht lizenzfrei als Webanwendung zur Verfügung.
\newline 
PUMA ist so konzipiert, dass es als alleiniges Eingabeportal für bibliografische Metadaten dienen kann. Außerdem können zu Literatureinträgen Dokumente hochgeladen werden. \newline
Durch die Vielzahl an Exportformaten und Schnittstellen zu anderen Programmen müssen die Nutzer ihre Daten nur einmal pflegen und können sie in anderen Systemen nachnutzen. Die wiederholte manuelle Eingabe von Publikationslisten entfällt. So können Forscher ihre Publikationslisten direkt aus PUMA auf ihre Homepage laden.  
\section{Für wen ist dieses Buch?} 
Dieses Buch richtet sich an die Angehörigen der Universität Stuttgart.
Die im Buch erklärten Beispiele basieren auf der PUMA-Installation der 
Universität. Diese Beispiele gelten in leicht abgewandelter 
Version für jede Institution, die PUMA installiert hat. 
\newline
Externe, die nicht der Universität Stuttgart angehören, können sich nicht bei 
dem hier vorgestellten PUMA anmelden. Für sie bietet sich die Nutzung 
von BibSonomy an. Da PUMA und BibSonomy über fast die gleichen Funktionen und 
Möglichkeiten verfügen, lassen sich die Beispiele aus dem Buch mit leichten 
Abweichungen auch auf BibSonomy übertragen.\newline
Besonders geeignet ist PUMA für
\begin{itemize}
\item Forschende, die ihre eigenen Publikationslisten verwalten.
\item Mitarbeitende, die Publikationslisten von Projekten, 
Instituten oder Fakultäten pflegen.
\item Studierende, die Material für Examensarbeiten verwalten 
möchten.
\item Autorinnen und Autoren, die ihre Veröffentlichungen der Unibibliografie 
melden möchten.
\item Studierende sowie Wissenschaftlerinnen und Wissenschaftler, 
die in Arbeitsgruppen Literatur teilen möchten.
\end{itemize}
\section{Typische Anwendungsbeispiele für PUMA}
PUMA ist für die Nutzung im akademischen Bereich entwickelt worden.
Es hilft bei Literaturrecherchen für eine Haus-, Bachelor- oder Masterarbeit, 
indem die recherchierte Literatur in PUMA gespeichert werden kann. Webseiten und 
Publikationen können mittels einer Schaltfläche (Bookmarklet) im Browser direkt 
in PUMA abspeichert werden. Am Ende der Arbeit hilft PUMA dabei das 
Literaturverzeichnis zu erstellen. Bei der Erstellung des Verzeichnisses kann 
aus 7.500 Zitationsstilen der passende ausgewählt werden oder auch eine 
individuelle Anpassung per Citation Style Language (CSL) vorgenommen werden.
\newline 
Eigene Veröffentlichungen können mit Hilfe von PUMA gepflegt und  durch den Tag 
\enquote{myown} gekennzeichnet werden. Dies vereinfacht das Erstellen einer 
Publikationsliste der eigenen Veröffentlichungen. Mit Hilfe des 
OpenCMS-Plugins kann die Publikationsliste direkt zum Beispiel auf der eigenen 
Homepage veröffentlicht werden und ist so für alle sichtbar.
\newline 
Ein weiteres typisches Anwendungsfeld bilden Institutspublikationslisten. 
Durch Erstellen einer Gruppe in PUMA kann eine gemeinsame Sammlung von 
Publikationen angelegt werden. Mit Hilfe des Plugins 
\enquote{Publikationsliste (aus BibSonomy/PUMA)} \index{Plugin für 
OpenCMS} kann eine Publikationsliste aus dieser 
Sammlung erzeugt werden.

   
\section{Anmelden\index{Anmeldung} bei PUMA} 
\begin{figure}[h!]
 \centering
 \fbox{\includegraphics[width=10cm]{Bilder/Kapitel1/Startseite_PUMA}}
 \caption{Startseite PUMA}
 \label{figure001}
\end{figure}
\textbf{Vorab:} Es wird ein st-, fn- oder ac-Konto der Universität Stuttgart benötigt.
\begin{enumerate}
    \item Anmeldung über die PUMA-Homepage:\newline \url{https://puma.ub.uni-stuttgart.de/}
    \item Das ac- oder st-Konto der Universität Stuttgart und das Passwort eingeben (seltener ist ein fn-Konto).  
    \item Bei der Erstanmeldung muss ein selbstgewählter Benutzername vergeben werden.
\end{enumerate}
 \begin{figure}[h!]
 \centering
 \fbox{\includegraphics[width=9cm]{Bilder/Kapitel1/Anmeldung_bei_PUMA}}
 \caption{Anmeldung bei PUMA}
 \label{figure002}
\end{figure} 
Neben Publikationseinträgen ist der Benutzername öffentlich sichtbar (Grundeinstellung). Die Sichtbarkeit kann bei jedem Eintrag eingestellt werden. Der Benutzername wird mit einem vorangestellten \@-Zeichen dargestellt.
\section{Der PUMA-Blog}
Die Universitätsbibliothek Stuttgart (UB) informiert auf ihrem Blog über PUMA-Updates und neue Funktionen. Mit Hilfe eines RSS-Feeds können Interessierte die PUMA-Nachrichten abonnieren. Über den Link: \newline \url{http://blog.ub.uni-stuttgart.de/category/puma/feed/} werden die aktuellen Informationen angezeigt. \footnote{RSS ist ein einfaches Anzeigeformat für Internetnachrichten. Über \enquote{Jetzt abonnieren} auf der Blog-Seite der UB öffnet sich ein Fenster, indem das Abonnieren nochmals bestätigt werden muss. Ab sofort können die Informationen über die Lesezeichen-Leiste im Browser angezeigt werden.}
 \begin{figure}[h!]
 \centering
 \fbox{\includegraphics[width=9cm]{Bilder/Kapitel1/Anmeldung_Blog}}
 \caption{Anmeldung für RSS}
 \label{Anmeldung für RSS}
\end{figure} 
\section{BibSonomy\index{BibSonomy}}
BiBSonomy wurde von einem Team von Studierenden und Wissenschaftlern vom Fachgebiet Wissensverarbeitung der Universität Kassel entwickelt und bereitgestellt. In Zusammenarbeit mit der Universitätsbibliothek Kassel wurde die auf BibSonomy basierende Online-Literaturverwaltung PUMA entwickelt. Daher kann BiBSonomy weltweit nach einer Anmeldung \newline \url{http://www.bibsonomy.org/?lang=de} am System genutzt werden. Einträge aus PUMA können einfach nach BiBSonomy exportiert werden.

%\section{BibSonomy\index{BibSonomy} vs. PUMA}
%\suppressfloats[t]
\begin{table}[h!]
\tabulinesep=1.5mm
\begin{tabu}{|X[1.4,c]|X[2.2,m]|X[2,m]|} 
\tabucline[0.5pt]-\everyrow{\tabucline[0.5pt]-} 
\rowfont\bfseries
Unterschiede & PUMA \emph{Uni Stuttgart} & BibSonomy\\ \tabucline[1pt]-
\bfseries{Anmeldung}\strut & Nur möglich mit einem st-, fn- oder ac-Konto der Universität, mit dem sich die Nutzer authentifizieren.  & Für jeden frei zugänglich, ein Benutzerkonto muss selber angelegt werden. \\ 
\bfseries{Gruppen}\index{Gruppen} & Gruppen können jederzeit und selbständig gegründet werden. & Die Gründung einer Gruppe erfordert die Freigabe des BibSonomy-Admins. \\
\bfseries{OPUS}\index{OPUS} & Schnittstelle zu OPUS geplant. Ermöglicht den Nutzern ein direktes Veröffentlichen auf dem Dokumentenserver OPUS. & \\ 
\bfseries{Unibibliografie}\index{Unibibliografie}& Weiterverwendung von Metadaten aus der Unibibliografie möglich.&\everyrow{} \\ \tabucline[1.0pt]-
\end{tabu}
\caption{Unterschiede zwischen PUMA und BibSonomy}
\end{table}
%\normalsize
\section{Schnelleinstieg}
\textit{Sammeln, Verwalten und teilen- so lautet die Devise.}
\subsection{Sammeln}
Bookmarks, bibliografische Daten und Volltexte kostenlos online speichern.\newline\newline
\textbf{Login}
\begin{itemize}
\item Sichere Verbindung: \url{https://puma.ub.uni-stuttgart.de/}
\item Angehörige der Universität Stuttgart: st- oder ac-Konto
\item Benutzername frei wählbar
\item Keine Lizenzgebühr
\end{itemize}
\textbf{Die ersten Schritte}
\begin{itemize}
\item Publikationen manuell eintragen oder per ISBN/ DOI automatisch aus Datenbanken abrufen
\item Vergabe von Schlagwörtern (\enquote{Tags})
\item Mindestens ein \enquote{Tag} ist obligatorisch
\item Sichtbarkeit einstellen: privat, öffentlich oder für eine Gruppe
\item Ausführliche Anleitung: PUMA-Hilfe im Benutzermenü
\item Schulung für Einsteiger in der Universitätsbibliothek Stuttgart: \url{www.ub.uni-stuttgart.de/puma}
\end{itemize}
\textbf{Schlagworte (\enquote{Tags}) nutzen}
\begin{itemize}
\item Ausgabe von eigenen Publikationslisten per Systemtag \textit{myown}
\item Beliebige Sortiermöglichkeiten über Vergabe von Tags
\item Kombination von Tags zur Eingrenzung von Abfragen
\end{itemize}
\subsection{Verwalten}
Literaturlisten in unterschiedlichen Zitationsstilen erzeugen und exportieren.\newline\newline
\textbf{Import und Export von Daten}
\begin{itemize}
\item Vorhandene Publikationslisten importieren (BibTex oder Endnote-Formate)
\item Duplikatserkennung
\item Ausgabe von Literaturlisten in rund 7.500 Zitationsstilen
\item Oder individuelle Anpassung per Citation Style Language (CSL)
\item Export der Daten in Standardformate wie BibTex, Endnote oder XML
\item Viele Sortiermöglichkeiten nach Autor, Jahr, Dokumententyp u.a.
\end{itemize}
\subsection{Teilen}
Einträge per Plugin auf der Website veröffentlichen oder in Gruppen Zugänglich machen.\newline\newline
\textbf{Soziale Funktion}
\begin{itemize}
\item Gruppen anlegen und Mitglieder veralten
\item Sichtbarkeit auf die Gruppe beschränken
\item Gemeinsam Einträge bearbeiten (Community-Posts)
\item Freunden folgen und Einträge senden
\end{itemize}
\textbf{Plugin für OpenCMS der Universität Stuttgart}
\begin{itemize}
\item Verbindung mit dem PUMA-Benutzerkonto
\item Eigene Publikationslisten aus PUMA laden
\item Dokumentation im Typkatalog des TIK: \url{http://www.tik.uni-stuttgart.de/dienste/opencms/typkatalog/typ/PumaPublicationList/}
\end{itemize} 
\textbf{Typo3 Plugin}
\begin{itemize}
\item Verbindung mit dem PUMA-Benutzerkonto (Funktionsweise wie bei OpenCMS)
\item Eigene Publikationen aus PUMA laden und auf der eigenen Homepage veröffentlichen
\item Einfügen von eigenen CSL-Styles möglich
\end{itemize}
\chapter{Typische Anwendungsbeispiele}
\label{ch:typischeAnwendungsbeispiele}

\section{Institutspublikationsliste}
\label{sec:institutspublikationsliste}
Für die Veröffentlichung der Institutspublikationsliste (\autoref{subsec:opencms}) gibt es mehrere Möglichkeiten:
\begin{itemize}
\item Über die Gruppenfunktion (\autoref{sec:gruppen}) von PUMA: Diese Funktion ermöglicht, gemeinsam ein PUMA-Gruppenkonto zu verwalten. Gruppenadmins können Einträge bearbeiten, die Mitglieder der Gruppe an diese per Systemtag \textit{for:Gruppenname} (\autoref{systemtag}) geschickt haben. Damit ist der Eintrag im Besitz der Gruppe (\textit{@Gruppenname}) und des Gruppenmitglieds. Dieser Weg hat mehrere Vorteile: Zum einen können mehrere Personen Einträge bearbeiten. Zum anderen können in der Gruppe gleichzeitig auch Volltexte geteilt und Dubletten schneller erkannt werden. Aus dieser  Sammlung kann wiederum eine gemeinsame Institutspublikationsliste generiert werden.
\item Über ein Funktionskonto (fn-Konto): Das fn-Konto muss beim TIK über den Benutzerverwaltungsadministrator des Instituts beantragt werden. Dieses Konto funktioniert wie das st- oder ac-Konto, es hat den Vorteil, dass es nicht an eine Person gebunden ist und kann von verschiedenen Personen bedient werden. Aus der Sammlung kann eine gemeinsame Institutspublikationsliste generiert werden.
\item Über ein persönliches Konto werden die Publikationen gesammelt und in PUMA eingetragen. Aus der eigenen Sammlung wird dann die Institutspublikationsliste erstellt.
\end{itemize}
Scheidet ein Institutsmitarbeiter der Universität Stuttgart aus, bleibt sein Benutzerkonto bei PUMA auch nach Ausscheiden (\autoref{subsec:kontoAufloesen}) erhalten. Eine Authentifizierung ist nicht mehr möglich. Damit können diese Einträge vom Nutzer nicht mehr bearbeitet werden. Öffentlich geteilte Einträge des Ausgeschiedenen bleiben weiterhin nutzbar.\\
PUMA-Nutzer können ihr Konto jederzeit löschen (\autoref{subsec:kontoLoeschen}). Damit sind ihre Einträge ebenfalls gelöscht. Wenn die Publikationen am Institut erhalten bleiben sollen, kann ein anderer Nutzer des Instituts die Einträge vorher in die eigene Sammlung kopieren oder der Nutzer sendet per Systemtag \textit{for:Gruppenname} (\autoref{systemtag}) seine Publikationseinträge an eine Institutsgruppe. Die Pflege einer Institutspublikationsliste sollte aber möglichst von vorneherein über die Gruppenfunktion oder ein fn-Konto erfolgen.

\section{Eigene Publikationslisten verwalten und veröffentlichen}
\label{sec:eigenePublistenVerwalten}
Das Verwalten und Sortieren der eigenen Publikationen gestaltet sich mit PUMA einfach. Eigene  Publikationen können in PUMA gekennzeichnet werden. Nutzer können sich als Autoren der Publikation zu erkennen geben, indem beim Publikationseintrag ein Häkchen bei \textit{Ich bin (Mit-)Autor} gesetzt wird. Dies erzeugt automatisiert den Systemtag \textit{myown}. Systemtags sind \enquote{Tags}, die einheitlich definiert sind und weitere Funktionen beinhalten (\autoref{systemtag}).\\
Mit Hilfe des Plugins \enquote{Publikationsliste (aus BibSonomy/PUMA)} für OpenCms (\autoref{subsec:opencms}) kann die Publikationsliste auf einer Internetseite der Universität Stuttgart veröffentlicht werden. Das Plugin wird über das Zauberstab-Menü im Seiteneditor ausgewählt. Durch Verbindung mit dem PUMA-Benutzerkonto über den API-Schlüssel und Auswahl des eigenen Benutzerkontos werden alle Einträge, die mit diesem Konto bibliografiert wurden, angezeigt. Mit der Eingabe des \enquote{Tags} \textit{myown} wird nur die eigene Publikationsliste veröffentlicht.



\section{Materialsammlung für die Bachelorarbeit}
\label{sec:materialsammlungBachelorarbeit}
Um den Überblick über die Materialrecherche nicht zu verlieren, empfiehlt es sich spätestens bei der Bachelorarbeit den Umgang mit Literaturverwaltungsprogrammen zu üben. Bei der Literatursuche im Internet und in Bibliothekskatalogen unterstützt PUMA im Browser das Sammeln von Lesezeichen und Publikationen, ohne dass die Suche unterbrochen werden muss. Mit Unterstützung des PUMA-Add-ons für den Firefox (siehe \autoref{sec:addon}) oder Bookmarklet-Buttons (siehe \autoref{sec:button}) können diese in die eigenen PUMA-Sammlung eingetragen werden.\\
Es kommt nicht selten vor, dass einem bei der Menge an gesammelten Publikationen und Lesezeichen der Überblick verloren geht. In PUMA werden Materialsammlungen mit \enquote{Tags} (\autoref{subsec:tags}) strukturiert. Der Nutzer gibt diese \enquote{Tags} beim Eintragen mit an. Es gibt keine Beschränkung was die Anzahl der Tags betrifft. Die Vergabe der Tags erleichtert einem später das Finden der passenden Literatur zu einem Thema in der eigenen Sammlung. Wird nach einem Tag gesucht, zeigt PUMA alle Einträge mit diesem \enquote{Tag} an. So unterstützt PUMA das Entstehen einer strukturierten Literaturliste.\\
Im nächsten Schritt wird die Literatursammlung in Word importiert (siehe \autoref{importWord}). LaTeX-Nutzer können eine BibTeX-Datei aus PUMA exportiert. Beim Schreiben kann die Literatur dann referenziert und ein Literaturverzeichnis erzeugt werden.

\section{Aufbau}
\textit{Ein Blick genügt: Der Aufbau von PUMA ist durchschaubar. Durch ausprobieren lernt man ihn am Besten kennen.}
\begin{figure}[h!]
 \centering
 \fbox{\includegraphics[width=12cm]{Bilder/Kapitel4/Puma_Hauptmenue}}
 \caption{Hauptmenü}
 \label{figure3}
\end{figure} 
\subsection{Suchleiste}
Die Suche\index{Suche} bei PUMA bietet viele Möglichkeiten den Datensatz nach unterschiedlichen Informationen zu durchsuchen. Durch das Klicken auf den kleinen Pfeil neben \enquote{Suche}\index{Suche} erscheint ein Dropdown-Menü.  Der zu durchsuchende Datensatz wird durch anklicken ausgewählt. Anschließend wird der Suchbegriff in das weiße Feld daneben eingegeben. Durch das Klicken auf \enquote{Suche} oder \enquote{Enter} gibt PUMA in wenigen Sekunden die entsprechende Ergebnisse aus.

\begin{figure}[h!]
 \centering
 \fbox{\includegraphics[width=13cm]{Bilder/Kapitel4/Suchleiste}}
 \caption{Suchleiste}
 \label{figure3}
\end{figure}  

\subsection{Spracheinstellung}
Hier haben Sie die Möglichkeit zwischen den drei verfügbaren Sprachen\index{Sprachen} in PUMA zu wechseln. Es gibt die Möglichkeiten zwischen Englisch (en), Deutsch (de) und Russisch (ru) zu wählen.
%\begin{wrapfigure}{l}{5cm}
\begin{mdframed}[style=mdfexample1,frametitle={\texttt{\textbf{TIPP}}},backgroundcolor=gray!40] \texttt{Um die Sprache für die Seite festzulegen, sodass diese bei jedem neuen Besuch bei PUMA gleich ist, müssen Sie dies in den Einstellungen festlegen. Dorthin gelangen Sie über das Personensymbol rechts im Hauptmenü. Klicken Sie auf den Reiter \enquote{Einstellungen} und die Seite mit den Einstellungen öffnet sich. Anschließend klicken Sie auf die Rubrik \enquote{Einstellungen}. Auf der erscheinenden Seite können Sie nun die gewünschte Sprache festlegen und müssen anschließend auf \enquote{Layout speichern} die Änderung bestätigen.}
\end{mdframed}
%\end{wrapfigure}
  

\subsection{Linkes Hauptmenü} 
Das Hauptmenü stellt die wichtigsten Funktionen von PUMA bereit. Es ist zu beachten, dass sich einige Einstellungen unterscheiden, da es bei PUMA die Unterscheidung zwischen einfachen Funktionen\index{Funktionen!Einfache} und erweiterten Funktionen\index{Funktionen!Erweiterte} gibt. Dies betrifft vor allem den unten genannten Punkt \enquote{Mein PUMA}. 
\subsubsection{Home\index{Home}}
Damit gelangen Sie zur Startseite und erhalten einen Überblick über Publikationen und Lesezeichen, die vor kurzer Zeit eingetragen wurden. In jeder der beiden Spalten können Sie separate Einstellungen für die Publikationen oder Lesezeichen vornehmen. Mit dem schwarzen Stift können alle eigenen Einträge in der Liste bearbeitet werden, der Trichter ermöglicht es ihnen, die angezeigten Einträge zu filtern, mit dem Pfeil nach unten können Sie die Reihenfolge und Sortierung der angezeigten Einträge ändern und mit dem Exportzeichen können Sie die Exportmöglichkeiten für die angezeigten Einträge festlegen.
\subsubsection{Mein PUMA\index{Mein PUMA}}
Es ist zu beachten, dass sich einige Einstellungen unterscheiden, da es bei PUMA die Unterscheidung zwischen einfachen Funktionen\index{Funktionen} und erweiterten Funktionen gibt. Durch das Freischalten der erweiterten Funktionen kommen weitere Funktionen hinzu.
\begin{enumerate}
    \item Einfache Funktion\index{Funktionen!Einfache}:
    \begin{itemize}
        \item meine Einträge\index{Einträge}: Hier gelangen Sie zu dem eigenen Publikations- und Lesezeichenverzeichnis, das Sie sich angelegt haben.
        \item Diskutierte Einträge\index{Einträge!diskutierte}: Hier finden Sie alle Publikationen und Lesezeichen, die Sie selber bewertet haben. Ebenso werden Kommentare und Rezessionen von anderen Nutzern zu Ihren Publikationen hier angezeigt.
        \item verfolgte Einträge\index{Einträge!verfolgen}: Hier werden Ihnen alle Publikationen und Lesezeichen der Nutzer angezeigt, denen Sie folgen.
        \item Einträge von Freunden\index{Einträge!von Freunden}: Es werden Ihnen alle Einträge Ihrer Freunde angezeigt.
    \end{itemize}
    \item Erweiterte Funktionen\index{Funktionen!Erweiterte}:
    \begin{itemize}
        \item Private Einträge\index{Einträge!privat}: Damit gelangen Sie zu Ihren Einträgen, die nur für Sie sichtbar sind. 
        \item Einträge für Freunde\index{Einträge!Freunde}: Zeigt die Einträge, die nur Sie selber und Ihre Freunde sehen können.
        \item Dokumente\index{Dokumente}: Wenn Sie Ihren Einträgen Dokumente (z.B. eine PDF-Datei) angehängt haben, können Sie hier eine Übersicht über die angehängten Dokumente sehen.
        \item Duplikate\index{Duplikate}: Zeigt Ihnen die Einträge, die wahrscheinlich Duplikate sind. So können Sie ihre Literaturliste ganz einfach bereinigen. 
        \item Konzepte\index{Konzepte}: Konzepte ermöglichen es Ihnen mehreren Schlagwörtern zu gruppieren. 
        \item Lebenslauf\index{Konzepte}: Hier können Sie Ihre persönlichen Daten hinterlegen, welche für andere Nutzer in PUMA sichtbar sind.
        \item Publikationen durchstöbern: Mit dieser Funktion können Sie Ihre eigenen Lesezeichen/Publikationen durchstöbern. Sie erhalten so einen schnellen Überblick über den eigenen Literaturbestand. 
        \item BibTex\index{BibTex} exportieren: Exportiert Ihre Daten in das BibTex-Format.
    \end{itemize}
\end{enumerate}
\subsubsection{Eintragen}
\begin{enumerate}
    \item Lesezeichen eintragen\index{Lesezeichen!eintragen}: Fügen Sie ein neues Lesezeichen Ihrer Sammlung hinzu.  
    \item Publikation eintragen\index{Publikationen!eintragen}: Fügen Sie ein neue Publikation Ihrer Sammlung hinzu. 
    \item Lesezeichen importieren\index{Lesezeichen!Import}: Importieren Sie Lesezeichen aus Ihrem Browser oder Ihren Delicious Daten.
    \item Publikationen importieren\index{Publikationen!Import}: Importieren Sie eine bestehende BibTeX- oder EndNote-Datei in PUMA.
\end{enumerate}


\subsubsection{Gruppen}
Zeigt Ihnen die Funktionen zu Gruppen\index{Gruppen} an sowie die Gruppen in denen Sie Mitglied sind.
\begin{enumerate}
    \item Alle Gruppen: Verschafft Ihnen einen Überblick über alle existierenden Gruppen bei PUMA.
    \item Eine neue Gruppe erstellen: Bietet Ihnen die Möglichkeit eine eigene Gruppe zu erstellen.
\end{enumerate}
\subsubsection{Beliebt\index{Beliebt}}
Ermöglicht Ihnen, die derzeitig beliebtesten Einträge bei PUMA zu durchforsten.
\begin{enumerate}
    \item Einträge: Zeigt die beliebtesten Einträge an.
    \item Tags: Zeigt die beliebtesten Tags in einer Schlagwortwolke an. Je größer ein Tag ist, desto beliebter ist er.
    \item Autor: Zeigt die beliebtesten Autoren an.
    \item Konzepte: Zeigt die beliebtesten Konzepte und deren Zuordnungen an. 
    \item Diskussionen: Zeigt Lesezeichen und Publikationen an, über welche viel diskutiert wurde. 
\end{enumerate}
\subsubsection{Genealogie}
Die PUMA Genealogie erstellt nutzerbasiert einen Stammbaum der Forschung an deutschen Universitäten. Ausgangspunkt ist der Dissertationskatalog der Deutschen Nationalbibliothek. Nutzerbasiert werden Beziehungen zwischen den an der Dissertation beteiligten Personen (Autor\_in, Betreuer\_in etc.) ergänzt.

\subsection{Rechtes Hauptmenü}
\subsubsection{@username\index{@username}}
Über diesen Button gelangen Sie zu Ihrer Publikations- und Lesezeichensammlung. 
\subsubsection{Das Personensymbol}
\begin{enumerate}
    \item Eingang\index{Eingang}: Dies ist Ihr Lesenzeichen-/Publikations-Posteingang. Freunde/Gruppen können Ihnen Publikationen und Lesezeichen zuschicken, diese Eingänge landen dann hier.
    \item Ablage\index{Ablage}: In der Ablage können Sie aktuelle Literaturlisten zusammenstellen. 
    \item Freunde\index{Freunde}: Hier erhalten Sie einen Überblick über Ihre Freunde. 
    \item Tags bearbeiten\index{Tags!bearbeiten}: Hier können Sie Tags und Konzepte überarbeiten, beispielsweise alte Tags durch neue ersetzten. 
    \item Einstellungen\index{Einstellungen}: Zeigt Ihre persönlichen Benutzereinstellungen an. Sie können hier Ihr Profil, die allgemeinen Einstellungen, Ihren Lebenslauf sowie Einstellungen zu Gruppen ändern.
    \item Weblog\index{Weblog}: Leitet Sie zu dem Weblog\footnote{\url{http://blog.ub.uni-stuttgart.de/category/puma/}} von PUMA weiter.
    \item Hilfe\index{Hilfe}: Damit gelangen Sie zur Online-Hilfe.
    \item Abmelden\index{Abmelden}: Wenn Sie PUMA verlassen wollen, melden Sie sich hier ab. 
\end{enumerate}

\textbf{Der PUMA-Aufbau im Überblick}
%\small 
\begin{longtabu}{|X|X|X|X|}\hline 
\bfseries Hauptmenü&\bfseries Untermenü&Reiter &\bfseries Rubrik\\  \hline
Eintragen&Lesezeichen eintragen &- &-\\ \cline{2-4}
&Publikation eintragen &Per Hand& Trage Sie Ihre Publikation hier ein\\\cline{3-4}
&&BibTex/EndNote-Schnipsel& Tragen Sie hier Ihre BibTex- oder EndNote-Schnipsel ein\\ \cline{4-4}
&&& Einstellungen\\ \cline{3-4}
&&Datei hochladen& Laden Sie ihre BibTex- oder EndNote-Datei hier hoch\\ \cline{4-4}
&&&Einstellungen\\ \cline{3-4}
&&ISBN/DOI& ISBN\\ \cline{4-4}
&&& ISSN \\ \cline{4-4}
&&&DOI\\ \cline{3-4}
&&Code scannen&-\\ \cline{2-4}
&Lesezeichen importieren&- &Importieren Sie Ihre Lesezeichen aus Ihrem Browser\\ \cline{4-4}
&& & Importieren Sie Ihre Delicious Daten\\ \cline{2-4} 
&Publikationen importieren &-&-\\ \hline
Gruppen&Alle Gruppen& Gruppen von A-Z &-\\ \cline{2-4}
&Gruppen, in denen der Nutzer Mitglied ist& Einträge &-\\\cline{3-4}
&&Interessant für &- \\ \cline{3-4}
&&Sichtbar&-\\ \cline{3-4}
&&Dokumente&-\\ \cline{3-4}
&&diskutierte Einträge&-\\ \cline{3-4}
&&Einstellungen&Gruppeneinstellungen und Mitgliederliste\\ \cline{2-4}
&Eine neue Gruppe erstellen&-&-\\ \hline
Personensymbol&Eingang&-&-\\ \cline{2-4}
&Ablage&-&-\\ \cline{2-4}
&Freunde&Ihre Freunde&- \\ \cline{3-4}
&&Sie sind ein Freund von&- \\ \cline{2-4}
&Tags bearbeiten&-&Umbenennen/ Ersetzen von Tags\\ \cline{4-4}
&&&Subtags zu Konzepten hinzufügen\\ \cline{4-4}
&&&Subtags von Konzept löschen\\ \cline{2-4}
&Einstellungen&Mein Profil&Allgemeine Informationen\\ \cline{4-4}
&&&Kontakt\\ \cline{4-4}
&&&Über mich\\ \cline{4-4}
&&&Ein Bild für meinen Lebenslauf\\\cline{3-4}
&&Einstellungen&Layouts Ihrer Tagbox und Ihrer Eintragsbilder\\\cline{4-4}
&&&API\\\cline{4-4}
&&&Logging und Löschen\\\cline{4-4}
&&&Passwort ändern\\\cline{4-4}
&&&Mein Konto löschen\\\cline{3-4}
&&JabRef Layout-Datei&-\\\cline{3-4}
&&Lebenslauf& Lebenslauf editieren\\\cline{4-4}
&&&layout editieren\\ \cline{4-4}
&&&Vorschau wird erzeugt\\\cline{3-4}
&&OAuth-Consumers&- \\\cline{3-4}
&&Gruppen&Eine neue Gruppe erstellen\\\cline{4-4}
&&&Gruppen\\\cline{2-4}
&Einstellungen&Synchronisation&-\\\cline{2-4}
&Weblog&-&-\\\cline{2-4}
&Hilfe&-&-\\\cline{2-4}
&Abmelden&-&-\\\hline
\end{longtabu}
%\normalsize

\subsection{Inhaltsbereich}
Hier sehen Sie die aktuellsten Lesezeichen und Publikationen von Ihnen und anderen Nutzern. 
\subsection{Beliebte Tags}
Zeigt Ihnen die beliebtesten Tags\index{Tags} an. Sie können zwischen der Wolken- oder Listen-Ansicht wählen.
\chapter{Basics}
\label{ch:basics}
\section{Einstellungen}
\label{sec:einstellungen}
Über das Personensymbol (\autoref{subsec:Personensymbol}) gelangt man zu den Einstellungen\index{Einstellungen} indem im Dropdown-Menü der entsprechende Reiter ausgewählt wird. In den Einstellungen kann zwischen unterschiedlichen Reitern gewählt werden:
\begin{description}
   \item[Mein Profil:]\index{Mein Profil}
	\item[]
Hier können persönliche Informationen ergänzt werden, die auf der Seite \enquote{meinLebenslauf} dargestellt werden. Im Auswahlmenü \enquote{Profil einsehbar für} wird eingestellt, wer diese Informationen einsehen kann (privat, Freunde, öffentlich). Hier kann auch der Link für die OpenUrl (\autoref{subsec:OpenURL}) hinterlegt werden. \index{OpenURL}
   \item[Einstellungen:] 
   \begin{itemize}
      \item[] % schönerer Umbruch
      \item Anzeige der \tag-Auswahl ändern: Liste oder Wolke
      \item Anzeige der \tag-Reihenfolge ändern: Alphabet oder Häufigkeit
      \item \tag-Tipps: Funktion ist nicht mehr vorhanden
      \item \tag-Auswahl: Top X oder mind. Häufigkeit. Bei TOP X werden nur die X häufigsten \tags angezeigt, bei min. Häufigkeit werden nur die \tags angezeigt die X-Mal vorkommen.
      \item Schrankenwert: Einstellung von X
      \item Einträge pro Seite: Kann bis 10000 hoch gesetzt werden, um im persönlichen Konto die Einträge sortieren zu können. Das Laden dieser Einträge dauert allerdings sehr lange. Empfehlung: Die Sortierung kann über zwei andere Wege gelöst werden \todo{Sortierung referenzieren und an anderer Stelle beschreiben}
      \item \enquote{Bevorzugte Exportformate}: Auswahl von Zitationsstilen, die beim Export als erstes angezeigt werden sollen.
      \item Erscheinungsbild: erweitert oder einfach. Beim einfachen werden einige Funktionen, die nicht unbedingt benötigt werden nicht angezeigt: zum Beispiel werden unter \enquote{mein PUMA} mehr Auswahlfelder angeboten.
      \item Sprachauswahl: deutsch, englisch oder russisch
      \item API-Schlüssel\index{API-Schlüssel}: wird benötigt, um von externen Programmen auf PUMA zuzugreifen. Zum Beispiel für die Einbindung der Literaturliste in OpenCMS. Alternativer Weg über OAUTH \todo[inline]{Referenz setzen und an anderer Stelle beschreiben}
      \item Klick-Aufzeichnungen erlauben: Klicks auf externe Links werden aufgezeichnet, um wissenschaftlich Auswertungen vornehmen zu können ~\autoref{ch:pumaForschungsprojekt}.
      \item Bestätigung vor Löschen: Wenn ein Eintrag gelöscht wird, kann hier ein eingestellt werden, ob noch einmal vor dem löschen nachgefragt werden soll.
      \item PUMA-Konto löschen: hier kann das gesamte Profil gelöscht werden. Aus Sicherheitsgründen kann der Name eines einmal gelöschtes Kontos nicht wieder vergeben werden.
			\end{itemize}
   \item[JabRef Layout-Datei:]\index{JabRef! Layout-Datei} 
	In diesem Reiter können JabRef-Layout-Dateien hochgeladen werden, um Publikationslisten nach eigenen Wünschen darzustellen. Dazu eine einzelne oder die in drei Teile vorhandene Layout-Datei hochladen, und abschicken. Das Layout kann dann unter dem PUMA-Link im Text geöffnet werden. %Dies ist eine Alternative zur \href{http://citationstyles.org/}{Citation Style Language}. PUMA bietet bereits viele dieser Exportmöglichkeiten (~\autoref{ch:exportImport}) an.\footnote{Unter \url{https://github.com/JabRef/layouts.jabref.org} werden weitere Stile angeboten.} Wenn ein Stil für einen erweiterten Personenkreis benötigt wird, kann dieser auf Anfrage in den allgemeinen Export aufgenommen werden.
\begin{tip} Die Jabref-Layout Dateien werden von PUMA nicht mehr weiter entwickelt. Als Alternative wird die die \href{http://citationstyles.org/}{Citation Style Language} angeboten.
\end{tip}
\todo{verweis einfügen und Abscnhitt über csl schreiben}
\item[Lebenslauf\index{Lebenslauf}]
Daten aus dem persönlichen Profil werden anderen Nutzern (Personenkreis je nach Einstellung in \enquote{meinProfil}) in dem hier eingestellten Layout dargestellt. Es kann zwischen unterschiedlichen Layouts ausgewählt werden oder ein eigenes mit Hilfe der MediaWiki-Syntax\footnote{\url{https://en.wikipedia.org/wiki/Help:Wiki_markup}}erstellt werden (~\autoref{sec:cv}). Unter \enquote{meine Publikationen} bzw. \enquote{meine Lesezeichen} werden alle Einträge, die mit \enquote{myown} getagt wurden, angezeigt.
   \item[OAuth-Consumers:\index{OAuth}]
\todo[inline]{\url{https://www.bibsonomy.org/help_de/OAuth}, ist das in PUMA auch möglich? testen!}
   \item[Gruppen:\index{Gruppen}]
Überblick über alle Gruppen, in denen man Mitglied ist. Über \enquote{teilen} kann eingestellt werden, ob die eigenen Dokumente geteilt werden sollen. Die prinzipielle Möglichkeit Dokumente zu in einer Gruppe zu teilen kann der Gruppenadmin über \enquote{Einstellungen bearbeiten} treffen, dort können auch weitere Gruppeneinstellungen getroffen werden. Auch Einladungen an andere PUMA-Nutzer für einen Gruppe können hier versendet werden (VGl. auch ~\autoref{sec:gruppen}.  \todo[inline]{Verweis oder die Gruppeneinstellungen noch näher beschreiben (Einstellungen, Mitgliederliste, Lebenslauf, Gruppe löschen)}
   \item[Synchronisation:]
\todo[inline]{Es sind keine Synchronisationsclients oder Server für dieses System konfiguriert. $\to$ bei Mario nachfragen}
\end{description}
%\section{Lebenslauf}
%\label{sec:cv}
%Über \enquote{meinPuma} $\to$ \enquote{Lebenslauf} kann über das Zahnradsymbol sowohl auf das eigene Profil als auch auf die Einstellungen des Layouts (~\hyperref{subseceigenesLayout}) des Lebenslaufs zugegriffen werden. Das eigene Profil kann mit persönlichen Informationen über den Ort, das Geburtsdatum, Beruf und Institution erweitert werden. Darüber hinaus können die wissenschaftlichen Interessen und Hobbies eingetragen werden. Publikationen und Lesezeichen, die mit \tag \textit{<myown />} getaggt werden, erscheinen ebenfalls auf dieser Seite. In den Profileinstellungen kann die Sichtbarkeit dieser Informationen auf privat, für Freunde oder öffentlich sichtbar eingestellt werden. Bei privat sehen andere Nutzer nur den Nutzernamen.
%%\begin{figure}[h!]
 %%\centering
 %%\fbox{\includegraphics[width=11cm]{Bilder/Kapitel5/CV_Benutzerkonto}}
 %%\caption{Das Benutzerkonto}
 %%\label{fig:benutzerkonto}
%%\end{figure}  
%%Lebenslauf bearbeiten\index{Lebenslauf!bearbeiten}: 
%%\begin{enumerate}
    %%\item Klicken Sie auf Ihren Benutzernamen (@username).
    %%\item Klicken Sie rechts neben Ihrem Profilbild auf den CV-Button.
    %%\item Ihr Lebenslauf öffnet sich (Ansicht: So wie ihn andere Nutzer sehen).
    %%\item Um den Lebenslauf zu bearbeiten klicken Sie auf das schwarze Zahnrad neben \enquote{Curriculum Vitae}.
    %%\item Klicken Sie anschließend im Untermenü auf \enquote{Lebenslauf bearbeiten}. Sie können nun ein vordefiniertes Layout auswählen oder selbst ein Layout mit der MediaWiki-Syntax definieren.
%%\end{enumerate}
%%\textbf{Alternativer Weg:} 
%%\begin{enumerate}
    %%\item Klicken Sie auf das Personensymbol. Es öffnet sich ein Untermenü.
    %%\item Klicken Sie im Untermenü auf \enquote{Einstellungen}.
    %%\item Eine neue Seite öffnet sich. Klicken Sie auf den Reiter \enquote{Lebenslauf}. Sie können nun ein vordefiniertes Layout auswählen oder selber ein Layout mit der MediaWiki-Syntax definieren.
%%\end{enumerate}
%%\begin{wrapfigure}{l}{5cm}
%\begin{mdframed}[style=tipp]\texttt{Wenn Sie zwischen den vordefinierten Layouts wechseln, geht Ihr selbst definiertes Layout verloren.} \
%\end{mdframed}
%\todo{vielleicht zwei Styles Achtung und Tipp: Bin insgesamt mit dem Layout dieser Kästen nicht zufrieden}
%%\end{wrapfigure}
%\begin{figure}[h!]
 %\centering
 %\fbox{\includegraphics[width=11cm]{Bilder/Kapitel5/CV_Seite}}
 %\caption{Curriculum Vitae-Seite}
 %\label{fig:curriculumVitaeSeite}
%\end{figure}
%\todo[inline]{Bild ist veraltet, neuen Screenshot}
%\subsection{Eigenes Layout\index{Lebenslauf!Eigenes Layout}:}
%\label{subsec:eigenesLayout}
%Um sich selbst ein Layout zu definieren, wird die MediaWiki-Syntax verwendet. Dazu gibt es einige XHTML-Tags:
%\begin{figure}[h!]
 %\centering
 %\fbox{\includegraphics[width=11cm]{Bilder/Kapitel5/xhtml_tags}}
 %\caption{XHTML-Tags}
 %\label{fig:xhtmlTags}
%\end{figure} % Tabelle % ist XHTML bei wiki???
%\newline
%Für die Publikations- und Lesezeichenanzeige können außerdem zusätzliche Tags angegeben werden, um diese aus der eigenen Sammlung im Lebenslauf anzeigen zu lassen. Beispielsweise liefert \textit{<publications tags=\enquote{data mining} />} alle Publikationen, die sowohl mit data als auch mit mining getaggt wurden.  
\todo[inline]{Eigener Abschnitt zu Lebenslauf auskommentiert, ist das wirklich nötig? Eventuell noch oben ergänzen}
\section{Publikationen und Lesezeichen verwalten}
\label{sec:publikationen}
 Mit PUMA können sowohl Publikationen als auch Lesezeichen verwaltet werden. Es gibt verschiedene Wege Publikationen und Lesezeichen in PUMA einzutragen. Jede Publikation bzw. jedes Lesezeichen benötigt mindestens ein \tag. Mit Hilfe dieser \tags  werden die Einträge kategorisiert. Grundsätzlich sind alle Einträge öffentlich. Öffentliche Einträge sind auch für nicht angemeldete Nutzer sichtbar. Die Sichtbarkeit kann auf privat oder andere (Gruppen oder Freunde) eingeschränkt werden. 
\subsection{Publikationen}
\begin{figure}[htb]
 \centering
 \fbox{\includegraphics[width=11cm]{Bilder/Kapitel5/Publikation_eintragen}}
 \caption{Publikationen hinzufügen}
 \label{fig:publikationenHinzufügen}
\end{figure}  
Unter dem Menü \enquote{Eintragen} $\to$ \enquote{Publikation hinzufügen}\index{Publikationen!hinzufügen} kann direkt in das Textfeld der Titel oder die ISBN/~ISSN/~DOI der Publikationen eingeben werden. Des Weiteren gibt es zusätzliche Möglichkeiten Publikationen einzutragen:
%\begin{figure}[h!]
 %\centering
 %\fbox{\includegraphics[width=11cm]{Bilder/Kapitel5/Eintragsmoeglichkeiten}}
 %\caption{Eintragsmöglichkeiten}
 %\label{fig:eintragungsmoeglichkeiten}
%\end{figure}  
    \begin{itemize}
    	\item Dokument hochladen:\newline
        Der Volltext einer Publikation kann hier hochgeladen werden. Das übliche Formular zum Eintragen (mit dem hoch geladenen Volltext) der Publikationen öffnet sich. Einige Felder sind bereits ausgefüllt und können ergänzt werden.
				\begin{tip}
Achtung Betaversion! Diese Funktion befindet sich gerade in der Entwicklung.
\end{tip}
			\item BibTex\index{BibTex}/EndNote\index{EndNote}-Schnipsel:\newline
			Kopierte Zitationen können hier eingefügt werden. Diese Informationen werden dann übernommen und können ergänzt werden.
        \item Code scannen\index{Code Scannen}: \newline
Über eine Webcam kann der Barcode gescannt werden. Ggf. muss der Zugriff von PUMA auf die Webcam erlaubt werden. Sobald der Barcode erkannt wurde, werden die Daten angezeigt und können ergänzt werden
        \item Per Hand:
				Hier kann der Eintragstyp, Titel, Autor(en), Herausgeber und Jahr der Publikation eingetragen werden. Mit \enquote{Weiter} erscheinen alle Felder der Eingabemaske. \todo{Verweis auf Eintragstypen}
    \end{itemize}
\begin{tip} Über die Url \url{https://puma.ub.uni-stuttgart.de/editPublication} kann direkt auf die erweiterte Eingabemaske zugegriffen werden
\end{tip}		
%\begin{figure}[h!]
 %\centering
 %\fbox{\includegraphics[width=11cm]{Bilder/Kapitel5/Code_scannen}}
 %\caption{Code scannen}
 %\label{fig:codeScannen}
%\end{figure}
Über \enquote{Eintragen} $\to$ \enquote{Publikation importieren} kann eine Text-Datei hochgeladen werden. Folgende Aktionen können bei den Einträgen der Datei in PUMA ausgewählt werden:
\begin{tip}
Bei EndNote wird beim Speichern eine Datei mit der Endung .enl erzeugt, diese enthält die komplette Datenbank. Für den Import in PUMA sollte ein Textdatei importiert werden. Dazu muss die Datei in EndNote nicht speichern, sondern exportieren. PUMA kann mit verschiedenen Formaten umgehen. Um BibTex zu exportieren muss der Output-Stil angepasst werden. Wenn BibTex nicht angezeigt wird, unter \enquote{Edit} $\to$ \enquote{Output-Styles} $\to$ \enquote{Open Style Manager} BibTex aus der Liste auswählen. Über \enquote{File} $\to$\enquote{Export} den \enquote{BibTex Export} auswählen. Beim Speichern \enquote{Text only} auswählen und die Datei mit der Endung .bib speichern.
\end{tip}
Nach dem Hochladen stehen verschiedene Möglichkeiten zur Verfügung die Einträge weiter zu bearbeiten:
\begin{itemize}
\item Tags zu allen ausgewählten Einträgen hinzufügen: An jeden Eintrag wird der gleiche \tag hinzugefügt
\item Die Tags aller ausgewählten Einträge separat bearbeiten: \tags können auch an einzelne Einträge hinzugefügt werden
\item BibTeX-Schlüssel normalisieren: BibTex-Schlüssel werden nach dem Schema NameJahrTitel normalisiert, wobei Name der Nachname des primären Autors , Jahr das Jahr der Veröffentlichung und Titel das erste Wort des Titels der Publikation ist, das mehr als fünf Buchstaben enthält. Beispiel: knuth1998programming.
\item Sichtbarkeit einstellen: öffentlich, privat oder für Freunde \todo{referenz und erklärung ergänzen}
\end{itemize}
\begin{tip} Einträge, die bereits in der eigenen Sammlung vorhanden sind, werden als Fehler angezeigt. Wenn alte Einträge überschrieben werden sollen, kann dies vor dem hochladen der Datei ausgewählt werden.
\end{tip}
\subsection{Lesezeichen} % 2Screenshots: Anfang+Möglichkeiten
\label{subsec:lesezeichen}
Die URL eines Lesezeichen \index{Lesezeichen!hinzufügen} kann über den Menüpunkt \enquote{Eintragen} $\to$ \enquote{Lesezeichen hinzufügen} hinzugefügt werden. 
\begin{figure}[h!]
 \centering
 \fbox{\includegraphics[width=11cm]{Bilder/Kapitel5/Lesezeichen_hinzufuegen}}
 \caption{Lesezeichen hinzufügen}
 \label{fig:lesezeichenHinzufuegen}
\end{figure}  
\underline{Wichtig bei der Recherche und Archivierung von Lesezeichen:}
\begin{itemize}
    \item Puma speichert nicht das eigentliche Dokument, sondern nur die Adresse des Internet-Dokuments. Es kann somit passieren, dass ein Dokument zu einem späteren Zeitpunkt nicht mehr abrufbar ist, da z.B. sich die Adresse geändert hat oder es gelöscht wurde.  
    \item In der Literaturangabe zu einem Internet-Dokument sollte immer das Datum des letzten Abrufs mit angegeben werden. Folgende Möglichkeiten bieten sich an dies in der Literaturliste anzupassen: Die Lesezeicheneinträge werden beim BibTeX-Export als \enquote{electronic} exportiert. Entweder diese Liste für die Literaturangaben manuell im Dokument anpassen, im dem man das \enquote{added-on} Feld in das Bibtex-Feld \enquote{urldate} umbenennt\autocite[Vgl. das Benutzerhandbuch zum BibLaTeX Paket][S.10 (der Eingabetyp \enquote{online} wird synonym zu \enquote{electronic} verwendet. Damit der Zitationsstil das Abrufdatum hinzufügt muss das Feld urldate ausgefüllt sein.)]{lehmann2016biblatex}
    \item PUMA unterstützt die RFC 7089\footnote{\url{http://tools.ietf.org/html/rfc7089}} Spezifikation\index{RFC 7089 Spezifikation}. Damit wird es möglich, Lesezeichen so zu betrachten, wie sie in PUMA gespeichert wurden, selbst wenn sich die Seite in der Zwischenzeit geändert hat. Um diese Funktion zu nutzen, müssen Sie das Memento-Plugin in ihrem Browser installieren. Das Plugin existiert für Mozilla Firefox\footnote{\url{https://addons.mozilla.org/de/firefox/addon/mementofox/}} und Google Chrome\footnote{\url{https://chrome.google.com/webstore/detail/memento-time-travel/jgbfpjledahoajcppakbgilmojkaghgm?hl=en&gl=US}}. \todo{auf Marios anwort warten}
\end{itemize}
\section{Versionierung der Publikationen und Lesezeichen}
\label{sec:versionierung}
Publikationen und Lesezeichen können jederzeit eingetragen und bearbeitet werden. Um sich einen Überblick über die vorgenommen Änderungen zu verschaffen, bietet PUMA eine Versionsgeschichte\index{Versionierung} zu jeder Publikation und jedem Lesezeichen an. Klicken Sie in Ihrer eigenen Sammlung auf den kleinen schwarzen Pfeil neben einer beliebigen Publikation. Es öffnet sich ein Dropdown-Menü, in dem Sie \enquote{Verlauf dieser Publikation} auswählen. Ihnen wird sofort die Versionsgeschichte der jeweiligen Publikation/~Lesezeichen angezeigt und Sie können jede Ihrer Änderungen nachverfolgen. 
\begin{figure}[h!]
 \centering
 \fbox{\includegraphics[width=9cm]{Bilder/Kapitel5/Versionsgeschichte}}
 \caption{Die Versionsgeschichte}
 \label{fig:versionsgeschichte}
\end{figure} 
\section{Lesezeichen importieren}
\label{sec:lesezeichenImportieren}
\subsection{Browser}
\label{subsec:browser}
PUMA ermöglicht es Ihnen HTML-Dateien in PUMA zu importieren. Hierfür exportieren Sie ihre Lesezeichen aus ihrem Browser als HTML-Datei und importieren diese anschließend. Je nach Browser unterscheidet sich das Exportieren der Lesezeichen.
\newline
\newline
\textbf{Chrome}%Screenshots hab ich schon
\newline Um ihre Lesezeichen in Chrome\index{Chrome} als HTML-Datei zu exportieren, klicken Sie im Menü oben rechts auf \enquote{Lesezeichen} und anschließend auf \enquote{Lesezeichen-Manager}. Es öffnet sich ein neues Fenster, in dem Sie auf \enquote{Organisieren} klicken und im Dropdown-Menü \enquote{Lesezeichen in HTML-Datei exportieren...} wählen. Speichern Sie die Datei ab und fahren mit Schritt 1 von HTML-Datei in PUMA importieren fort, um Ihre Lesezeichen endgültig nach PUMA zu importieren.  
\begin{figure}[h!]
 \centering
 \fbox{\includegraphics[width=11cm]{Bilder/Kapitel5/Lesezeichen-Manager_Chrome}}
 \caption{Der Lesezeichen-Manager}
 \label{fig:lesezeichenManager}
\end{figure}
\begin{figure}[ht]
 \centering
 \fbox{\includegraphics[width=9cm]{Bilder/Kapitel5/Lesezeichen_HTML_Chrome}}
 \caption{Lesezeichen in HTML-Datei exportieren}
 \label{fig:lesezeichenHtmlExportieren}
\end{figure}

\textbf{Firefox}
\newline Um Ihre Lesezeichen in Firefox\index{Firefox} als HTML-Datei zu exportieren, klicken Sie auf das Lesezeichensymbol rechts neben der Suchleiste. Wählen Sie im Dropdown-Menü \enquote{Lesezeichen verwalten} aus.

\begin{figure}[h!]
 \centering
 \fbox{\includegraphics[width=11cm]{Bilder/Kapitel5/Firefox_Lesezeichen_verwalten}}
 \caption{Lesezeichen verwalten}
 \label{fig:lesezeichenVerwalten}
\end{figure}
Anschließend klicken Sie auf \enquote{Importieren und Sichern} und wählen \enquote{Lesezeichen nach HTML exportieren} aus. Speichern Sie die Datei ab und fahren mit Schritt 1 von HTML-Datei in PUMA importieren fort, um Ihre Lesezeichen endgültig nach PUMA zu importieren.  

\begin{figure}[h!]
 \centering
 \fbox{\includegraphics[width=11cm]{Bilder/Kapitel5/Firefox_Importieren_Speichern}}
 \caption{Importieren und Sichern}
 \label{fig:importierenSichern}
\end{figure}
\subsection{HTML-Datei\index{HTML-Datei} in PUMA importieren}
\label{subsec:htmlDateiImportieren}
\begin{enumerate}
    \item Klicken Sie auf \enquote{Eintragen} und wählen im Dropdown-Menü \enquote{Lesezeichen importieren} aus.
    \item Es öffnet sich eine neue Seite. In dem Bereich \enquote{Importieren Sie Ihre Lesezeichen aus Ihrem Browser} können Sie nun die entsprechende Datei hochladen. 
    \item Legen Sie die Sichtbarkeit der Lesezeichen fest und bestätigen Sie Ihren Import anschließend mit \enquote{Importieren}.
\end{enumerate}
\begin{figure}[h!]
 \centering
 \fbox{\includegraphics[width=11cm]{Bilder/Kapitel5/HTML-Datei_hochladen}}
 \caption{HTML-Datei hochladen}
 \label{fig:htmlDateiHochladen}
\end{figure}
\subsection{Delicious}
\label{subsec:delicious}
Sie möchten Ihre Lesezeichen von Delicious\index{Delicious} nach PUMA importieren. Klicken Sie auf \enquote{Eintragen} und wählen im Dropdown-Menü \enquote{Lesezeichen importieren} aus. Geben Sie unter dem Bereich \enquote{Importieren Sie Ihre Delicious Daten} Ihre Delicious-Nutzerdaten ein. \newline
Legen Sie im darauffolgenden Schritt fest, ob Ihre Delicious Lesezeichen bereits vorhandene Lesezeichen in Ihrer Sammlung mit der selben URL überschreiben sollen.\newline
Sie können im letzten Schritt festlegen, ob Sie Ihre Lesezeichen oder Tag-Bundles importieren möchten. Wenn Sie die Option \enquote{Lesezeichen} wählen, werden zusammen mit Ihren Lesezeichen die dazugehörigen Tags und Sichtbarkeitsdefinitionen mit übernommen.
\newline Klicken Sie abschließend auf \enquote{Importieren} um den Import endgültig durchzuführen.
\begin{figure}[h!]
 \centering
 \fbox{\includegraphics[width=11cm]{Bilder/Kapitel5/Delicious_Daten}}
 \caption{Delicious Daten}
 \label{fig:deliciousDaten}
\end{figure} 
\section{Publikationen importieren}
\label{sec:publikationenImportieren}
PUMA ermöglicht Ihnen bereits bestehende BibTeX- oder EndNote-Datei hochladen. Vergewissern Sie sich hierbei, dass Sie die korrekte Kodierung gewählt haben. Falls die Datei nur wenige Einträge enthält, können Sie diese auf der folgenden Seite bearbeiten. 
\begin{enumerate}
	\item Klicken Sie auf das Feld \enquote{Durchsuchen} und laden die entsprechende Datei hoch. 
	\item Wählen Sie die Sichtbarkeit aus.
	\item In den \enquote{Erweiterte Einstellungen} können Sie anschließend noch die Datei vor dem Import bearbeiten und festlegen, ob ein älterer Eintrag überschrieben werden soll, wenn der importierte Eintrag die gleiche Publikation referenziert wie ein bereits existierender Eintrag.
	\item Klicken Sie anschließend auf \enquote{Weiter}, um den Eintrag zu vervollständigen und zu speichern.
\end{enumerate}
\begin{figure}[h!]
 \centering
 \fbox{\includegraphics[width=11cm]{Bilder/Kapitel5/Publikationen_importieren}}
 \caption{Publikationen importieren}
 \label{fig:publikationenImportieren}
\end{figure}  



  
\section{Bookmarklet-Buttons für Ihre Lesezeichen-Leiste}\label{sec:button}
Die Bookmarklet-Buttons\index{Bookmarklet-Buttons} ermöglichen Ihnen ein schnelles Arbeiten mit PUMA, während Sie im Internet unterwegs sind. Sie vereinfachen Ihnen das Eintragen von Publikationen und Lesezeichen und gelangen mit dem PUMA-Home Bookmarklet-Button direkt zu PUMA. Ziehen Sie die Buttons\footnote{\url{https://puma.ub.uni-stuttgart.de/buttons}} einfach in Ihre Lesezeichen-Leiste und schon können Sie loslegen.
\begin{figure}[h!]
 \centering
 \fbox{\includegraphics[width=11cm]{Bilder/Kapitel5/Bookmarklet-Buttons}}
 \caption{Bookmarklet-Buttons}
 \label{fig:bookmarkletButtons}
\end{figure} 
\section{PUMA-Browser-Add-ons\index{Add-ons}}\label{sec:addon}
Erweitern Sie Ihren Browser mit diesem Add-on um drei nützliche PUMA-Schaltflächen: Mit einem Klick zu PUMA, eine Publikation oder ein Lesezeichen speichern.\newline
\begin{enumerate}
\item Klicken Sie rechts oben im Firefox-Browser auf das Menü.
\begin{figure}[h!]
 \centering
 \fbox{\includegraphics[width=11cm]{Bilder/Kapitel5/Menue_Firefox}}
 \caption{Firefox-Browser}
 \label{fig:firefoxBrowser}
\end{figure} 
\item Es öffnet sich das Menü, wählen Sie den Reiter \enquote{Add-ons} aus. 
\item Geben Sie in die Suchleiste oben rechts \enquote{puma} ein.
\item Es erscheint das Add-on \enquote{PUMA Buttons}. Installieren Sie die Buttons (Version 1.6.2).
\begin{figure}[h!]
 \centering
 \fbox{\includegraphics[width=11cm]{Bilder/Kapitel5/PUMA_Buttons}}
 \caption{Puma Buttons}
 \label{fig:pumaButtons}
\end{figure} 
\item Klicken Sie anschließend auf \enquote{mehr} und scrollen auf der Seite runter bis zum Abschnitt \enquote{Instanz wechseln}. 
\item Durch das Klicken auf \enquote{Instanz wechseln} öffnet sich eine Übersicht über alle verfügbaren PUMAs. Wählen Sie  \enquote{UB Stuttgart} aus und speichern Ihre Wahl.
\item Falls die Buttons nicht sofort in der Taskleiste neben dem Menüsymbol  erscheinen, schließen Sie Firefox. Beim erneuten Öffnen des Firefox-Browsers wurden die Buttons eingerichtet.
\end{enumerate}
\section{Ablage}
\label{sec:ablage}
Die Ablage\index{Ablage} ermöglicht es Ihnen eigene und fremde Publikationen vorzumerken. Sie können so in der Ablage aktuelle Literaturlisten zusammenstellen.
\newline
Publikationen in Ablage aufnehmen: %Screenshot
\begin{enumerate}
    \item Klicken Sie auf das Symbol \enquote{Diese Publikation zur Ablage hinzufügen}.
    \item Die Publikationen gelangen direkt in die Ablage. Zur Ablage gelangen Sie über das Personensymbol.
\end{enumerate}
\begin{figure}[h!]
 \centering
 \fbox{\includegraphics[width=11cm]{Bilder/Kapitel5/Ablage}}
 \caption{Die Ablage}
 \label{fig:ablage}
\end{figure} 
Falls Sie die vorgemerkten Publikationen nicht mehr in der Ablage haben möchten können Sie diese löschen, indem Sie auf das schwarze \enquote{X} (diese Publikation aus Ihrer Sammlung löschen) klicken.\newline
%\begin{wrapfigure}{l}{5cm}
\begin{tip}Wenn Sie die Publikation in der Ablage löschen ist diese gleichzeitig auch in Ihrer Sammlung gelöscht und kann nicht wiederhergestellt werden.
\end{tip}
%\end{wrapfigure}

Eine andere Möglichkeit ist das Leeren der Ablage. Dies erreichen Sie, indem Sie auf der rechten Seite auf das blaue Feld \enquote{Ablage leeren} klicken. In diesem Fall werden die Publikationen aus der Ablage entfernt, sind aber in Ihrer Sammlung noch vorhanden.
\section{Freischalten erweiterter Funktionen}
\label{sec:freischaltenErweiterterFunktionen}
Bei PUMA gibt es die Unterscheidung zwischen einfachen\index{Funktionen!Einfache} und erweiterten Funktionen\index{Funktionen!Erweiterte}. In den Grundeinstellungen stehen jedem Nutzer, bei dessen Anmeldung bei PUMA, die einfachen Funktionen zur Verfügung. Durch das Freischalten der erweiterten Funktionen kommen weitere Funktionen hinzu, sodass Sie mehr Möglichkeiten haben, PUMA zu nutzen.  Wenn Sie die erweiterten Funktionen freischalten möchten, gehen Sie wie folgt vor:
\begin{enumerate}
    \item Klicken Sie auf das Personensymbol. Ein Dropdown-Menü öffnet sich, klicken Sie auf \enquote{Einstellungen}.
    \item Es öffnet sich die Einstellungs-Seite. Klicken Sie oben auf den Reiter \enquote{Einstellungen}.
    \item Unter dem Bereich \enquote{Layouts Ihrer Tagbox und Ihrer Eintragslisten} befindet sich das Feld \enquote{Erscheinungsbild}. Sie können nun zwischen den Standardeinstellungen \textit{Erweitert} (Alle Optionen werden stets angezeigt) oder \textit{Einfach} (Einige \enquote{Experten}-Optionen werden standardmäßig nicht angezeigt) wählen.
    \begin{figure}[h!]
 \centering
 \fbox{\includegraphics[width=11cm]{Bilder/Kapitel5/Erweiterte_Funktionen}}
 \caption{Erweiterte Funktionen}
 \label{fig:erweiterteFunktionen}
\end{figure} 
    \item Klicken Sie anschließend auf \enquote{Layout speichern}, um Ihre Änderung zu sichern.
\end{enumerate}
\section{Konto auflösen}
\label{sec:kontoAufloesen} 
\subsection{Konto löschen}\index{Konto!löschen} \label{subsec:kontoAufloesen}
\subsection{Ausscheiden aus der Uni}\index{Konto!auflösen} \label{subsec:kontoLoeschen}

\chapter{Erweiterte Funktionen}
\textit{Richtig verwalten ist das A\&O in PUMA. Es geht einfach und spart Zeit.}
\section{Richtig verwalten}

\subsection{Tags/ Schlagwortsystem}
\label{subsec:tags}
Tags\index{Tags} (dt. Schlagwörter) ermöglichen ein übersichtliches Organisieren und Strukturieren der Lesezeichen. Einem Literatureintrag können so viele Tags zu geordnet werden, wie Sie wollen. Durch den Gebrauch von Tags wird die Suche zu einem bestimmten Thema erleichtert, da Sie in die Such-Leiste nur den entsprechenden Tag eingeben müssen und Ihnen werden alle Einträge mit diesem Tag vorgelegt. Ein weiterer Vorteil des Tag-Systems ist, dass Sie bei der Literatursuche Tags kombinieren können und so spezifische Ergebnisse erhalten. So können Sie beispielsweise, wenn Sie Literatur zu dem Thema \enquote{Politik in Deutschland} suchen, die Tags \enquote{Politik} und \enquote{Deutschland} eingeben und erhalten die gesamte Literatur, die sich mit den Themen befasst. 
\begin{figure}[h!]
 \centering
 \fbox{\includegraphics[width=10cm]{Bilder/Kapitel6/Tags}}
 \caption{Tags}
 \label{figure025}
\end{figure}
\textbf{Tags zu Lesezeichen/~Publikationen hinzufügen}\newline
Tags ermöglichen ein übersichtliches Organisieren und Strukturieren der Lesezeichen. Sie können so viele Tags verwenden, wie Sie wollen. Die einzelnen Tags werden durch Leerzeichen voneinander getrennt.
\begin{figure}[h!]
 \centering
 \fbox{\includegraphics[width=10cm]{Bilder/Kapitel6/Tags_hinzufuegen}}
 \caption{Tags hinzufügen}
 \label{figure026}
\end{figure} 
%\begin{wrapfigure}{l}{7cm}
\begin{mdframed}
[style=mdfexample1,frametitle={\texttt{TIPP}},backgroundcolor=gray!40]
\texttt{Wenn Sie einen Tag verwenden möchten, der aus mehreren Worten besteht (z.~B. Fachbereich Architektur) dann verwenden Sie PascalCase\index{PascalCase} (z.~B. FachbereichArchitektur).} 
\end{mdframed}
%\end{wrapfigure}
\textbf{Tags von Lesezeichen/~Publikationen bearbeiten} \newline
PUMA bietet Ihnen die Möglichkeit bei Publikationen/~Lesezeichen, die schon Teil Ihrer Sammlung sind, die Tags zu bearbeiten. Es gibt drei Möglichkeiten die Tags\index{Tags!bearbeiten} zu bearbeiten:
\begin{enumerate}
    \item \underline{Tags bearbeiten über die \enquote{Schnellbearbeitung}}\newline
    Klicken Sie neben der Publikation/~Lesezeichen auf den blauen Stift (Tags bearbeiten). Es öffnet sich ein Pop-Up-Fenster. Sie können nun alte Tags entfernen, indem Sie auf das \enquote{X-Symbol} klicken. Um neue Tags hinzuzufügen, klicken Sie in das Textfeld und geben die Tags getrennt durch Leerzeichen ein. Um die Änderungen zu speichern klicken Sie auf \enquote{Speichern} und anschließend auf das \enquote{X} um das PopUp-Fenster zu schließen. Wenn Sie die Änderung verwerfen möchten, klicken Sie auf \enquote{Schließen}.
    \item \underline{Tags bearbeiten über \enquote{Eintrag bearbeiten}}\newline
    Klicken Sie auf den schwarzen Stift (Dieses Lesezeichen/Diese Publikation bearbeiten) rechts neben einem Eintrag. Sie können nun die Informationen, die Tags und die Sichtbarkeit des Eintrages bearbeiten. Klicken Sie anschließend auf \enquote{Speichern}.
    \item \underline{Tags bearbeiten über \enquote{Tags bearbeiten}}\newline
    PUMA bietet nicht nur die Möglichkeit die Tags eines einzelnen Eintrags zu bearbeiten, sondern auch alle Tags, die Sie verwenden. Klicken Sie auf das Personensymbol und wählen \enquote{Tags bearbeiten}. Sie können auf dieser Seite Tags und Konzepte\index{Konzepte} bearbeiten:
    \begin{enumerate}
        \item Umbenennen/~Ersetzten von Tags: \newline Hier können Sie alte Tags durch Neue ersetzen. Sie haben so die Möglichkeit, ähnliche Tags zu einem Tag zusammenzufügen.
        \item Subtags zu Konzepten hinzufügen: \newline
        Um ein Subtag zu einem Konzept hinzuzufügen, geben Sie den Namen des Konzepts in das Feld \enquote{Supertag} ein und das Tag, das Sie hinzufügen möchten in das Feld \enquote{Subtag}. Anschließend klicken Sie auf \enquote{Einfügen}.
        \item Subtags von Konzepten löschen:\newline Um ein Subtag von einem Konzept zu löschen, geben Sie den Namen des Konzepts in das Feld \enquote{Supertag} und das Tag, welches Sie löschen wollen, in das Feld \enquote{Subtag} ein. Anschließend klicken Sie auf \enquote{Löschen}.
    \end{enumerate}
\end{enumerate}
\textbf{Suchen\index{Suche} via Tags}\newline
PUMA ermöglicht, dass Sie mit Hilfe der Tags Lesezeichen und Publikationen finden können. \newline\newline
\underline{Möglichkeit 1:} Um einen Eintrag mit einem bestimmten Tag zu finden klicken Sie in der Suchleiste neben \enquote{Suche} auf den blauen Pfeil und wählen im Dropdown-Menü \enquote{Tags} aus. Geben Sie den Tag in das Suchfeld ein und drücken auf das Lupensymbol oder die Enter-Taste.\newline \newline
\underline{Möglichkeit 2:} Wenn Sie bei einem Eintrag auf einen Tag klicken, öffnet sich eine Seite mit allen Einträgen des Nutzer mit diesem bestimmten Tag. Auf der rechten Seite sehen Sie Informationen zu diesen Tag: Der Tag als Tag von allen Nutzern, verwandte Tags, die Konzepte des Nutzers und die verwendeten Tags des Nutzers. 
\newline
\newline
\subsubsection*{Systemtags\index{Systemtags}}
Systemtags\index{Tags!Systemtags} \label{systemtag} sind spezielle Tags, die eine feste Bedeutung haben. Derzeit bietet PUMA drei Typen von Systemtags an:
\begin{description}
\item[Ausführbare Systemtags:]
Ausführbare Systemtags werden zu einem Eintrag hinzugefügt, um eine spezielle Aktion mit diesem Eintrag auszuführen. Sie tragen ausführende Systemtags, wie die anderen Tags, in das Feld \enquote{Tags} ein. 
\begin{enumerate}
    \item \textit{for:<Gruppenname>} : Mit diesem Systemtag kopieren Sie den Eintrag in die Sammlung der Gruppe. In der Gruppe wird der Tag durch \textit{from:<IhrBenutzername>} ersetzt. Wenn Sie ihren Eintrag löschen oder bearbeiten, so bleibt der in die Gruppe kopierte Eintrag unverändert. Nur Mitglieder der Gruppe können Einträge für die Gruppe kopieren.
    \item \textit{send:<Benutzername>} : Damit senden Sie den Eintrag in den Eingang eines anderen Benutzers. Damit dies funktioniert, muss der Empfänger Sie als Freund eingetragen haben oder Sie müssen Mitglied in der gleichen Gruppe sein. Sobald der Eintrag bei dem Nutzer angekommen ist wird der Tag durch \textit{sent:<Benutzername>} ersetzt.
\end{enumerate}
\item[Meta-Systemtags:]  
Mit Meta-Systemtags markieren Sie Einträge. Derzeit werden folgende Meta-Systemtags unterstützt:
\begin{enumerate}
    \item \textit{myown:} Ein Eintrag, der mit dem Tag myown\index{myown} versehen wurde, erscheint auf Ihrer CV-Seite. Durch den Tag geben Sie an, dass Sie der Verfasser des Lesezeichen/der Publikation sind.
    \item \textit{sys:relevantFor:<Gruppenname>} : Einträge mit dem Tag sys:relevantFor:xy werden auf der \enquote{Interessant für\index{Interessant für}}-Seite der Gruppe xy angezeigt. Damit hat dieser Tag den gleichen Effekt, wie das  Auswählen der Gruppe xy in der \enquote{Interessant für}-Box beim Bearbeiten eines Eintrages. Der Tag wird durch eine blaue Blume am Anfang der Tag-Reihe dargestellt. 
    \item \textit{sys:hidden:<tag>} : Der Tag ist nur für Sie selbst sichtbar. Man findet diesen Tag bei einer Publikation, die im Inhaltsbereich abgebildet wird, nicht sichtbar in der Reihe der anderen Tags. Der Tag wird durch eine blaue Blume am Anfang der Tag-Reihe dargestellt. Wenn Sie auf die Detailansicht der Publikation klicken, taucht er sichtbar in der Tag-Reihe auf.
\end{enumerate}
\item[Such-Systemtags:] 
\label{itm:suchSystemtag}
Such-Systemtags sind nicht dazu da, um in einen Eintrag geschrieben zu werden, sondern um Einträge nach Suchanfragen zu filtern. Alle Such-Systemtags haben die gleiche Syntax: \textit{sys:<Feldname>:<Feldwert>}. Beispielsweise werden  bei der Suchanfrage \textit{sys:author:xyz} nur die Einträge angezeigt, welche von dem Autor \textit{xyz} stammen.\newline
Die Syntax kann entweder in die Suchleiste oder mit der URL eingeben werden. Folgende Filter unterstützt PUMA (Suche beschränkt sich auf die Publikationseinträge):\newline
\newline
Für die Suche nach einem bestimmten Autor oder Erscheinungsjahr müssen Sie vorher festlegen, in welchen Einträgen eines Nutzers Sie nach dem Autor oder dem Erscheinungsjahr suchen möchten. Zum Beispiel suchen Sie, wenn Sie diese Daten eingeben: \newline https://puma.ub.uni-stuttgart.de/user/\newline droessler/sys:year:2013 \newline Publikationen aus dem Jahr 2013 in den Einträgen des Nutzers Droessler. 
\begin{enumerate}
    \item \textit{sys:author:<Autorenname>} filtert die Suche nach dem Autor.
    \item \textit{sys:year:<Jahr>} filtert die Suche nach dem Erscheinungsjahr. Dabei sind mehrere Schreibweisen für das Jahr möglich:
    \begin{enumerate}
        \item 2000: Alle Einträge aus dem Jahr 2000
        \item 2000-: Alle Einträge aus dem Jahr 2000 oder einem Jahr danach
        \item -2000: Alle Einträge aus dem Jahr 2000 oder einem Jahr davor
        \item 1990-2000: Alle Einträge aus den Jahren 1990 bis 2000
    \end{enumerate}
%muss noch raus rutschen
Bei der Suche nach Titel, Gruppe, Nutzer, usw. spielt der Nutzer, bei dem Sie suchen keine Rolle. Sie müssen dementsprechend nur den Zusatz tag/ vor die Suchsyntax setzten, zum Beispiel  https://puma.ub.uni-stuttgart.de/tag/sys:entrytype:article. Hier finden Sie nun alle Artikel, die auf PUMA eingetragen wurden.
    \item \textit{sys:title:<title>} sucht nach Einträgen mit diesem Titel.
    \item \textit{sys:user:<user>} sucht nach Einträgen eines Nutzers.
    \item \textit{sys:group:<group>} filtert die Suche nach einer bestimmten Gruppe.
    \item \textit{sys:entrytype:<Eintragstyp>} filtert die Suche nach dem Eintragstypen. Eintragstypen\footnote{\url{https://www.ctan.org/pkg/biblatex?lang=de}} werden verwendet, um BibTex-Einträge nach ihren Typen zu klassifizieren. Derzeit unterstützt Puma folgende Eintragstypen\index{Eintragstypen}: 
    \begin{enumerate}
        \item \textbf{article\index{Article}:} Zeitungs- oder Zeitschriftenartikel\newline
        Erforderliche Felder: Autor, Titel, Zeitschriftentitel, Ausgabennummer, Jahr/Datum 
        \item \textbf{book\index{Book}:} Buch, Monografie mit angegebenem Verlag\newline
        Erforderliche Felder: Autor, Titel, Jahr
        \item \textbf{booklet\index{Booklet}:} Gebundenes Druckwerk, aber ohne Verlag oder Sponsororganisation\newline
        Erforderliche Felder: Autor/Lektor, Titel, Jahr/Datum
        \item \textbf{conference\index{Conference}:} Ein Beitrag zu einer Konferenz, der nicht in einem Konferenzband erschienen ist\newline
        Erforderliche Felder: Autor, Titel, Buchtitel, Jahr/Datum
        \item \textbf{electronic\index{Electronic}:} Elektronische Veröffentlichungen, z. B. eBooks oder Blogeinträge\newline 
        Erforderliche Felder: Autor, Buchtitel, Verlag, Adresse, Jahr
        \item \textbf{inbook\index{Inbook}:} Teil eines Buches, z. B. ein Kapitel oder ein Seitenbereich\newline
        Erforderliche Felder: Autor, Titel, Buchtitel, Jahr/Datum 
        \item \textbf{incollection\index{Incollection}:} Teil eines Buches mit einem eigenem Titel, z. B. Beitrag in einem Sammelband\newline
        Erforderliche Felder: Autor, Titel, Buchtitel, Jahr/Datum
        \item \textbf{inproceedings\index{Inproceedings}:} Artikel in einem Tagungsband bzw. Konferenzband\newline
        Erforderliche Felder: Autor, Titel, Buchtitel, Jahr/Datum
        \item \textbf{manual\index{Manual}:} Technische Dokumentation, Handbuch\newline
        Erforderliche Felder: Autor/Lektor, Titel, Jahr/Datum
        \item \textbf{mastersthesis\index{Mastersthesis}:} Master-, Magister- oder Diplomarbeit\newline
        Erforderliche Felder: Autor, Titel, Art der Arbeit, Institut, Jahr/Datum
        \item \textbf{misc\index{Misc}:} Diesen Eintragstyp können Sie wählen, wenn nichts anderes zu passen scheint. \newline
        Erforderliche Felder: Autor/Lektor, Titel, Jahr/Datum
        \item \textbf{patent\index{Patent}:} Patent\newline 
        Erforderliche Felder: Autor, Titel, Nummer, Jahr/Datum
        \item \textbf{periodical\index{Periodical}:} Ein regelmäßig erscheinendes Werk, z.B. Zeitschrift\newline
        Erforderliche Felder: Lektor, Titel, Jahr/Datum
        \item \textbf{phdthesis\index{Phdthesis}:} Doktor- oder andere Promotionsarbeit\newline 
        Erforderliche Felder: Autor, Titel, Hochschule/Universität, Jahr 
        \item \textbf{presentation\index{Presentation}:} Präsentation, Vortrag auf einer Veranstaltung\newline 
        Erforderliche Felder: Verfasser, Titel, Anlass der Präsentation, Jahr
        \item \textbf{proceedings\index{Proceedings}:} Tagungsband einer Konferenz\newline
        Erforderliche Felder: Titel, Jahr/Datum
        \item \textbf{standard\index{Standard}:} Standard\newline 
        Erforderliche Felder: Autor, Buchtitel, Verlag, Adresse, Jahr 
        \item \textbf{techreport\index{Techreport}:} Bericht einer Hochschule oder einer anderen Institution\newline
        Erforderliche Felder: Autor, Titel, Jahr/Datum
        \item \textbf{unpublished\index{Unpublished}:} Nicht formell veröffentlichtes Dokument\newline 
        Erforderliche Felder: Autor, Titel, Jahr/Datum
    \end{enumerate}
    \item \textit{sys:not:<tag>} filtert die Suche, indem alle Ergebnisse ignoriert werden, die diesen Tag enthalten. An dieser Stelle können Sie auch Platzhalter verwenden, z.B. werden bei sys:not:news\_ 
    alle Ergebnisse ignoriert, die Tags enthalten, die mit news\_
    beginnen.
    \item \textit{sys:bibtexkey:<bibtexkey>} filtert die Suche nach einem bestimmten BibTeX-Schlüssel.
\end{enumerate}
\end{description}
\subsection{Konzepte}
\underline{Was sind Konzepte\index{Konzepte}?}
\newline
Durch Konzepte können Sie Ihre Tags nach Gruppen ordnen und sich so die Suche erleichtern. Sie haben beispielsweise das Konzept mit dem Supertag\index{Supertag} (Namen) Obst, diesem sind die Subtags\index{Subtag} Banane, Apfel und Kiwi zugeordnet. Wenn Sie nun mit dem Konzept Obst suchen, werden Ihnen automatisch alle Publikationen und Lesezeichen angezeigt, die mit mindestens einem der Subtags getagged wurden. Dies erleichtert Ihre Suche, da oft nach Publikationen/Lesezeichen zu einem bestimmten Thema gesucht wird. 
\newline Ihre angelegten Konzepte finden Sie über das Untermenü von \enquote{mein PUMA}. Um zu den beliebten Konzepten von PUMA zu gelangen klicken Sie im Hauptmenü auf \enquote{Beliebte} und anschließend im Untermenü auf \enquote{Konzepte}. 
\newline
\newline
\underline{Konzepte erstellen}
\newline
Um Konzepte zu erstellen oder zu überarbeiten, klicken Sie auf das Personensymbol auf der rechten Seite. Ein Untermenü öffnet sich und Sie klicken auf Tags bearbeiten. 
\begin{figure}[h!]
 \centering
 \fbox{\includegraphics[width=10cm]{Bilder/Kapitel6/Konzepte_erstellen}}
 \caption{Ein Konzept erstellen}
 \label{figure027}
\end{figure}
\textbf{Subtags zu Konzepten hinzufügen:} PUMA ermöglicht Ihnen neue Konzepte zu erstellen oder zu einem bereits existierenden Konzept neue Tags hinzufügen. Um ein neues Konzept hinzuzufügen, wählen Sie einen Tag, der als Name für das Konzept stehen soll, aus. Diesen Tag geben Sie in das Feld \enquote{Supertag} ein. Den Tag, der dem Konzept hinzugefügt werden soll, geben Sie in das Feld \enquote{Subtag} ein.
\begin{mdframed}[style=mdfexample1,frametitle={\texttt{ACHTUNG}},backgroundcolor=gray!40]\texttt{Es kann immer nur ein Subtag eingegeben werden, wenn Sie zwei Subtags gleichzeitig eingeben wird das Konzept nicht erstellt. Um mehrere Subtags in einem Konzept zu vereinen, müssen sie den oben genannten Ablauf zur Erstellung eines Konzeptes mit jedem neuen Subtag wiederholen und dabei das Supertag unverändert lassen.} 
\end{mdframed}
\textbf{Subtags von Konzept löschen:} Sie können auch Tags aus einem Konzept entfernen. Dafür geben Sie in das Feld \enquote{Supertag} den Namen des Konzepts ein und in das Feld \enquote{Subtag} den Tag, der gelöscht werden soll. 
\begin{mdframed}[style=mdfexample1,frametitle={\texttt{ACHTUNG}},backgroundcolor=gray!40]\texttt{Hier kann ebenfalls immer nur ein Tag in das Feld Subtag eingegeben werden, da sonst die Aktion nicht durchgeführt wird.}
\end{mdframed}
\underline{Navigation mit Konzepten}
\newline
Um mit Konzepten zu suchen, benutzen Sie einfach die Suchleiste rechts oben. Klicken Sie auf den blauen Pfeil neben \enquote{Suche} und wählen Sie im Dropdown-Menü Konzepte aus. Geben Sie den Namen des Konzepts, mit dem Sie suchen möchten, in das Suchfeld ein und klicken Sie auf das Lupensymbol oder drücken Sie die Enter-Taste. Die Lesezeichen/Publikationen, die mit einem der Subtags des Konzepts getagged worden sind, werden Ihnen angezeigt. 
\subsection{Duplikate}
Beim Sammeln von Publikationen und Lesezeichen kann es vorkommen, dass Publikationen zweimal in eine PUMA-Sammlung eintragen werden. Hier bietet PUMA die Möglichkeit Duplikate\index{Duplikate} sofort zu erkennen und seine Sammlung aufzuräumen. Um einen Überblick über alle Duplikate in seiner Sammlung zu erhalten, klicken Sie im Hauptmenü auf \enquote{meinPuma}. Im Dropdown-Menü können Sie nun \enquote{Duplikate} auswählen und gelangen so auf die Übersichtsseite. Ein anderer Weg, um sich einen Überblick zu verschaffen, bieten die Zahlen oben rechts bei jedem Eintrag. Sie geben an, wie viele Einträge mit dem gleichen Titel es in der Sammlung gibt. Ist diese Zahl größer als 1 handelt es sich um Duplikate. Wenn Sie auf die Zahl klicken werden Ihnen die Duplikate angezeigt.
\begin{figure}[h!]
 \centering
 \fbox{\includegraphics[width=10cm]{Bilder/Kapitel6/Duplikate}}
 \caption{Duplikate}
 \label{figure027}
\end{figure}
\subsection{Private Dateien anhängen}
Sie können an jede Ihrer Publikationen ein Dokument\index{Dokumente! anhängen} anhängen (max. 50 MB pro Datei - erlaubte Dateiendungen: pdf, ps, djv, djvu, txt, tex, doc, docx, ppt, pptx, xls, xlsx, ods, odt, odp, jpg, jpeg, svg, tif, tiff, png, htm, html, epub). Der Anhang ist aus urheberrechtlichen Gründen nur für Sie selbst sichtbar.

\begin{mdframed}[style=mdfexample1,frametitle={\texttt{BEDINGUNG}},backgroundcolor=gray!40]\texttt{Um an eine Publikation eine Datei anzuhängen, muss die Publikation in Ihrer Sammlung eingetragen sein.}\end{mdframed}
\begin{enumerate}
    \item Klicken Sie auf den Titel der Publikation. Es öffnet sich die Detailansicht der Publikation.
    \item Klicken Sie nun entweder auf den schwarzen Stift oben rechts auf der Seite. Es öffnet sich eine neue Seite, auf der Sie die Publikation bearbeiten können. Scrollen Sie runter bis zu \enquote{private Dokumente} und klicken auf \enquote{Durchsuchen}. \newline \textbf{ODER:} Sie klicken in der Detailansicht auf das Bild der Publikation. Unterhalb des Bildes erscheint der Durchsuchen-Button. 
    \item Es öffnet sich ein Pop-Up Fenster, indem Sie das Dokument auswählen können, welches Sie anhängen wollen. Klicken Sie anschließend auf \enquote{Öffnen}.
    \item Der Upload startet automatisch. Sobald er abgeschlossen ist, wird der Dateienname der hochgeladenen Datei und ein schwarzes \enquote{X} unter dem Abschnitt \enquote{private Dokumente} angezeigt. Über das schwarze \enquote{X} kann das Dokument wieder entfernt werden.
    \item Klicken Sie anschließend ganz unten auf der Seite auf \enquote{Speichern}, da ansonsten Ihre Änderung nicht gespeichert wird.
\end{enumerate}
\begin{figure}[h!]
 \centering
 \fbox{\includegraphics[width=10cm]{Bilder/Kapitel6/Private_Dokumente}}
 \caption{Private Dokumente anhängen}
 \label{figure028}
\end{figure}
Wenn die angehängte Datei auch für Gruppenmitglieder sichtbar seien soll, muss der  Gruppenadministrator das Teilen von Dokumenten erlauben
und die einzelnen Mitglieder dies ebenfalls in ihrer Gruppeneinstellung freischalten. In den Einstellungen kann der Nutzer unter dem Reiter \enquote{Gruppen} für jede einzelne Gruppe festlegen, ob Dateien geteilt werden oder nicht. Diese Funktion kann jederzeit wieder deaktiviert werden.
\subsection{Publikationen durchstöbern}
Oftmals verliert man schnell den Überblick über seine Einträge. Um sich schnell einen Überblick über seinen Literaturbestand machen zu können bietet PUMA die Funktion \enquote{Publikation\index{Publikationen!durchstöbern} durchstöbern} an. 
\begin{enumerate}
    \item Klicken Sie im Hauptmenü auf \enquote{meinPUMA}. Ein Dropdown-Menü öffnet sich.
    \item Klicken Sie auf \enquote{Publikationen durchstöbern}.
    \item Unter \enquote{Suchoptionen} können Sie verschiedene Tags und Autoren auswählen, zu denen Sie die Einträge sehen möchten. Um mehrere Begriffe aus der Liste auszuwählen, halten Sie die STRG- bzw. CTRL-Taste während des Mausklicks gedrückt.
\begin{figure}[h!]
 \centering
 \fbox{\includegraphics[width=10cm]{Bilder/Kapitel6/Publikationen_durchstoebern}}
 \caption{Publikationen durchstöbern}
 \label{figure029}
\end{figure}
    \item Die Buttons \enquote{und/~oder} können Sie dazu nutzen, um die Listenauswahl unterschiedlich zu verknüpfen. 
    \item Unter \enquote{Suchergebnisse} sehen Sie alle Ergebnisse, die zu ihren Vorgaben aus 3. und 4. passen.
    \item Das Textfeld \enquote{Filter} ermöglicht es die Ergebnisse aus Schritt 5 noch weiter zu filtern.
\end{enumerate}
\subsection{Eigene Einträge bearbeiten}
Ist ein Eintrag einmal hinzugefügt, heißt dies nicht, dass man ihn nie mehr bearbeiten kann. Die erste Möglichkeit wäre, dass der Nutzer jeden Eintrag selber von Hand nachträglich bearbeitet. Die zweite Lösung ist zeitsparender und schneller.\newline
In der Sammlung befindet sich oben rechts oberhalb der Publikations- und Lesezeichenspalte ein Stift, mit dem Sie Ihre eigenen Eintrage bearbeiten können. Durch das Klicken auf den Stift gelangen Sie auf die Seite \enquote{Eigene Einträge bearbeiten}. Hier können Sie auswählen, was Sie an der/~den Publikation/~en ändern möchten:
\begin{itemize}
\item Tags zu allen ausgewählten Posts hinzufügen
\item Die Tags aller ausgewählten Einträge separat bearbeiten
\item BibTex-Schlüssel normalisieren
\item Einträge löschen
\item Sichtbarkeit einstellen
\end{itemize}
Die Änderungen können für eine Publikation/Lesezeichen oder mehrere sein. Dies können Sie unten auf der Seite festlegen. Durch klicken auf die entsprechenden Einträge wählen Sie diese aus.
\begin{figure}[h!]
 \centering
 \fbox{\includegraphics[width=10cm]{Bilder/Kapitel6/Eigene_Eintraege_bearbeiten}}
 \caption{Eigene Einträge bearbeiten}
 \label{figure030}
\end{figure}
\subsection{OpenURL-Resolver/Bestandsanfrage}
\label{subsec:OpenURL}
Mit Hilfe von OpenURL\index{OpenURL} kann man bei Publikationen aus seiner eigenen Sammlung überprüfen, ob sich diese im Katalog der jeweiligen Büchereien befindet. Dafür müssen Sie  die folgende URL:  
\url{http://www.redi-bw.de/links/unist} in Ihre Einstellungen kopieren, dabei gehen Sie wie folgt vor:
\begin{enumerate}
    \item Klicken Sie auf das Personensymbol, ein Untermenü öffnet sich.
    \item Klicken Sie auf \enquote{Einstellungen}.
    \item Geben Sie in der Rubrik \enquote{Kontakt} in das Feld \enquote{OpenURL} die URL \url{http://www.redi-bw.de/links/unist} ein. 
\begin{figure}[h!]
 \centering
 \fbox{\includegraphics[width=10cm]{Bilder/Kapitel6/Open_URL}}
 \caption{Open-URL der Universität Stuttgart}
 \label{figure031}
\end{figure}
    \item Speichern Sie die Änderung auf dem Ende der Seite.
\end{enumerate}
Ab sofort befindet sich bei jeder Ihrer Publikationen unter dem Bereich \enquote{Links und Ressourcen} die entsprechende Open-URL, über die Sie nun eine Bestandsabfrage durchführen können.
\begin{figure}[h!]
 \centering
 \fbox{\includegraphics[width=10cm]{Bilder/Kapitel6/OpenURL-Resolver}}
 \caption{OpenURL-Resolver}
 \label{figure032}
\end{figure}
\subsection{Open Access-Zugriff auf Publikationsdienste}%Screenshot
Der Zugriff auf Open Access\index{Open Access} Publikationsdienste ermöglicht Ihnen über die Detailansicht einer Publikation nach der digitalen Ausgabe in einer Open Access-Datenbank zu suchen. Voraussetzung hierfür ist, dass die Detailansicht der Publikation aufgerufen ist. Zur Detailansicht gelangen Sie, indem Sie im Inhaltsbereich oder Ihrer persönlichen Sammlung (unter \enquote{Mein PUMA}) auf den Titel einer Publikation klicken. 
\begin{enumerate}
    \item Klicken Sie auf das Auswahlmenü \enquote{Suchen auf}. Ein Untermenü erscheint.
\begin{figure}[h!]
 \centering
 \fbox{\includegraphics[width=10cm]{Bilder/Kapitel6/Open-Access}}
 \caption{Open Access}
 \label{figure033}
\end{figure}
    \item Wählen Sie aus der angezeigten Liste die Open Access-Datenbank, die Sie nach diesem Artikel durchsuchen möchten. 
\end{enumerate}
 So gelangen Sie schnell und einfach zu der digitalen Ausgabe einer Publikation. 
\subsection{Eingang}
In Ihrem Eingang\index{Eingang} finden Sie alle Beträge, die Ihnen von Freunden geschickt wurden.
\begin{figure}[h!]
 \centering
 \fbox{\includegraphics[width=10cm]{Bilder/Kapitel6/Eingang}}
 \caption{Der Eingang}
 \label{figure034}
\end{figure}
\underline{Einträge verschicken\index{Einträge!verschicken}}
\newline
Um einem anderen Nutzer ein Lesezeichen oder eine Publikation zu schicken, verwenden Sie das Systemtag \textit{send:xyz}. Dieses Tag geben Sie mit weiteren Tags beim Eintragen einer Publikation/Lesezeichen mit ein. Der Eintrag wird dann getaggt mit from:<YourUserName> und in den Eingang von dem Nutzer xyz kopiert. Um den Missbrauch des Eingangs zu verhindern, muss der Empfänger des Eintrags
\begin{enumerate}
    \item entweder mit Ihnen befreundet sein
    \item oder Mitglied einer gemeinsamen Gruppe sein.
\end{enumerate}
Nachdem der Eintrag gesendet wurde wird der Tag von \textit{send:xyz} in \textit{sent:xyz} automatisch umgewandelt.
\newline
\newline
\underline{Einträge erhalten\index{Einträge!erhalten}}
\newline
In Ihrem Eingang liegen alle Einträge, die Ihnen geschickt wurden. Sie können diese Einträge über den Button \enquote{Diese Publikation in die eigene Sammlung einfügen}, rechts neben dem Eintrag (zwei Blätter) übernehmen. Mit \enquote{Diese Publikation aus Ihrem Eingang entfernen} können Sie den Eintrag aus dem Eingang löschen und über das schwarze Zahnrad den ganzen Eingang leeren.
\section{Literaturlisten erstellen}
PUMA bietet Ihnen die Möglichkeit, aus Ihren gesammelten Publikationen Literaturlisten\index{Literaturlisten} zu erstellen, die Sie später beispielsweise auf externen Webseiten verwenden können. \newline
Hierfür fügen Sie die Publikationen, die in das Literaturverzeichnis sollen, zu Ihrer Ablage\index{Ablage} hinzu. Wenn Sie alle Publikationen hinzugefügt haben, klicken Sie in der Ablage, oberhalb von den Publikationen, auf Exportzeichen und wählen Sie \enquote{mehr...} aus. Sie gelangen nun zu einer Übersichtsseite, auf der Ihnen alle verfügbaren Zitationsstile angezeigt werden, und Sie nur noch den passenden aussuchen müssen. 
\subsection{Eigene Literaturlisten erstellen} 
Neben den Layouts für das Erstellen einer Literaturliste können Sie auch folgenden URLs verwenden, die Sie in Ihren Browser eingeben:
\begin{enumerate}%Beispielscreenshots ?
    \item \textbf{Allgemeine Liste:}\newline
    \textit{https://puma.ub.uni-stuttgart.de/publ/user/<username>} \newline
    Ersetzen Sie \textit{<username>} durch Ihren Benutzernamen und Ihnen werden alle Publikationen aus Ihrer Sammlung in einer Literaturliste angezeigt.\newline
    \textbf{Beispiel:} https://puma.ub.uni-stuttgart.de/publ/user/eckert 
\begin{figure}[h!]
 \centering
 \fbox{\includegraphics[width=11cm]{Bilder/Kapitel6/Allgemeine_Liste}}
 \caption{Allgemeine Liste}
 \label{figure035}
\end{figure}

    \item \textbf{Allgemeine Liste ohne Tags:}\newline
    \textit{https://puma.ub.uni-stuttgart.de/publ/user/<username>?notags=1}\newline
    Ersetzen Sie \textit{<username>} durch Ihren Benutzernamen und Ihnen werden alle Publikationen aus Ihrer Sammlung, ohne Tags, in einer Literaturliste angezeigt.\newline
    \textbf{Beispiel:} https://puma.ub.uni-stuttgart.de/publ/user/eckert?notags=1 
    
\begin{figure}[h!]
 \centering
 \fbox{\includegraphics[width=11cm]{Bilder/Kapitel6/Allgemeine_Liste_ohne_Tags}}
 \caption{Allgemeine Liste ohne Tags}
 \label{figure036}
\end{figure}

    \item \textbf{Allgemeine Liste mit Tag-Einschränkung:}\newline
    \textit{https://puma.ub.uni-stuttgart.de/publ/user/<username>/<tagname>}\newline
    Ersetzen Sie \textit{<username>} durch Ihren Benutzernamen und \textit{<tagname>} durch den Tag, der in den Publikationen enthalten sein soll. Ihnen wird eine Literaturliste angezeigt, die jene Publikationen aus Ihrer Sammlung enthält, die den speziellen Tag enthalten. Ein besonderes Beispiel hierfür ist der Tag \textit{myown\index{myown}}. Durch diesen Tag geben Sie an, dass Sie der/~die Verfasser/~in der Publikation sind. \newline
    \textbf{Beispiel:} https://puma.ub.uni-stuttgart.de/publ/user/eckert/puma
   
\begin{figure}[h!]
 \centering
 \fbox{\includegraphics[width=11cm]{Bilder/Kapitel6/Allgemeine_Liste_Tag_Einschraenkung}}
 \caption{Liste mit Tagauswahl}
 \label{figure037}
\end{figure}

    \item \textbf{BibTeX-Liste:}\newline
    \textit{https://puma.ub.uni-stuttgart.de/bib/user/<username>} \newline
    Ersetzen Sie \textit{<username>} durch Ihren Benutzernamen. Ihnen wird eine Literaturliste mit all Ihren Publikationen im BibTex-Format\index{BibTex} angezeigt.\newline
    \textbf{Beispiel:} https://puma.ub.uni-stuttgart.de/bib/user/eckert 

\begin{figure}[h!]
 \centering
 \fbox{\includegraphics[width=11cm]{Bilder/Kapitel6/Bibtex_Liste}}
 \caption{BibTex-Liste}
 \label{figure038}
\end{figure}

\end{enumerate}
\subsection{JabRef-Layouts}
Einen kompletten Überblick zu allen verfügbaren Jabref-Layouts\index{JabRef!Layouts} erhalten Sie auf der Export-Seite von PUMA.
\begin{enumerate}
	\item  \textbf{/layout/simplehtml/}\newline
	Sie erhalten eine HTML-Übersicht - über alle Publikationen im 		Inhaltsbereich - ohne Kopf- oder Fußzeile nützlich für die 			Einbindung von Publikationslisten in andere HTML-Seiten.
	\item \textbf{/layout/html/}\newline
    Eine einfache Übersicht aller Publikationen aus dem Inhaltsbereich, in der jeder Eintrag als Zeile in einer Tabelle dargestellt ist.
	\item \textbf{/layout/tablerefs/} \newline
    HTML-Ausgabe mit jedem Eintrag als Zeile in einer Tabelle und einer zusätzlichen JavaScript-Suchfunktion.
\item \textbf{/layout/tablerefsabsbib/} \newline
    Ähnelt \textit{/layout/tablerefs/}. Enthält auch die BibTeX-Quelle und die Kurzbeschreibung der Publikation.
\item \textbf{/layout/docbook/} \newline
    Dies ist eine XML-Ausgabe gemäß dem DocBook-Schema.
\item \textbf{/layout/endnote/} \newline
    Sie erhalten eine Ausgabe in RIS, welche von dem Literaturverwaltungsprogramm EndNote verwendet wird.
\item \textbf{/layout/dblp/} \newline
    DBLP exportiert alle Publikationen aus dem Inhaltsbereich in eine DBLP-konforme XML-Struktur. 
\item \textbf{/layout/text/}\newline
    Alle Publikationen aus dem Inhaltsbereich werden in einer BibTeX-Ausgabe dargestellt.
\end{enumerate}

\todo[inline]{Sortieren: Datum! Bibtexfelder auswählen Persönlicher Bereich vs. Lucine Index}
\todo[inline]{Dubletten erklären}

\chapter{Zusammenspiel mit anderen Anwendungen und Programmen}
\label{ch:exportImport}
\label{sec:EExpo}

\section{Literaturlisten exportieren}
\label{sec:llExportieren}
PUMA ermöglicht einen Export\index{Export} von Publikationslisten aus PUMA in andere Programme.
Der Export erfolgt in zwei Schritten. Es wird zuerst ein Literaturverzeichnis in Puma zusammengestellt und exportiert, bevor es dann in ein anderes Programm importiert wird.
Um ein Literaturverzeichnis zusammenzustellen, müssen im ersten Schritt die Publikationen, die exportiert werden sollen, in die Ablage kopiert werden (vgl. \autoref{sec:ablage}).
		
\begin{figure}[h!]
 \centering
 \fbox{\includegraphics[width=10cm]{Bilder/Kapitel7/Zur_Ablage_hinzufuegen}}
 \caption{Zur Ablage hinzufügen}
 \label{fig:zurAblageHinzu}
\end{figure}

In der Ablage über das Exportzeichen oben rechts das Format, in dem das Literaturverzeichnis exportiert werden soll auswählen. Über \enquote{mehr...} stehen weitere Exportformaten zur Verfügung. Sollte das gewünschte Format nicht dabei sein, kann auch ein eigenes generiert werden (vgl. \todo{Verweis}).
\begin{figure}[h!]
 \centering
 \fbox{\includegraphics[width=11cm]{Bilder/Kapitel7/Exportformat_auswaehlen}}
 \caption{Das Exportformat auswählen}
 \label{fig:exportformatAuswaehlen}
\end{figure}


		
\begin{figure}[h!]
 \centering
 \fbox{\includegraphics[width=11cm]{Bilder/Kapitel7/Das_Literaturverzeichnis}}
 \caption{Das Literaturverzeichnis}
 \label{fig:literaturverzeichnis}
\end{figure}


\subsection{Literaturverzeichnis exportieren - Programmspezifisch}
\label{subsec:lvExportProgramme}
\subsubsection*{Export nach Word\index{Export!Word}} \index{Word} \label{sss:exportWord}
In Microsoft Word kann die, im Format \enquote{MSOffice XML} \index{MSOffice XML} gespeicherte Datei verwendet werden, indem im Quellen-Manager (\enquote{Verweise}\enquote{Quellen verwalten}) die gespeicherte Datei ausgewählt wird \footcite{Genaue Anleitung im Blogbeitrag:https://blog.ub.uni-stuttgart.de/category/puma/}{}.\todo[inline]{mir geht die Beschreibung zu weit. Ich würde bei PUMA aufhören.}
%\begin{enumerate}
    %\item Klicken Sie in der Ablage auf das Exportzeichen.
    %\item Wählen Sie im Dropdown-Menü \enquote{mehr...} aus.
    %\item Es öffnet sich die Übersichtsseite der Exportformate. . Speichern Sie anschließend die Datei.
\begin{figure}[h!]
 \centering
 \fbox{\includegraphics[width=11cm]{Bilder/Kapitel7/MSOffice_XML}}
 \caption{Das Exportformat MSOffice XML}
 \label{fig:exportformatMSOfficeXml}
\end{figure}

\begin{figure}[h!]
 \centering
 \fbox{\includegraphics[width=11cm]{Bilder/Kapitel7/Word}}
 \caption{Reiter Verweise}
 \label{fig:reiterVerweise}
\end{figure}
\begin{figure}[h!]
 \centering
 \fbox{\includegraphics[width=11cm]{Bilder/Kapitel7/Quellen-Manager}}
 \caption{Quellen-Manager}
 \label{fig:quellenManager}
\end{figure} 

\subsubsection*{Export nach Citavi\index{Export!Citavi}}\label{sss:exportCitavi}
Für Citavi die Literaturliste in BibTeX exportieren und in die Zwischenablage kopieren. 
In Citavi in der Menüleiste oben links \enquote{Datei} auswählen und im Dropdown-Menü \enquote{Importieren} auswählen. 
Über \enquote{Aus einer Textdatei (Ris-, BibTex-formatiert o.ä.)} BibTeX als Format auswählen und anschließend über \enquote{Textdaten in der Zwischenablage verwenden} Datei hochladen. Die BibTeX-Keys über \enquote{Importierte BibTex Keys ersetzen} normalisieren und die \enquote{Titel übernehmen}.

\subsubsection*{Export nach Zotero\index{Export!Zotero}}\label{sss:exportZotero}
Auf der gewünschten PUMA-Seite, von der eine Publikation in die Zotero-Bibliothek übernommen werden soll, auf den schwarzen Pfeil neben dem Zotero\index{Zotero}-Symbol oben rechts bei Firefox.
Ein Dropdown-Menü öffnet sich. \enquote{In Zotero mit \enquote{unAPI} speichern} auswählen.
    
\begin{figure}[h!]
 \centering
 \fbox{\includegraphics[width=10cm]{Bilder/Kapitel7/Zotero_Dropdown_Menue}}
 \caption{Dropdown-Menü}
 \label{fig:dropdownMenue}
\end{figure}

Es öffnet sich ein Popup-Fenster, in dem alle Publikationen der entsprechenden PUMA-Seite aufgelistet sind. Die Publikationen auswählen, die in die Zotero-Bibliothek übernommen werden sollen.
\begin{figure}[h!]
 \centering
 \fbox{\includegraphics[width=10cm]{Bilder/Kapitel7/Zotero_Eintraege_auswaehlen}}
 \caption{Einträge auswählen}
 \label{fig:eintraegeAuswaehlen}
\end{figure}


\subsubsection*{Export nach JabRef\index{Export!JabRef}}\label{sss:exportJabref}
Für JabRef die Literaturliste in BibTeX exportieren und speichern. In JabRef dann die Datei über \enquote{Importieren in neue Datenbank} oder \enquote{Importieren in aktuelle Datenbank} importieren.

\section{Literaturlisten importieren}
\label{sec:llImportieren}
Das Importieren\index{Import} von Literaturlisten aus anderen Programmen in PUMA ist jederzeit möglich. Der Import erfolgt in zwei Schritten. Zuerst werden die gewünschten Publikationen aus dem Literaturverwaltungsprogramm exportiert, um dann anschließend in PUMA importiert zu werden. 
\subsection{BibTex-Export aus verwendeten Literaturverwaltungsprogrammen}
\label{subsec:bibtexExport}
\subsubsection*{Import aus Zotero\index{Import!Zotero}} \label{sss:importZotero}

Um den Import von Zotero\index{Zotero} zu PUMA zu ermöglichen, muss Zotero zunächst für PUMA konfiguriert werden:

Zotero öffnen und in den Einstellungen zu den Website-spezifischen Einstellungen einen neuen Eintrag hinzufügen. Dazu auf das \enquote{ '+'-Symbol}klicken und bei Domain/Pfad \textit{puma.ub.uni-stuttgart.de} eingeben und \textit{BibTeX\index{BibTex}} als Ausgabeformat auswählen.
\begin{figure}[h!]
 \centering
 \fbox{\includegraphics[width=9cm]{Bilder/Kapitel7/Zotero_Import}}
 \caption{Import aus Zotero}
 \label{fig:importZotero}
\end{figure}
Im Dropdown-Menü \enquote{Einstellungen} $/to$ \enquote{Export}auswählen.

\begin{figure}[h!]
 \centering
 \fbox{\includegraphics[width=9cm]{Bilder/Kapitel7/Zotero_Menuepunkt_Export}}
 \caption{Menüpunkt Export}
 \label{fig:menueExport}
\end{figure} 
    
Nachdem die Konfiguration vorgenommen wurde, können die Publikationen nach PUMA importiert werden. 
Dazu im Reiter \enquote{BibTex\index{BibTex}/EndNote\index{EndNote}-Schnipsel}. In das Feld \enquote{Auswahl} den entsprechenden Eintrag aus der Zotero-Bibliothek durch Drag und Drop hineinziehen.
\begin{figure}[h!]
 \centering
 \fbox{\includegraphics[width=10cm]{Bilder/Kapitel7/Zu_PUMA_importieren}}
 \caption{Zu PUMA importieren}
 \label{fig:zuPumaImportieren}
\end{figure}

\subsubsection*{Bib-TeX-Export aus Citavi\index{Import!Citavi}}\label{sss:importCitavi} 
Bei Citavi\index{Citavi} oben rechts über \enquote{Datei} im Dropdown-Menü \enquote{Exportieren} auswählen. Die gewünschten Einträge markieren und als Export-Format \enquote{BibTex} angeben.
    
\begin{figure}[h!]
 \centering
 \fbox{\includegraphics[width=9cm]{Bilder/Kapitel7/Citavi_Schritt2}}
 \caption{Auswählen der zu exportierenden Artikel}
 \label{fig:exportierendenArtikelAuswaehlen}
\end{figure}

  
\begin{figure}[h!]
 \centering
 \fbox{\includegraphics[width=9cm]{Bilder/Kapitel7/Citavi_Schritt3}}
 \caption{Export-Format festlegen}
 \label{fig:exportFormatFestlegen}
\end{figure}

 Als Speicherort \enquote{Textdaten in der Zwischenablage speichern} wählen und die Export-Vorlage unter dem Namen \textit{BibTex} speichern.
   
\begin{figure}[h!]
 \centering
 \fbox{\includegraphics[width=9cm]{Bilder/Kapitel7/Citavi_Schritt4}}
 \caption{Speicherort}
 \label{fig:speicherort}
\end{figure}

\begin{figure}[h!]
 \centering
 \fbox{\includegraphics[width=9cm]{Bilder/Kapitel7/Citavi_Schritt5}}
 \caption{Export-Vorlage speichern}
 \label{fig:exportVorlageSpeichern}
\end{figure}

Die exportierte Daten befinden sich nun in der Zwischenablage. Wie in \nameref{subsec:bibtexImportieren} beschrieben fortfahren, um die Daten endgültig nach PUMA zu importieren.\newline

\subsubsection{Import aus JabRef\index{Import!JabRef}\index{JabRef}}\label{sss:importJabRef}
Mit der rechten Maustaste auf die Publikation, die nach PUMA importiert werden soll klicken.
Diese in die Zwischenablage im Export-Format BibTex kopieren.


\section{RSS-Feed abonnieren} 
\label{sec:rssFeedAbonnieren}
RSS\index{RSS} (engl. Really Simple Syndication)-Feeds informieren über Veränderungen auf Websites. In PUMA können eigene oder fremde Publikations-/~Lesezeichenlisten als RSS-Feed abonniert werden. Dies funktioniert auch mit Publikationslisten von Gruppen. Dazu über das Exportzeichen in der Publikations-/~Lesezeichenspalte, die abonniert werden soll auf \enquote{RSS} klicken. 

\begin{figure}[h!]
 \centering
 \fbox{\includegraphics[width=10cm]{Bilder/Kapitel7/RSS-feed_abonnieren}}
 \caption{RSS-Feed abonnieren}
 \label{fig:rssFeedAbbonnieren}
\end{figure}

\begin{figure}[h!]
 \centering
 \fbox{\includegraphics[width=11cm]{Bilder/Kapitel7/RSS_dynamisches_Lesezeichen}}
 \caption{Das dynamische Lesezeichen}
 \label{fig:dynamischesLesezeichen}
\end{figure}

\begin{figure}[h!]
 \centering
 \fbox{\includegraphics[width=11cm]{Bilder/Kapitel7/RSS-Reader}}
 \caption{Der RSS-Reader}
 \label{fig:rssReader}
\end{figure}

\section{Universitätsbibliografie\index{Unibibliografie}}
\label{sec:unibibliografie}
Die Universitätsbibliografie (kurz: Unibibliografie) bietet eine möglichst vollständige Übersicht über die Publikationen, die an der Universität Stuttgart veröffentlicht werden. Seit 2015 werden sämtliche Publikationen aller wissenschaftlichen Mitglieder (nach §9 LHG) der Universität Stuttgart angezeigt, die während und ggf. nach ihrer Zugehörigkeit zur Universität verfasst bzw. herausgegeben, öffentlich und dauerhaft verfügbar gemacht wurden.\newline\newline
Geführt wird die Unibibliografie der Universität Stuttgart über PUMA. Für die Mitglieder der Universität besteht die Möglichkeit, Publikationen über PUMA zu melden. Die Universitätsbibliothek bearbeitet die Datensätze und veröffentlicht diese Publikationsmetadaten in der Gruppe \textit{unibibliografie} in PUMA. Die Einträge dieser Gruppe können über \texttt{/group/unibibliografie} eingesehen und über \texttt{/export/group/unibibliographie} in viele verschiedene Formate exportiert werden (siehe auch \autoref{sss:nachGruppe} und \autoref{subsec:export}). Weitere Informationen gibt es auf der Homepage der Universitätsbibliothek Stuttgart. \footnote{\url{http://www.ub.uni-stuttgart.de/forschen-publizieren/unibibliografie/}}
\section{OPUS}
\label{sec:opus}
OPUS ist der Dokumentenserver (das institutionelle Repositorium) der Universität Stuttgart. Alle Angehörigen der Universität Stuttgart können über OPUS ihre Dokumente, die von dauerhaftem Interesse für Forschung und Lehre sind, online im Sinne von Open Access\index{Open Access} veröffentlichen. Durch diesen Schritt werden Publikationen im Internet langfristig und weltweit frei zugänglich. Zudem werden die Veröffentlichungen auch in Bibliothekskatalogen, Datenbanken und allen gängigen Suchmaschinen nachgewiesen und somit die Sichtbarkeit der Publikationen deutlich erhöht. Die Zitierfähigkeit der Veröffentlichungen wird durch eine dauerhafte, stabile Internet-Adresse (Persistent Identifier) garantiert.
\newline\newline
Geplant ist die Möglichkeit, aus PUMA direkt in OPUS\index{OPUS} zu veröffentlichen. Bei der Verwendung des \tags \textit{myown} (\autoref{sss:systemtags}) wird die Veröffentlichung auf OPUS angeboten.  

%\subsection{OPUS und PUMA}
%\label{subsec:opusPuma}
%Beim Eintragen einer Veröffentlichung/Publikation in PUMA wird die eigene Veröffentlichung mit \textit{myown} getaggt. Sie werden von PUMA gefragt, ob Sie auf OPUS veröffentlichen wollen. Wenn Sie diese Frage mit \enquote{Ja} bestätigen, wird eine SWORD-Datenverbindung\index{SWORD} zur Sherpa/Romeo-Liste\index{Sherpa/Romeo-Liste} hergestellt. Diese zeigt an, was die Verlage im Bezug auf die Veröffentlichung erlauben. Ist das Hochladen \enquote{grün}, so wird die Veröffentlichung auf OPUS  hochgeladen (so genanntes \enquote{Self Archiving}).
%\newline \newline
%Die Sherpa/Romeo-Liste\footnote{\url{http://www.sherpa.ac.uk/romeo/index.php}} ist eine Datenbank in Manchester, über die die Verlagskonditionen für Zweitveröffentlichungen abgefragt werden können. Bei deutschen Verlagen gilt ein Zeitfenster von 12 Monate nach Erstveröffentlichung, bevor die Autoren eine Zweitveröffentlichung machen können (Grüner Weg des Open Access). Bei Verlagen im Ausland gelten z. T. deutlich höhere Schutzfristen.
\section{DBLP}
\label{dblp}
Das Digital Bibliography \& Library Project (DBLP\index{DBLP}; zu deutsch: Digitales Bibliographie- und Bibliotheksprojekt) betreibt eine online verfügbare bibliographische Datenbank. In der Sammlung befinden sich mehr als 3 Mio. unterschiedliche wissenschaftliche Publikationen aus dem Bereich Informatik.\newline
PUMA und BibSonomy sind mit der DBLP-Datenbank verbunden. Die Datenbank wird mehrmals wöchentlich aktualisiert und stellt PUMA und BibSonomy Publikationen zur Nachnutzung zur Verfügung. \newline
Die Nutzer können so Einträge der DBLP-Datenbank mit ein paar Klicks in die eigene Sammlung übernehmen. 

\chapter{Zusammenarbeit und soziale Funktion}
\label{ch:zusammenarbeit}

\section{Freunde}% Screenshot von beiden Seiten (Menü und nutzerseite)
\label{sec:freunde}
PUMA bietet den Nutzern die Möglichkeit, andere Nutzer als Freunde zu kennzeichnen. Freundschaften\index{Freunde} ermöglichen das Teilen von Publikationen und Lesezeichen. Dazu in der Sichtbarkeitseinstellung eines Eintrags \enquote{andere} und dann \enquote{friends} auswählen, dann können zusätzlich \enquote{Freunde} diesen Eintrag sehen. Auf die gleiche Weise können Freunde Einträge sichtbar machen. Einen Überblick über diese Einträge erhält man über den Menüeintrag \enquote{meinPUMA} (\autoref{subsec:meinPuma}) unter \enquote{Einträge von Freunden} bzw. \enquote{Einträge für Freunde}.\newline

\subsection{Freund hinzufügen}
\label{subsec:freundHinzu}

Entweder auf den Benutzernamen unter einem Eintrag klicken oder im Suchfeld auf Benutzer einschränken oder in der allgemeinen Suche user:Benutzername eingeben (\autoref{sec:suche}).Eine Seite des Nutzers wird angezeigt, auf der alle öffentlichen Einträge zu sehen sind. In der rechten Menüleiste wird der Name eingeblendet darunter besteht die Möglichkeit diesen Nutzer über als Freund hinzuzufügen.
    
		
		%\begin{figure}[h!]
 %\centering
 %\fbox{\includegraphics[width=5cm]{Bilder/Kapitel8/Benutzername_in_Eintrag}}
 %\caption{Benutzername anklicken}
 %\label{fig:benutzerAnklicken}
%\end{figure}

		\begin{figure}[h!]
 \centering
 \fbox{\includegraphics[width=11cm]{Bilder/Kapitel8/Benutzer_suchen}}
 \caption{Benutzer suchen}
 \label{fig:benutzerSuchen}
\end{figure}

\begin{figure}[h!]
 \centering
 \fbox{\includegraphics[width=11cm]{Bilder/Kapitel8/Nutzerseite}}
 \caption{Die Nutzerseite}
 \label{fig:nutzerseite}
\end{figure}

\subsection{Freundesübersicht}
\label{subsec:freundesuebersicht}
Die Freundesübersicht bietet einen Überblick über Freunde in PUMA. Zu der Übersicht gelangt man über das Dropdown-Menü des Personensymbols. Unter dem Reiter \enquote{Freunde} erhält man einen Überblick über Freunde und kann sehen, welche Nutzer einen als Freund angegeben hat. Am Ende der Seite sind alle Publikationen aufgelistet, die mit Freunden geteilt wurden oder Freunden geteilt haben.\newline

\begin{figure}[h!]
 \centering
 \fbox{\includegraphics[width=11cm]{Bilder/Kapitel8/Freundesuebersicht}}
 \caption{Freundesübersicht}
 \label{fig:freundesuebersicht}
\end{figure}

Es besteht jederzeit die Möglichkeit, Freunde wieder zu entfernen. Hierfür mit der Maus auf den jeweiligen grünen Kasten \enquote{Freund} des Freundes, der entfernt werden soll klicken.

\section{Gruppen}
\label{sec:gruppen}
Gruppen\index{Gruppen} vereinfachen die Zusammenarbeit auf Puma. Sie ermöglichen eine gemeinsame Literaturrecherche und erleichtern so die Umsetzung von gemeinsamen Projekten. Gleichzeitig kann innerhalb einer Institution oder Arbeitsgruppe der Austausch über neue, interessante, fremde oder eigene Artikel mit Hilfe von PUMA erfolgen und somit die Kommunikation vereinfacht werden.

\subsection{Gruppen \index{Gruppen!beitreten}suchen und beitreten}
\label{subsec:gruppenSuchenBeitreten}

\begin{figure}[h!]
 \centering
 \fbox{\includegraphics[width=11cm]{Bilder/Kapitel8/Gruppen-Uebersichtsseite}}
 \caption{Allgemeine Liste}
 \label{fig:allgemeineListe}
\end{figure}
\begin{enumerate}

\item Im Hauptmenü \enquote{Gruppen} im . Ein Dropdown-Menü öffnet sich.
    \item Klicken Sie im Dropdown-Menü auf \enquote{Alle Gruppen}.
    \item Es öffnet sich eine Übersicht über alle Gruppen bei PUMA in alphabetischer Reihenfolge. Rechts neben dem jeweiligen Gruppennamen befindet sich ein Button, um der Gruppe beizutreten. Klicken Sie auf den Beitreten-Button der gewünschten Gruppe.
		\end{enumerate}
\begin{figure}[h!]
 \centering
 \fbox{\includegraphics[width=11cm]{Bilder/Kapitel8/Beitreten_einer_Gruppe}}
 \caption{Beitreten einer Gruppe}
 \label{fig:gruppeBeitreten}
\end{figure}
\begin{enumerate}
    \item Eine neue Seite erscheint. Geben Sie in das Feld \enquote{Begründung} ein, warum Sie der Gruppe beitreten möchten.
    \item Geben Sie den angezeigten Captcha-Text in das vorgegebene Feld ein. Damit soll verhindert werden, dass Programme automatisiert Gruppen beitreten. 
    \item Klicken Sie anschließend auf \enquote{Anfrage absenden}.
    \item Der Gruppen-Administrator erhält eine E-Mail-Benachrichtigung, dass Sie in die Gruppe eintreten wollen. Allein der Administrator entscheidet über die Aufnahme, weswegen ein plausibler Begründungstext sinnvoll ist.
\end{enumerate}
In der E-Mail, die alle Administratoren erhalten, befindet sich ein Link (erster Link in der E-Mail). Durch das Anklicken des Links öffnet sich die Einstellungsseite der Gruppe. Der Administrator kann nun den Nutzer aufnehmen oder dessen Beitrittsanfrage ablehnen.
\subsection{Gruppen erstellen\index{Gruppen!erstellen}}
\label{subsec:gruppenErstellen}
\begin{enumerate}
    \item Klicken Sie im Hauptmenü auf \enquote{Gruppen}. Ein Dropdown-Menü öffnet sich.
    \item Klicken Sie im Dropdown-Menü auf \enquote{Eine neue Gruppe erstellen}.
\begin{figure}[h!]
 \centering
 \fbox{\includegraphics[width=11cm]{Bilder/Kapitel8/Neue_Gruppe_erstellen}}
 \caption{Erstellung einer neuen Gruppe}
 \label{fig:erstellungNeueGruppe}
\end{figure}
    \item Geben Sie einen Gruppennamen und eine Beschreibung der Gruppe an. 
    \item Klicken Sie anschließend auf \enquote{Gruppe erstellen}. 
\end{enumerate}
Ab sofort können Sie die Vorteile der gemeinsamen Literaturrecherche von PUMA nutzen und Publikationen für spezielle Gruppen sichtbar machen. Dies legen Sie beim Eintragen einer neuen Publikation oder eines neuen Lesezeichens fest, indem Sie bei der Sichtbarkeit\index{Sichtbarkeit} unter dem Punkt \textit{andere} die spezielle Gruppe auswählen. Wenn Sie diese Publikation nun speichern, sehen diese automatisch alle Gruppenmitglieder.
\subsection{Die Gruppenseite}
\label{subsec:gruppenseite}
Um zur Gruppenseite\index{Gruppen} zu gelangen, klicken Sie im Dropdown-Menü vom Reiter  \enquote{Gruppen} auf den entsprechenden Namen der Gruppe. Sie gelangen zur Gruppenseite, auf der Sie einen Überblick über alle Lesezeichen und Publikationen erhalten.%Screenshot
\newline\newline
Funktionen auf der Gruppenseite:
\begin{figure}[h!]
 \centering
 \fbox{\includegraphics[width=11cm]{Bilder/Kapitel8/Gruppenseite}}
 \caption{Die Gruppenseite}
 \label{fig:gruppenseite}
\end{figure}
\begin{description}
\item [CV/Lebenslauf der Gruppe] \hfill \\
Durch Klicken auf den CV-Button auf der rechten Seite erhalten Sie alle wichtigen Informationen zu der Gruppe.
\item [Mitglieder-Liste] \hfill \\
Unterhalb des Gruppenbildes befindet sich die Liste aller Mitglieder. 
\item [Diskussionen] Um sich einen Überblick über die diskutierten Einträge zu verschaffen, klicken Sie unter dem Abschnitt Diskussion auf der rechten Seite auf \enquote{Zeige kürzlich diskutierte Einträge von PUMA}. 
\end{description}
 
\subsection{Rollen in einer Gruppe}
\label{subsec:RollenInGruppe}
In einer Gruppe können die unterschiedlichsten Rollen und Aufgaben übernommen werden. In PUMA gibt es drei Rollenarten:
\begin{description}
    \item [Administrator\index{Administrator}:] Er hat die größte Befugnis in der Gruppe. Er ist zuständig für die Einstellungen der Gruppenseite und kann das Layout des Gruppenlebenslaufes editieren. Einträge, die in die Gruppe eingetragen werden, können von Ihm/Ihr bearbeitet werden. Ebenfalls kann er neue Mitglieder einladen und vorhandene ausladen sowie die Rollen der anderen Mitglieder verändern (z.B. weiteren Administrator ernennen).
    \item [Moderator\index{Moderator}:] Der Moderator hat Zugriff auf die Mitgliederliste und kann andere Nutzer in die Gruppe einladen und seine eigene Rolle auf \textit{Nutzer} herabsetzen.
    \item [Nutzer\index{Nutzer}:] Er ist ein Mitglied der Gruppe und hat keine Befugnisse, in der Gruppe Änderungen oder neue Einstellungen vorzunehmen.
\end{description}

\subsection{Einträge für eine Gruppe}
\label{subsec:gruppenfunktion}
Sobald ein Nutzer Mitglied einer Gruppe ist, werden seine öffentlichen Einträge automatisch in der Sammlung der Gruppe angezeigt. Die anderen Mitglieder können diese Publikation aber erst bearbeiten wenn sie die Publikation in Ihre eigene Sammlung übertragen. Eine weitere Möglichkeit, eine Publikation in die Gruppensammlung zu übertragen, bietet das Gruppenoptionsfeld. Damit kann beim Eintragen der Publikation eine Gruppe ausgewählt werden, für die die Publikation interessant ist. Auch diese Einträge können erst dann von den Mitgliedern bearbeitet werden, wenn sie in die eigene Sammlung aufgenommen werden.

Damit andere Mitglieder (Administratoren) einen Eintrag bearbeiten können, muss beim Eintragen der Publikation der Systemtag \textit{for:gruppenname} eingegeben werden. Der Eintrag erscheint wie alle anderen Einträge in der Sammlung der Gruppe. Als Nutzer dieses Eintrags wird der Gruppenname (@gruppenname) angegeben. In der Reihe der Tags erscheint der Systemtag \textit{from:Benutzername}, welcher den genauen Verfasser des Eintrags angibt. In der Detailansicht der Publikation erhalten nun die Administratoren der Gruppe die Möglichkeit, über den schwarzen Stift oben rechts die Publikation zu bearbeiten. Hierfür muss die Publikation nicht in die Sammlung des Administrators übernommen werden.

\section{Community Post}
\label{sec:communityPost}
Ein Community Post\index{Community Post} ist ein Gemeinschaftseintrag, auf den mehrere Personen Zugriff haben. \newline \newline
\textbf{Erstellen eines Community Posts:}
\begin{enumerate}
	\item Klicken Sie auf den Titel der Publikation, um zur Detailansicht der Publikation zu gelangen. 
	\item Gehen Sie mit der Maus auf den kleinen schwarzen Pfeil oben rechts neben dem Publikationstitel. 
	\item Wählen Sie im Dropdown-Menü \enquote{CommunityPost} aus. \end{enumerate}
\begin{figure}[h!]
 \centering
 \fbox{\includegraphics[width=11cm]{Bilder/Kapitel8/Community_post_anlegen}}
 \caption{Community Post anlegen}
 \label{fig:communityPostAnlegen}
\end{figure}
Der Community Post öffnet sich. Sie können nun Änderungen an der Publikation vornehmen, indem Sie oben links auf der Seite auf den Stift klicken, oder weiter unten auf der Community-Seite die Publikation bewerten. Die Änderungen werden in einer Übersicht - der Versionsgeschichte\index{Versionierung} - dargestellt. Zu dieser gelangen Sie oben links, durch einen Klick auf das Verzeichnissymbol.\newline
Durch die Erstellung eines Community Post können Nutzer jederzeit auf die Versionsgeschichte des Eintrages zugreifen und sehen, was und wann von wem geändert wurde. So erleichtert er die Zusammenarbeit und ermöglicht einen umfassenden Überblick. 

Im Bereich \textit{Tags} werden die Tags des Gemeinschaftseintrags angezeigt. Durch einen Klick auf einen Tag werden einem alle Publikationen mit diesem Tag angezeigt.

Unter dem Bereich \textit{Nutzer} werden alle Nutzer angezeigt, die diese Publikation in Ihrer Sammlung eingetragen haben.
\begin{figure}[h!]
 \centering
 \fbox{\includegraphics[width=11cm]{Bilder/Kapitel8/Community_post_Versionsgeschichte}}
 \caption{Versionierung}
 \label{fig:versionierung}
\end{figure}
\section{Nutzern folgen}
\label{sec:nutzernFolgen}
Wie in sozialen Netzwerken bietet PUMA seinen Nutzern die Möglichkeit, anderen Nutzern zu folgen. 

Um einem Nutzer zu folgen, gehen Sie auf dessen Benutzerseite (\autoref{subsec:freundHinzu}). Klicken Sie rechts oben, unterhalb des Benutzerprofilbildes, auf das Feld \enquote{folgen}. Ab sofort sind Sie ein Follower des Nutzers. Sie können jedem Nutzer folgen, egal ob befreundet oder nicht. \newline \newline
Um eine Überblick über die Einträge der verfolgten Person zu erhalten, klicken Sie im Reiter \enquote{meinPUMA} (\autoref{subsec:meinPuma}) auf den Unterpunkt \enquote{verfolgte Einträge}. Es erscheint eine Übersichtsseite mit allen Einträgen der Nutzer, denen Sie folgen. 




%Überarbeiten:neue version anders

\section{Kommentare, Rezensionen und Bewertungen}
\label{sec:kommentare}
PUMA verfügt über die Möglichkeit, Publikationen und Lesezeichen zu bewerten\index{Bewerten} und Rezensionen\index{Rezensionen} zu verfassen. Man kann mit anderen Nutzern über Publikationen/~Lesezeichen diskutieren und seine eigene Meinung zu einer Publikation/~einem Lesezeichen durch die Vergabe von Sternen verdeutlichen.
\newline
\newline
Publikationen/Lesezeichen bewerten:
\begin{enumerate}
    \item Klicken Sie auf die Stern-Leiste (siehe \autoref{fig:sternenleiste}) unterhalb des Lesezeichens oder der Publikation, die Sie bewerten möchten.
\begin{figure}[h!]
 \centering
 \fbox{\includegraphics[width=11cm]{Bilder/Kapitel8/Die_Sternenleiste}}
 \caption{Die Stern-Leiste}
 \label{fig:sternenleiste}
\end{figure}  
    \item Es öffnet sich die Gemeinschaftsseite des Eintrages. Neben den Bereichen \textit{Tags} und \textit{Zitieren Sie diese Publikation} finden Sie hier auch den Bereich \textit{Kommentare und Rezensionen}. \autoref{fig:publikationBewerten} zeigt die Elemente des Bereichs:
\begin{figure}[h!]
 \centering
 \fbox{\includegraphics[width=11cm]{Bilder/Kapitel8/Publikation_bewerten}}
 \caption{Publikation bewerten}
 \label{fig:publikationBewerten}
\end{figure}
    \begin{description} 
        \item [Bewertungsverteilung (A):] Das Balkendiagramm stellt dar, welche Bewertungen wie oft vergeben wurden.  %In diesem Fall ... Beispiel an Hand eines Bildes
        \item [Durchschnittliche Bewertung (B):] In der Stern-Leiste wird der Mittelwert der Bewertungen angezeigt.
        \item [Rezension schreiben (C):] \hfill \\
				Durch einen Klick auf den Button "Rezension schreiben" öffnet sich ein Textfeld, das Ihnen die Möglichkeit bietet, ein Review zu verfassen. Oberhalb des Textfeldes können Sie den Beitrag mit null bis fünf Sternen bewerten. Je höher die Anzahl der Sterne, umso besser ist die Bewertung. Unterhalb des Textfeldes können Sie die Sichtbarkeit Ihrer Bewertung festlegen und so entscheiden, wer sie sehen darf. Es gibt folgende Möglichkeiten:
        \begin{enumerate}
            \item öffentlich: Jeder Nutzer kann Ihre Rezension sehen.
            \item privat: Nur Sie können Ihre Rezension sehen.
            \item Freunde: Sie können einzelne Freunde festlegen, die Ihre Rezension sehen sollen.
            \item Gruppen: Es werden Ihnen alle Gruppen angezeigt, in denen Sie Mitglied sind. Wählen Sie aus, welche Gruppe die Rezension sehen soll.
            \item anonym: Ihr Kommentar wird ohne Ihren Benutzernamen veröffentlicht. Die Bewertung ist für alle Nutzer sichtbar.
        \end{enumerate}
       	Klicken Sie abschließend auf \enquote{Bewerten}, um die Rezension abzuschließen und sie sichtbar zu machen.
        \item [Kommentar schreiben (D):] \hfill \\
				In diesem Textfeld können Sie einen Kommentar verfassen. Ein Kommentar hat die gleichen Möglichkeiten der Sichtbarkeit wie eine Rezension.
\newline Es kann beliebig oft auf Kommentare/~Bewertungen reagiert und geantwortet werden. Neben jedem Kommentar befindet sich ein Button mit einem kleinen schwarzen Pfeil, über den Sie Rezensionen direkt kommentieren können. 
    \end{description}
\end{enumerate}
\section{Für Nerds}
\textit{Die eigene Sammlung auf der eigenen Homepage veröffentlichen. Mit Hilfe von Pugins, wie OpenCms,  und dem JavaScript-Codeschnipsel kein Problem für PUMA.}
\subsection{URL-Sytax\index{URL-Syntax}}
\textbf{Parameter zum Sortieren} \newline
Immer wenn Sie in PUMA Zugriff auf eine Lesezeichen/Publikationsliste haben, können Sie diese sortieren, indem Sie an die URL einen/mehrere der folgenden Parameter anhängen. Folgende Parameter stehen Ihnen zur Verfügung:
\begin{enumerate}
    \item \textbf{sortPage - Wonach wird sortiert?}
    \begin{enumerate}
        %\item Werte (können durch | verknüpft werden):%ist nicht so
        \item author - Autorenname
        \item editor - Herausgebername
        \item year - Erscheinungsjahr
        \item entrytype - Publikationstyp
        \item title - Titel
        \item booktitle - Buchtitel (insb. bei Artikel in Sammelbänden)
        \item journal - Journalname
        \item school - Universitätsname 
    \end{enumerate}
    \item \textbf{sortPageOrder - Reihenfolge der Sortierung}
    \begin{enumerate}
        \item asc - aufsteigend
        \item desc - absteigend 
    \end{enumerate}
    \item \textbf{duplicates- Duplikate\index{Duplikate}}
    \begin{enumerate}
        \item yes - Erlaubt Duplikate in der Lesezeichen/- Publikationsliste
        \item no - Entfernt Duplikate aus der Ergebnisliste
    \end{enumerate}
\end{enumerate}
%Beispiel: \url{?sortPage=year%\&sortPageOrder=asc\&duplicates=no} \newline
Sortiere nach Erscheinungsjahr (sortPage=year) aufsteigend (sortPageOrder=asc) und entferne alle Duplikate (duplicates=no). \newline
\newline
\textbf{Allgemeine Seiten}
\begin{enumerate}
    \item \textbf{/} \newline
    Homepage von PUMA, zeigt die aktuellsten 50 öffentlichen Einträge.
    \item \textbf{/popular} \newline
    Zeigt die 100 häufigsten Einträge der letzten 100.000 öffentlichen Einträge.
    \item \textbf{/IhrBenutzername} \newline
    Sie gelangen zu Ihrer persönlichen Sammlung.
    \item \textbf{/settings} \newline
    Auf dieser Seite können Sie:
    \begin{enumerate}
        \item Ihr Profil bearbeiten und Kontoeinstellungen ändern,
        \item einen Benutzer zu Ihrer Gruppe hinzufügen,
        \item Ihren API-Schlüssel finden und einen neuen erzeugen,
        \item Ihr Passwort ändern und
        \item Ihre Daten zwischen BibSonomy und PUMA synchronisieren.
    \end{enumerate}
    \item \textbf{/help\_de} \newline
    Sie gelangen zu der Hilfeseite.
    %\textbf{Verwaltungsseiten} ob diese Ordnung oder eine andere
    \item \textbf{/postBookmark} \newline
    Hier können Sie über die Eingabe der URL eines Lesezeichen überprüfen, ob sich dieses Lesezeichen schon in Ihrer Sammlung befindet. Durch die Überprüfung können Sie Duplikate vermeiden. Wenn sich das Lesezeichen noch nicht in Ihrer Sammlung befindet haben Sie im Anschluss an die Überprüfung die Möglichkeit die Metadaten des Lesezeichens einzutragen, um es in Ihre Sammlung auf zu nehmen.
    \item \textbf{/postPublication} \newline
    Auf dieser Seite können Sie neue Publikationen eintragen. 
    \item \textbf{/user/eckert} \newline
    Zeigt alle öffentlichen Einträge des Benutzers \textit{eckert}.
    \item \textbf{/user/eckert/politik} \newline
    Zeigt alle öffentlichen Einträge mit dem Tag \textit{politik} des Benutzers \textit{eckert}.
    \item \textbf{/user/eckert/politik+menschenrechte} \newline
    Zeigt alle öffentlichen Einträge mit dem Tag \textit{politik} und dem Tag \textit{menschnerechte} des Benutzers \textit{eckert}.
     \item \textbf{/myBibTeX} \newline
    Ihnen wird Ihre gesamte Sammlung im BibTex-Format angezeigt.
    \item \textbf{/myRelations} \newline
    Ihnen werden Ihre Konzepte/ Relationen angezeigt.
    \item \textbf{/myDuplicates} \newline
    Zeigt Ihre eigenen Duplikaten, die sich in Ihrer Sammlung befinden.
\end{enumerate}
    
\subsubsection{Suchen mit der URL-Syntax}
PUMA bietet Ihnen die Möglichkeit mit Hilfe der URL-Syntax nach bestimmten Seiten zu suchen. Es gibt verschiedene Wege, diese Suchergebnisse zu filtern. Gegenwärtig beinhalten die Filter: Die Tags, den Autor, das Publikationsjahr, den Benutzernamen der Person, die den Eintrag gespeichert hat sowie Freunde- und Gruppennamen. \newline
\newline
\textbf{Tag\index{Tags}-/Schlagwortseiten}
\begin{enumerate}
    \item \textbf{/tag/politik} \newline
    Zeigt alle öffentlichen Einträge mit dem Tag \textit{politik} an.
    \item \textbf{/tag/politik+menschenrechte}\newline
    Zeigt alle öffentlichen Einträge mit dem Tag \textit{politik} und dem Tag \textit{menschenrechte} an.
\end{enumerate}
\textbf{Autorenseiten\index{Autoren}}
\begin{enumerate}
    \item \textbf{/author/müller} \newline
    Zeigt alle Einträge mit dem Autornamen \textit{Müller} an.
    %\item \textbf{/author/stumme+hotho+schmitz} \newline
    %Zeigt alle Publikationen, die von den Autoren Stumme, Hotho und Schmitz veröffentlicht wurden.
    %\item \textbf{/author/stumme+hotho+!schmitz} \newline
    %Geben Sie den Namen eines Autors an, der nicht Teil der gesuchten Publikationen sein soll, z.B. zeige alle Publikationen an, die von Stumme und Hotho geschrieben sind, aber nicht von Schmitz.
    \item \textbf{/author/müller/dblp} \newline
    Zeigt alle Einträge mit dem Tag \textit{dblp} und dem Autor \textit{Müller}.
    \item \textbf{/author/müller/sys:user:eckert}\newline
    Zeigt alle Publikationen des Autors \textit{Müller} in der Sammlung von\textit{Eckert} an.
    %\item \textbf{/author/stumme+hotho+!schmitz+sys:year:2002-2007+sys:user:hoth} \newline
    %Diese Kombination von Suchergebnissen zeigt alle Publikationen der Autoren Stumme und Hotho, aber nicht von Schmitz in den Jahren 2002 bis 2007 in der Sammlung des Benutzers Hotho.
    \item \textbf{/author/müller/sys:group:puma} \newline
    Zeigt alle Publikationen des Autors \textit{Müller} in der Sammlung aller Gruppenmitglieder der Gruppe \textit{puma} an. 
\newline
\newline
Ein Systemtag (System-Schlagwort) kann das Ergebnis Ihrer Autoren-Suche auf ein bestimmtes Erscheinungsjahr oder einen bestimmten Zeitraum beschränken. Es sind vier Formate möglich:%muss rausrücken
    \item \textbf{/author/hotho+sys:year:2007} \newline
    Zeigt alle Publikationen des Autors \textit{Hotho} aus dem Jahre 2007.
    \item \textbf{/author/hotho+sys:year:2003-2007} \newline
    Zeigt alle Publikationen des Autors \textit{Hotho} zwischen 2003 und 2007.
    \item \textbf{/author/hotho+sys:year:-2005} \newline
    Zeigt alle Publikationen des Autors \textit{Hotho} bis zum Jahr 2005 an.
    \item \textbf{/author/hotho+sys:year:1997-} \newline
    Zeigt alle Publikationen des Autors \textit{Hotho} seit 1997 an.
\end{enumerate}
\textbf{Freundeseiten} 
\begin{enumerate}
    \item \textbf{/friends} \newline
    Hier werden die Einträge angezeigt, die nur für Freunde\index{Freunde} sichtbar sind und von Benutzern stammen auf deren Freundesliste Sie stehen.
    \item \textbf{/friend/eckert} \newline
    Hier werden alle Beiträge angezeigt, welche für Freunde des Benutzers \textit{eckert} sichtbar gesetzt sind. Sie können diese Einträge nur dann sehen, wenn \textit{eckert} Sie als Freund angegeben hat.
    \item \textbf{/friend/eckert/politik} \newline
    Zeigt alle Beiträge mit dem Tag \textit{politik} an, welche für Freunde des Benutzers \textit{eckert} sichtbar gesetzt sind. Sie können sie nur dann sehen, wenn \textit{eckert} Sie als Freund angegeben hat.
    \item \textbf{/friend/eckert/politik+menschenrechte} \newline
    Zeigt alle Beiträge mit den Tags \textit{politik} und \textit{menschenrechte} an, welche für Freunde des Benutzers \textit{eckert} sichtbar gesetzt sind. Sie können sie nur dann sehen, wenn \textit{eckert} Sie als Freund angegeben hat.
\end{enumerate}
\textbf{Gruppenseiten\index{Gruppen}}
\begin{enumerate}
    \item \textbf{/groups} \newline
    Zeigt alle Gruppen, die es in PUMA gibt, an.
    \item \textbf{/group/puma} \newline
    Zeigt Ihnen alle Einträge von Mitgliedern der Gruppe \textit{puma} an, wenn Sie Gruppenmitglied sind.
    \item \textbf{/group/puma/politik}\newline
    Zeigt Ihnen alle Einträge mit dem Tag \textit{politik} von Mitgliedern der Gruppe \textit{puma} an, wenn Sie Gruppenmitglied sind.
    \item \textbf{/group/puma/politik+menschnerechte}\newline
    Zeigt Ihnen alle Einträge mit dem Tag \textit{politik} und dem Tag \textit{menschenrechte} von Mitgliedern der Gruppe \textit{puma} an, wenn Sie Gruppenmitglied sind.
    \item \textbf{/relevantfor/group/puma} \newline
    Zeigt Ihnen alle Einträge an,  die für die Teilnehmer der Gruppe relevant sind.
    \item \textbf{/followers} \newline
    Zeigt die neuesten Einträge aller Benutzer, denen Sie folgen. Diese Einträge werden mittels eines Rankings so umsortiert, dass die für Sie relevantesten Einträge ganz oben stehen. %wie wird man follower???
\end{enumerate}
\textbf{Konzeptseiten}
\begin{enumerate}
    \item \textbf{/concepts/eckert} \newline
    Es werden Ihnen alle Konzepte\index{Konzepte} des Benutzers \textit{eckert} angezeigt.
    \item \textbf{/concept/user/eckert/psychologie} \newline
    Zeigt alle Lesezeichen und Publikationen des Benutzers \textit{eckert} an, denen das Tag \textit{psychologie} oder eines der Unterschlagwörter des Konzeptes als Tag zugeordnet ist. 
\end{enumerate}
\textbf{Suchseiten}\newline
Mit der URL-Syntax \textit{/search...} suchen Sie im Volltext nach einem bestimmten Wort. Es handelt sich dabei nicht um Schlagwörter/Tags. Bei Lesezeichen enthält der Volltext die URL, den Titel und die Beschreibung. Bei Publikationen sind der Titel, die Beschreibung und alle BibTex-Felder enthalten.
\begin{enumerate}
    \item \textbf{/search/politik} \newline
    Zeigt Ihnen alle öffentlichen Einträge an, die im Volltext (nicht in den Schlagwörtern!) das Wort \textit{politik} enthalten. 
    \item \textbf{/search/politik+menschenrechte}\newline
    Zeigt alle öffentlichen Einträge, die im Volltext (nicht in den Schlagwörtern!) das Wort \textit{politik} und das Wort \textit{menschenrechte} enthalten. 
    \item \textbf{/search/politik+-menschenrechte} \newline
    Zeigt alle öffentlichen Einträge an, die im Volltext (nicht in den Schlagwörtern!) das Wort \textit{politik}, aber nicht das Wort \textit{menschenrechte} enthalten. 
    \item \textbf{/search/politik+user:droessler} \newline
    Zeigt alle öffentlichen Einträge des Benutzers \textit{droessler}, die im Volltext (nicht in den Schlagwörtern!) das Wort \textit{politik} enthalten. 
    \item \textbf{/search/politik+menschenrechte+user:droessler}\newline
    Zeigt alle öffentlichen Einträge des Benutzers \textit{droessler} an, die im Volltext (nicht in den Schlagwörtern!) das Wort \textit{politik} und das Wort \textit{menschenrechte} enthalten. 
    \item \textbf{/search/politik+-menschenrechte+user:droessler} \newline
    Zeigt alle öffentlichen Einträge des Benutzers \textit{droessler} an, die im Volltext (nicht in den Schlagwörtern!) das Wort \textit{politik}, aber nicht das Wort \textit{menschnerechte} enthalten. 
    \item \textbf{/mySearch} \newline
    Diese Seite bietet eine Schnellsuche in Ihrer eigenen Sammlung.
\end{enumerate}
\textbf{Sichtbare Seiten\index{Sichtbarkeit}}
\begin{enumerate}
    \item \textbf{/viewable/public} \newline
    Zeigt alle Ihre Einträge an, die Sie als \enquote{öffentlich sichtbar\index{Sichtbarkeit!öffentlich}} eingestellt haben.
    \item \textbf{/viewable/public/politik} \newline
    Zeigt alle Ihre Einträge mit dem Tag \textit{politik} an, die Sie als \enquote{öffentlich sichtbar} eingestellt haben.
    \item \textbf{/viewable/public/politik+menschenrechte} \newline
    Zeigt alle Ihre Einträge mit dem Tag \textit{politik} und dem Tag \textit{menschenrechte}, die Sie als \enquote{öffentlich sichtbar} eingestellt haben.
    \item \textbf{/viewable/private} \newline
    Zeigt alle Ihre Einträge, die Sie als \enquote{privat sichtbar\index{Sichtbarkeit!privat}} eingestellt haben.
    \item \textbf{/viewable/private/politik} \newline
    Zeigt alle Ihre Einträge mit dem Tag \textit{politik} an, die Sie als \enquote{privat sichtbar} eingestellt haben.
    \item \textbf{/viewable/private/politik+menschenrechte} \newline
    Zeigt alle Ihre Einträge mit dem Tag \textit{politik} und dem Tag \textit{menschenrechte} an, die Sie als \enquote{privat sichtbar} eingestellt haben.
    \item \textbf{/viewable/friends} \newline
    Zeigt alle Ihre Einträge an, die Sie als \enquote{für Freunde sichtbar\index{Sichtbarkeit!Freunde}} eingestellt haben.
    \item \textbf{/viewable/friends/politik} \newline
    Zeigt alle Ihre Einträge mit dem Tag \textit{politik} an, die Sie als \enquote{für Freunde sichtbar} eingestellt haben.
    \item \textbf{/viewable/friends/politik+menschenrechte} \newline
    Zeigt alle Ihre Einträge mit dem Tag \textit{politik} und dem Tag \textit{menschenrechte} an, die Sie als \enquote{für Freunde sichtbar} eingestellt haben.
    \item \textbf{/viewable/puma} \newline
    Zeigt alle Einträge an, die für die Gruppe \textit{puma} als sichtbar eingestellt wurden.
    \item \textbf{/viewable/puma/politik} \newline
    Zeigt alle Einträge mit dem Tag \textit{politik} an, die für die Gruppe \textit{puma} als sichtbar eingestellt wurden.
    \item \textbf{/viewable/puma/politik+menschenrechte} \newline
    Zeigt alle Einträge mit dem Tag \textit{politik} und dem Tag \textit{menschenrechte}, die für die Gruppe \textit{puma} als sichtbar eingestellt wurden.
\end{enumerate}
\textbf{Umgang mit Duplikaten\index{Duplikate}}\newline
Auf Gruppenseiten kann es häufig vorkommen, dass Einträge (Publikationen) mehrfach angezeigt werden, wenn innerhalb einer Gruppe zwei oder mehr Benutzer denselben Eintrag in Ihrer Sammlung haben.\newline
Falls dies nicht gewünscht ist, kann das Verhalten mittels des Parameters \textit{duplicates} wie folgt angepasst werden:
\begin{enumerate}
    \item \textbf{/group/puma/myown} \newline
    Zeigt alle Einträge der Gruppe \textit{puma} an, die mit dem Tag \textit{myown} annotiert sind (auch Duplikate).
    \item \textbf{/group/puma/myown?duplicates=no} \newline
    Zeigt alle Einträge der Gruppe \textit{puma} an, die mit dem Tag \textit{myown} annotiert sind. Für jedes Duplikat wird nur der erste Eintrag angezeigt.
    \item \textbf{/group/puma/myown?duplicates=merged} \newline
    Zeigt alle Einträge der Gruppe \textit{puma} an, die mit dem Tag \textit{myown} annotiert sind. Für jedes Duplikat werden alle Tags \enquote{aufgesammelt} und aggregiert an einem einzelnen Eintrag angezeigt.
\end{enumerate}


\textbf{Export von Seiten}
\begin{enumerate}
    \item RSS Feeds\index{RSS}
    \begin{enumerate}
        \item \textbf{/publrss/} \newline
        Zeigt einen RSS-Feed der Publikationen aus dem Inhaltsbereich an.
        \item \textbf{/burst/} \newline
        Zeigt ein BuRST-Feed für die Publikationen aus dem Inhaltsbereich an.
        \item \textbf{/aparss/} \newline
        Zeigt ein RSS-Feed im APA-Format für die Publikationen aus dem Inhaltsbereich an.
    \end{enumerate}
    \item Referenz-Metadaten und Formatierung
    \begin{enumerate}
        \item \textbf{/bib/} \newline
        Zeigt alle Publikationen aus dem Inhaltsbereich im BibTeX-Format\index{BibTex} an.
        \item \textbf{/bib/user/eckert} \newline
        Zeigt alle öffentlichen Publikationseinträge des Nutzers \textit{eckert} im BibTeX-Format an.
        \item \textbf{/endnote/} \newline
        Zeigt alle Publikationen aus dem Inhaltsbereich im EndNote-Format an.
    \end{enumerate}
    \item HTML-Formatierung\index{HTML}
    \begin{enumerate}
        \item \textbf{ /publ/} \newline
        Es wird eine einfache Übersicht angezeigt, in der jeder Eintrag als Zeile in einer Tabelle dargestellt ist.
        \item \textbf{/publ/?notags=1} \newline
        In der einfachen Tabellenübersicht werden die PUMA-Schlagwörter in der HTML-Ausgabe unterdrückt.
        \item \textbf{/publ/user/eckert} \newline
        Es werden die Publikationen des Nutzers \textit{eckert} in Tabellenform dargestellt.
        \item \textbf{ /publ/user/eckert/myown} \newline
        Es werden die Publikationen des Nutzers \textit{eckert}, die unter dem Tag \textit{myown} abgespeichert wurden, in der Tabellenübersicht angezeigt. 
    \end{enumerate}
    \item Semantic Web-Formatierung
    \begin{enumerate}
        \item \textbf{/swrc/} \newline
        RDF-Ausgabe gemäß der SWRC-Ontologie.
    \end{enumerate}
\end{enumerate}







\textbf{URL- \index{URL}oder BibTex-Seiten\index{BibTex}}
\begin{enumerate}
    \item \textbf{/url/398aa54c3aea66c147ad74d3089c0612}\newline
    Zeigt Ihnen alle öffentlichen PUMA-Lesezeicheneinträge der URL mit dem MD5-Hash \textit{398aa54c3aea66c147ad74d3089c0612} an.
    \item \textbf{/url/0fa29f649ff82603a98854e0fbbd2cd1/eckert}\newline Zeigt Ihnen die PUMA-Lesezeicheneinträge des Benutzers \textit{eckert} mit dem MD5-Hash \textit{0fa29f649ff82603a98854e0fbbd2cd1} an.
	\item \textbf{/bibtex/1edc3d2bbf4673d84363a675ee64b49bd}\newline Zeigt alle öffentlichen PUMA-Publikationseinträge mit dem Hashkey\index{Hashkey}\newline \textit{1edc3d2bbf4673d84363a675ee64b49bd} an. Der benutzte Hash ist der Inter-Hash.
    \item \textbf{/bibtex/253aa20e7f5e790b745e604039667c47b/eckert}\newline
    Zeigt den PUMA-Publikationseintrag des Benutzers \textit{eckert} mit dem\newline Hashkey 253aa20e7f5e790b745e604039667c47b an. Der benutzte Hash ist der Intra-Hash. PUMA liefert einen Tag-JSON-Feed, der zu einem BibTeX-Eintrag gehört.
    \item \textbf{/json/tags/bibtex/218a34049610d50537e6e09ce71b65605}\newline Diese URL liefert eine JSON-Ausgabe. Sie enthält alle Schlagwörter, welche in Beziehung mit der Publikation stehen, die dem Inter-Hash \textit{218a34049610d50537e6e09ce71b65605} entsprechen. PUMA bietet die Möglichkeit, eine Publikation anhand ihres BibTex-Schlüssels abzurufen.
    \item \textbf{/bibtexkey/Martin\_2014} \newline
    Liefert Publikationen mit dem BibTex-Schlüssel \textit{Martin\_2014}.
    \item \textbf{/bibtexkey/Martin\_2014/droessler} 
    oder \newline \textbf{/bibtexkey/Martin\_2014+sys:user:droessler}\newline
    Liefert Publikationen mit dem BibTex-Schlüssel \textit{Martin\_2014} des Nutzers \textit{droessler}.
    \item \textbf{/bibtexkey/Martin\_2014+sys\%3Auser\%3Adroessler} \newline
    Zeigt alle Einträge mit dem vorgegebenen BibTex-Schlüssel \textit{Martin\_2014} des Benutzers \textit{droessler} an. Haben Sie mehr als einen Eintrag mit dem gleichen BibTeX-Schlüssel, so erhalten Sie eine Liste aller Treffer.
    \item \textbf{/bibtexkey/journals/jacm/HopcroftU69/dblp} \newline
    Sie können die BibTex-Semantik benutzen, um auf Einträge zu verweisen, die wir von DBLP spiegeln  %, sobald Sie gelernt haben, wie DBLP seine BibTeX-Schlüssel erzeugt. noch nachschauen
\end{enumerate}

\textbf{Inhaltsvereinbarungsseiten}
\begin{enumerate}
    \item 
    \item \textbf{/uri/author/eckert} \newline
    Zeigt alle Einträge mit dem Autor Namens Eckert.
\end{enumerate}

\subsubsection{Rest-API\index{Rest-API}}
PUMA bietet Ihnen einen Webservice auf Basis des Representational State Transfer (REST) an. REST ist ein Architekturstil für verteilte Softwaresysteme, bei dem eine einheitliche Schnittstelle die Interaktion erleichtert. \newline
Die REST-API ist für Softwareentwickler gedacht, deren Anwendungen mit PUMA interagieren sollen. Um auf die API zuzugreifen, können Sie die angebotene Client Library in der Programmiersprache Java\index{Java} nutzen. Falls Sie mit einer anderen Programmiersprache schreiben möchten, können Sie direkt mit dem Webserver interagieren.\newline
Das REST-API-Repository\footnote{\url{http://dev.bibsonomy.org/maven2/org/bibsonomy/bibsonomy-rest-client/}} und die Benutzung der REST-API\footnote{\url{https://bitbucket.org/bibsonomy/bibsonomy/wiki/documentation/api/REST API}} können Sie unter den Links nachlesen.
\newline
\newline
Um auf die API zugreifen zu können benötigen Sie den API-Key. Diesen finden Sie auf der Einstellungsseite unter dem Reiter \enquote{Einstellungen}. 



\subsubsection{OAuth}
OAuth\index{OAuth} ist ein etabliertes Protokoll für sichere API-Autorisierung, die es Nutzern ermöglicht, einer dritten Anwendung den Zugriff auf ihre Daten zu erlauben, ohne, dass sie ihre Anmeldeinformationen außerhalb von PUMA angeben müssen. 
\newline
\newline
\textbf {Wie erhalten Sie durch OAuth in Ihrer Anwendung Zugriff auf PUMA?} \newline
\textbf {1)} Beantragen Sie einen OAuth Consumer Key und Consumer Secret\newline 
Bevor Ihre Anwendung auf die API von PUMA zugreifen kann, müssen Sie für beide Anwendungen einen gesicherten Kommunikationskanal aufbauen. Dies wird durch den Austausch von Anmeldedaten, dem sogenannten Consumer Key und dem Consumer Secret, erreicht. Der Consumer Key identifiziert Ihre Anwendung. Durch den Consumer Secret werden Ihre Anfragen verifiziert. Sowohl symmetrische (HMAC\footnote{\url{https://de.wikipedia.org/wiki/Keyed-Hash_Message_Authentication_Code}}) als auch Public Key (RSA\footnote{\url{https://de.wikipedia.org/wiki/Public-Key-Verschlüsselungsverfahren}})-Verschlüsselung wird unterstützt.
Um einen consumer key und ein consumer secret zu beantragen, schreiben Sie bitte eine E-Mail an api-support@bibsonomy.org. \newline
\newline
\textbf{2)} Implementieren Sie den Autorisierungsprozess von OAuth\newline
Wenn ein Nutzer Ihrer Anwendung in PUMA Zugriff auf seine Daten gewährt, wird der Nutzer zwischen Ihrer Anwendung und PUMA hin- und wieder zurückgelenkt, bis am Ende der sogenannte \textit{access token} Ihre Anwendung erreicht. Dieser wird dann dazu genutzt, um Ihre Anfragen an die API zu autorisieren. Dieser Prozess wird in der OAuth-Anleitung \footnote{\url{https://hueniverse.com/oauth/guide/workflow/}} genauer beschrieben.\newline
Im Wesentlichen muss Ihre Anwendung den Nutzer zu der PUMA-OAuth-Autorisierungsseite weiterleiten, mit den vorher erhaltenen Anmeldedaten als Request-Parameter\newline (z.B. http://www.puma.ub.uni-stuttgart.de/oauth/authorize?oauth\_token=xxxxxxxxxxxx-xxxx-xxxx-xxxx-xxxxxxxxxxxx): %bild
\newline
\newline
Wenn der Nutzer Ihre temporären Anmeldedaten autorisiert, wird er entweder zu Ihrer Seite weitergeleitet (falls Sie eine Callback-URL angegeben haben), oder der Nutzer muss manuell zu Ihrer Seite wechseln. Die autorisierten Anmeldedaten können dann dazu genutzt werden, um den \textit{access token} zu erhalten, mit dem Anfragen autorisiert werden.
%bild
\newline
\newline
Die OAuth-Rest-API von PUMA für Java erleichtert diesen Prozess. Falls Sie Maven nutzen, fügen Sie einfach Ihrer pom.xml-Datei den folgenden Code hinzu:

%<project>\newline
 % <repositories>\newline
    %<repository>\newline
      %<id>bibsonomy-repo</id>\newline
      %<name>Releases von BibSonomy-Modulen</name>
      %<url>http://dev.bibsonomy.org/maven2/</url>
    %</repository>
  %[...]
  %<dependencies>
    %<dependency>
      %<groupId>org.bibsonomy</groupId>
      %<artifactId>bibsonomy-rest-client-oauth</artifactId>
     % <version>2.0.22-SNAPSHOT</version>
    %</dependency>
  %</dependencies>
  %[...]
  
Alternativ können Sie die jar-Dateien auch direkt herunterladen.\newline
Temporäre Anmeldedaten zu erhalten funktioniert dann folgendermaßen:\newline
\newline
\textit{BibSonomyOAuthAccesssor accessor = new BibSonomyOAuthAccesssor("<YOUR CONSUMER KEY>", "<YOUR CONSUMER SECRET>", "<YOUR CALLBACK URL>");\newline
String redirectURL = accessor.getAuthorizationUrl();}
\newline
\newline 
Nun müssen Sie den Nutzer zu redirectURL weiterleiten. Danach werden die vorher erhaltenen temporären Anmeldedaten zu einem access token umgeformt:\newline
\newline
\textit{accessor.obtainAccessToken();}
\newline
\newline
\textbf{3)} Machen Sie Anfragen an die PUMA-API\newline
Jetzt können Sie das PUMA-\textit{rest logic interface} nutzen, um auf die API zuzugreifen.
\newline
\newline
RestLogicFactory rlf = new RestLogicFactory("http://www.puma.ub.uni-stuttgart.de/api", RenderingFormat.XML);\newline
LogicInterface rl = rlf.getLogicAccess(accessor);\newline
[...]\newline
rl.createPosts(uploadPosts);\newline
[...]\newline




\subsection{Programmiersprachen}
\subsubsection{Java\index{Java}}    


\subsubsection{PHP\index{PHP}}
PUMA-API ist ein Paket aus PHP-Skripten, das einen REST-Client enthält, sowie einige Utilities, die hilfreich sind für die Entwicklung einer PHP-Applikation, die mit der PUMA-REST-API interagieren soll. Der REST-Client verwaltet Funktionen, die von der PUMA REST-API angeboten werden.
\newline
Weitere Informationen und Sources finden Sie in dem BitBucket-Repository\footnote{\url{https://bitbucket.org/bibsonomy/restclient-php}}.  

\subsubsection{Python\index{Python}}
Es gibt einen API Client, um mit Hilfe der Programmiersprache Python\footnote{\url{https://www.python.org/}} Einträge aus PUMA abzurufen. Der folgende Codeabschnitt beispielsweise erstellt eine Liste aus Ihren Publikationen:

%bibsonomy = BibSonomy('YOUR_USERNAME', 'YOUR_APIKEY')
%posts = bibsonomy.getPosts('bibtex')
%# do something with the posts...
%for post in posts:
%print post.resource.title %für bibsonomy für PUMA?
 


\subsection{JavaScript- Codeschnipsel}
Durch die Verwendung von JavaScript- Codeschnipseln\index{JavaScript-Code} erleichtern Sie Ihren Webseitenbesuchern das Vermerken und Arbeiten mit PUMA, und Sie vergrößern ihre eigene Reichweite. PUMA macht dies mit ein paar Zeilen JavaScript möglich. Fügen Sie den folgenden Code in Ihre Webseite ein und schon gelangen Besucher mit einem Klick zu PUMA und können dort ganz einfach Lesezeichen und Kommentare hinterlegen.
\newline
\newline
<!-- post bookmark to link code -->\newline
      <script type="text/JavaScript">\newline
      <!--\newline
      var url=encodeURIComponent(document.location.href);\newline
      var title=encodeURIComponent(document.title);\newline
      document.write(\"<a href=https://puma.ub.uni-	stuttgart.de/ShowBookmarkEntry?c=b\&jump=yes\&url="+url+ "\&description="+title +"\" title=\"Bookmark this page to PUMA Stuttgart.\">Bookmark to PUMA Stuttgart!</a>");\newline% " und \ stimmen noch nicht in der PDF muss noch geändert werden 
      //-->\newline
      </script>\newline
      <!-- end post bookmark to link code -->\newline

\subsection{Eigenen Webseiten}
Es gibt mehrere Möglichkeiten, Inhalte aus PUMA oder Links zu PUMA auf Ihrer eigenen Webseite zu integrieren.

\subsubsection{Bookmarklinks\index{Bookmarklinks}}
Einige Webseiten bieten  auf ihrer Seite einen Bookmarklink an, damit der Benutzer ganz einfach Artikel der Seite in sozialen Netzwerken teilen oder in einem Lesezeichensystem speichern kann. 
\newline Auch PUMA verfügt über einen Bookmarklink, den Sie auf Ihrer eigenen Webseite oder Blog hinzufügen können. Fügen Sie dafür Sie einen kurzen JavaScript-Code\index{JavaScript-Code} ein (dieser befindet sich auf der \enquote{Browser Add-ons \& Bookmarklets Seite}\footnote{\url{https://puma.ub.uni-stuttgart.de/buttons}}) und schon können Ihre Besucher zu Ihren Publikationen und Lesezeichen bei PUMA gelangen.

\subsubsection{Publikationslisten\index{Publikationsliste}}
Integrieren Sie Publikationslisten (z.B. Ihre eigenen Publikationen), die das gleiche Format haben wie in PUMA, auf Ihrer Webseite. Dazu müssen Sie ein iframe in Ihren HTML-Code\index{HTML} einfügen, das folgendermaßen aussieht: %Screenshot
\newline Die URL kann jede beliebige Seite aus PUMA sein, z.B. Ihre Benutzerseite oder die Seite einer Ihrer Gruppen. \textbf{Wichtig:} Am Ende der URL muss der Parameter \textit{format=embed} steht. Beispielsweise zeigt die URL \url{https://puma.ub.uni-stuttgart.de/user/droessler/myown?items=1000&resourcetype=publication&sortPage=year&sortPageOrder=desc&format=embed}
alle (bis zu 1000) Publikationen des Nutzers \textit{droessler} an, denen das Schlagwort \enquote{myown} zugeordnet wurde, absteigend nach Jahr sortiert.

\subsubsection{JSON-Feed\index{JSON-Feed}}
Sie können für jede PUMA-Seite einen JSON-Feed\footnote{\url{http://www.json.org/}} generieren, indem Sie \textit{json/} vor den Pfadteil der URL stellen. Um beispielsweise den JSON-Feed für \textit{/tag/json} zu bekommen, geben Sie \textit{/json/tag/json} ein.

Sie erhalten einen JSON-Feed, der mit Exhibit\footnote{\url{http://www.simile-widgets.org/exhibit/}} kompatibel ist und alle Lesezeichen und Publikationen der entsprechenden Seite enthält. Um den JSON-Feed in Ihr Exhibit einzugeben, fügen Sie einen Link dazu in den Header Ihres Exhibit HTML Codes:\newline
\newline
<link href="https://puma.ub.uni-stuttgart.de/tag/json?callback=cb" type=\enquote{application/jsonp} rel=\enquote{exhibit/data} ex:jsonp-callback=\enquote{cb}%muss noch geschaut werden wegen den " 
\newline
%  Ist von kassel :  Schauen Sie sich diese Liste von Publikationen\footnote{\url{http://www.kde.cs.uni-kassel.de/hotho/publication_json.html}} an, um zu sehen, welche Möglichkeiten Sie mit JSON und Exhibit haben.

\subsubsection{Zope\index{Zope}}
Sie können Inhalte aus PUMA dem Content Management System von Zope\footnote{\url{http://www.zope.org/}} hinzufügen.
\begin{enumerate}
    \item Publikationen\newline
    Publikationslisten können mit Hilfe des PUMA-RSS-Feeds\index{RSS} auf Ihrer Zope-Seite dargestellt werden. Eine detaillierte Beschreibung des RDF Summary Produkts\footnote{\url{http://old.zope.org/Members/EIONET/RDFSummary/}} erhalten Sie bei Zope.
    \item Tagwolken\newline
    Sie haben die Möglichkeit Tagwolken auf Ihren Zope-Seiten zu  erstellen. Eine Anleitung zum Vorgang wird im Folgenden erklärt.
\end{enumerate}
\textbf{Tag-Wolken auf Zope-Seiten} \newline
PUMA-Schlagwörter können auf einer Zope-Seite angezeigt werden. 
\begin{enumerate}
    \item Sie müssen auf eine PUMA-Seite aus Zope heraus zugreifen. Hierfür benötigen Sie das Produkt Kebas Data \footnote{\url{https://sourceforge.net/projects/kebasdata/}}.
    \item Für jede Tag-Wolke, die Sie anzeigen lassen wollen, benötigen Sie ein KebasData-Objekt. Bitte konfigurieren Sie es wie folgt (Benutzername etc. muss entsprechend ersetzt werden):%screenshot schon gemacht

    Es werden nun alle Tags in Ihrer Tag-Wolke angezeigt, die sich zwischen den Start- <ul ...> und den Ende- </ul> Schemata bewegen.
    \item Sie müssen jedoch die von PUMA ausgegebenen URLs überarbeiten, da diese sich auf das PUMA-Hauptverzeichnis und nicht auf Ihre Seite beziehen. Hierfür fügen Sie bitte ein \enquote{Script (Python)}-Objekt namens \textit{render\_fixbaseurl} in Zope an beliebiger Stelle oberhalb des Ordners ein, der Ihre Tag-Wolke enthält. Lassen Sie es zwei Parameter haben und folgendermaßen aussehen: 

    ul = context.match[0]\newline
    ul = ul.replace('href="/', 'href="https://puma.ub.uni-stuttgart.de/')\newline
    print ul\newline
    return printed\newline
    some code block

    \item Für die Anzeige Ihrer Tag-Wolke von \textit{DTML} aus müssen Sie diesen Befehl eingeben: 

    <ul class="tagbox">\newline
        <dtml-var tagcloud>\newline
    </ul> \newline

    \item Für die Anzeige Ihrer Tag-Wolke von einem \textit{Page Template} aus können Sie diesen Befehl benutzen: 

    <ul class=\"tagbox\">\newline
        <div tal:replace="structure here/tagcloud"/>\newline
    </ul>\newline

    \item Nutzen Sie CSS\index{CSS} zur Formatierung der Tag-Wolke nach Ihrem Geschmack. Hier sehen Sie, was wir benutzen; bitte beachten Sie, dass dies die selten vorkommenden Tags verbirgt. Sie können \textit{display: none} durch \textit{display: inline} ersetzen, um deren Anzeige zu aktivieren: 

    ul.tagbox \{ list-style: none; text-align: justify; \}\newline
    ul.tagbox li \{ display: inline; \}\newline
    ul.tagbox li a \{ display: none; text-decoration: none; color: \#e05698; font-size: 60\% \} \newline
    ul.tagbox li.tagone a \{  display: none; text-decoration: none; color: \#a3004e; font-size: 80\% \} \newline
    ul.tagbox li.tagten a \{  display: inline; text-decoration: none; color: \#830030; font-size: 100\% \} \newline
\end{enumerate}

\subsection{Plugins} 
\subsubsection{OpenCms\index{OpenCms}}
Mit dem OpenCMS Plugin bei PUMA lassen sich Publiaktionslsiten pflegen und die bibliografischen Daten in anderen Systemen nach nutzen. Ein typischer Anwendungsfall des Plugins ist das Erstellen einer Institutspublikationsliste. Mitarbeiter und/ oder Hilfskräfte des Instituts pflegen die Publiaktionsdaten in PUMA ein. Die eingetragenen Publiaktionen können in einer Institutspublikationsliste ausgegebene und auf der Institutswebsite veröffentlicht werden. Hierbei kann zwischen mehreren Zitatiosstil ausgewählt und nach Datum in absteigender oder aufsteigender Reihenfolge sortiert werden. Das Gleiche können die Mitarbeiter ebenfalls für ihre eigenen Publikationsliste machen.\newline\newline
\textbf{Die Umsetzung bei PUMA:}\newline
\begin{enumerate}
\item  Anmeldung auf \url{https://puma.ub.uni-stuttgart.de.}
\item Legen Sie eine Gruppe an.
\item Mitarbeiter und/ oder Hilfskräfte des Instituts tragen die Publikationen mit ihrem PUMA-Konto ein und taggen diese mit\textit{for:gruppenname}.
\item Platzieren Sie auf einer Freitextseite in OpenCMS den Typ \enquote{Publikationsliste aus BibSonomy/ PUMA} aus dem Typenkatalog.
\item Füllen Sie die Eingabefelder aus.
\end{enumerate}
%\newline\newline
Eine kurze Beschreibung der Eingabefelder:\newline
\small
\begin{longtable}{|c|p{7cm}|}\hline
\bfseries Eingabefeld &\bfseries Beschreibung\\ \hline
Titel& 	Der Titel ist frei wählbar und erscheint auf der Seite als Überschrift, z. B. Meine Publikationen. \\ \hline
API-Benutzer­­name &	Der API-Benutzername entspricht dem von Ihnen angegebenen Benutzernamen bei PUMA. Er beginnt nach dem \@-Zeichen und erscheint nach dem Login oben rechts im Benutzermenü. API steht für \enquote{Application Programming Interface} (Programmierschnittstelle, über die OpenCms und PUMA Daten austauschen).\\ \hline
API-Schlüssel &	Der API-Schlüs­sel ist ein Zahlencode, den Sie aus Ihrem Benutzermenü unter \enquote{Einstellungen} im Reiter \enquote{Einstellungen} finden und kopieren können.\\ \hline
API-Server &	Der API-Server ist bereits voreingestellt auf \url{https://puma.ub.uni-stuttgart.de/api/}\\ \hline
Quelle & Es kann aus drei möglichen Quellen ausgewählt werden: user (Benutzerkonto), group (Gruppenkonto) und viewable (öffentliche Einträge im System).\\ \hline
Source-ID &	Im Feld Source-ID wird der Benutzer oder Gruppenname eingetragen. Importiert werden also die Daten aus dem jeweils angegebenen Benutzerkonto, z. B. aus dem eigenen oder den öffentlich geteilten Einträgen. Bei group sind es dem entsprechend die Daten aus der Gruppe (z. B. schulung).\\ \hline
Tags &	Im Feld Tags (Schlagwörter) werden die Einträge eingegrenzt, die mit diesem Schlagwort vergeben sind. Eine Liste mit den eigenen Publikationensliste wird mit der Quelle user, der Source-ID Benutzername und dem Tag myown erzeugt.\\ \hline
Exclude-Tags& An dieser Stelle werden die Schlagwörter eingegeben, zu denen keine Publikationen angezeigt werden sollen.\\ \hline
Search &	Angabe eines Suchbegriffes: Komplette Volltextsuche, auch in hoch geladenen Volltexten. Ausgabe der Publikationseinträge, in denen Suchergebnisse vorkommen.\\ \hline
Anzahl der Pub­li­ka­tionen &	Es können bis zu 1.000 Einträge angezeigt werden. Voreingestellt sind 100 Einträge.\\ \hline
Sortierfeld &	Als Sortierfeld kann none (keine Sortierung), author (der Autor), entrytype (der Publikationstyp), title (der Titel) oder year (das Jahr) gewählt werden.\\ \hline
Sortierreihenfolge &	Definiert die auf (ascending)- oder absteigende (descending) Sortierung.\\ \hline
Gruppierung &	Gruppierte Ausgabe mit Überschriften. Bei author wird nach dem Alphabet (A,B,C, usw.) gegliedert. Bei entrytype wird nach dem Publikationstyp (article, book, conference, usw.) sortiert. Mit title werden die Publikationen nach dem Anfangsbuchstaben ihres Titels angezeigt, der erste Buchstabe erscheint als Abschnittsüberschrift. Bei year werden die Jahreszahlen als Überschriften angezeigt.\\ \hline
Duplikatunterdrückung &	Wenn PUMA einen doppelten Publikationseintrag erkennt, wird nur einer angezeigt.\\ \hline
CSL-Stil &	Im Feld CSL-Stil (Citation Style Language) wird der Zitationsstil ausgewählt. mit dem die Publikationen angezeigt werden sollen. Falls weitere, nicht in der Liste aufgeführte Zitationsstile benötigt werden, können über die CSL-Templates bis zu 7.500 Zitationsstile ausgeben werden.\\ \hline
Zeige Zusammenfassung& 	Das Abstract, falls vorhanden, wird als Link mit ausgegeben.\\ \hline
Zeige BibTex-Code &	Der BibTex-Code kann als Link mit angezeigt werden.
Zeige Link 	Der Link zum Volltext wird, falls vorhanden, angezeigt.\\ \hline
\end{longtable}
normalsize
Der Inhalt der Eingabefelder für eine Mitarbeiterpublikationsliste und für eine Institutspublikationsliste unterschieden sich. Bei dem folgenden Beispiel beziehen beide Listen die Publikationen aus der Institutsgruppe in PUMA.\newline\newline
\textbf{Muster-Abfrage der Institutspublikationsliste:}\newline %Screenshots
\textbf {Muster-Abfrage der Mitarbeiterpublikationsliste:}\newline%Screenshot
\newline
Wenn die gewünschten Einstellungen veröffentlicht werden, importiert das Plugin die Literatureinträge im ausgewählten Zitationsstil auf der entsprechenden OpenCms-Seite. Veränderungen, die bei PUMA vorgenommen werden, Korrekturen oder weitere Einträge, werden automatisch auf der OpenCms-Seite aktualisiert. OpenCms-Caching-Einstellungen wirken zusätzlich, Änderungen im PUMA-System werden zeitverzögert im OpenCms sichtbar.
\subsubsection{Typo3\index{Typo3}}
\textbf{Installation}\newline\newline
Um PUMA CSL zu installieren, melden Sie sich bei ihrer TYPO3-Instanz als Administrator an. Gehen Sie im Extension Manager zu den Extensions Import results. Suchen Sie nach der Extension ext\_bibsonomy\_csl und importieren Sie diese.
\begin{figure}[ht]
 \centering
 \includegraphics[scale=0.4]{puma-97}
 \caption{Extension Manager}
 \label{figure1}
\end{figure}
Nach erfolgreichem Import wird die Extension unter \enquote{Available Extensions} angezeigt. Klicken Sie auf das +-Symbol, um es zu aktivieren.
\begin{figure}[ht]
 \centering
 \includegraphics[scale=0.4]{puma-98}
 \caption{Available Extensions}
 \label{figure1}
\end{figure}
\newline\newline
\textbf{Publikationslisten hinzufügen mit dem Frontend Plugin}\newline\newline
Nachdem Sie die Extension importiert und aktiviert haben, können Sie Publikationslisten erstellen. Fügen Sie dazu der Seite, auf der die Publikationsliste erscheinen soll, ein neues Plugin hinzu. Wählen Sie hierfür aus der Liste \enquote{PUMA Publication List} aus.\newline
\newline
Im Reiter \enquote{General} tragen Sie den Titel für die Publikationsliste ein.
\begin{figure}[ht]
 \centering
 \includegraphics[scale=0.4]{puma-99}
 \caption{Reiter General}
 \label{figure1}
\end{figure}
\newline \newline
Gehen Sie auf den Reiter \enquote{Plugin} und nehmen Sie die gewünschten Einstellungen vor. Sie können zwischen \textit{user}, \textit{group} oder \textit{viewable} wählen, um den entsprechenden Inhalt aus PUMA zu definieren, den Sie in der Publikationsliste haben möchten.\newline
\newline
Beispiel: Sie möchten Ihre eigene Publikationsliste einbinden. Dazu wählen Sie zunächst unter der Rubrik \enquote{Content Source} die Option \enquote{user} aus. Tragen Sie dann den gewünschten (Ihren) Nutzernamen unter \enquote{Insert the id of user, group or viewable source} ein. Anschließend können Sie die Einträge der Publikationsliste über Tags filtern. geben Sie hierfür die gewünschten Tags in das Feld \enquote{Select content via tags} ein. Um nur eigene Einträge anzeigen zu lassen, verwenden Sie den Systemtag \textit{myown}. Sie können zudem die Anzahl der angezeigten Publikationen begrenzen sowie mittels Freitext filtern.
\begin{figure}[ht]
 \centering
 \includegraphics[scale=0.4]{puma-100}
 \caption{Reiter Plugin}
 \label{figure1}
\end{figure}\newline
\begin{shaded} \centering
\textbf{ACHTUNG:} Bitte beachten Sie, dass die Extension Ihre gesamte Publikationsliste darstellt, wie Sie sie in PUMA sehen \textbf{(auch private Publikationen)}. Wenn Sie z.B. an eine Publikation ein Dokument angehängt haben, dann wird dieses auch in der Publikationsliste angezeigt. Dadurch können für Sie \textbf{urheberrechtliche Konsequenzen} entstehen. Wir empfehlen Ihnen daher für die Nutzung einen zusätzlichen Account anzulegen, mit dem Sie in PUMA nur die TYPO3-Publikationen verwalten.
\end{shaded} 
In der Rubrik \enquote{Layout} können Sie die Gestaltung der Publikationsliste anpassen. Dies geschieht durch Citation Style Language (CSL\index{CSL}), das ist eine frei verfügbare XML-basierte Auszeichnungssprache. Eine große Liste an frei verfügbaren CSL-Vorlagen finden Sie hier: \url{http://www.zotero.org/styles/}.\begin{figure}[ht]
 \centering
 \includegraphics[scale=0.4]{puma-101}
 \caption{Citation Style Language(CSL)}
 \label{figure1}
\end{figure}
\newline
\newline
In der letzten Rubrik \enquote{Login} müssen Sie Ihre PUMA API-Daten\index{API} hinterlegen. Nur so kann sich die TYPO3 Extension bei PUMA anmelden und Ihre Daten abrufen. Ihre API-Daten finden Sie, wenn Sie über das Personensymbol in die Einstellungen gehen und im Reiter \enquote{Einstellungen} nach unten scrollen. \newline
\newline
\textbf{Tag-Wolken hinzufügen mit dem Frontend Plugin}\newline\newline
Sie können Ihren Webseiten nicht nur Publikationslisten hinzufügen, sondern auch Ihre PUMA  Tag-Wolke(Tagcloud). Fügen Sie dazu der Seite, auf der die Tag-Wolke\index{Tag!Wolke} erscheinen soll, ein neues Plugin hinzu. Wählen Sie aus der Liste \enquote{Bibsonomy Tag Cloud}.\newline \newline 
Wie bei \enquote{Publikationsliste hinzufügen} können Sie auch hier zwischen verschiedenen Modi wählen.\newline \newline 
\textbf{CSL Styles mit dem Backend Module verwalten}\newline
\newline
TYPO3-Extensions werden klassisch in zwei Module unterteilt, dem Frontend- und Backend-Module. Die Frontend-Module \enquote{Publikationsliste hinzufügen} und \enquote{Tag-Wolke hinzufügen} wurden bereits erklärt. In diesem Abschnitt soll es um das Backend-Modul gehen, mit dem Sie die CSL-Stylesheets verändern können.\newline \newline
Bereits mit der Extension-Installation werden Ihnen eine Reihe von CSL-Stylesheets ausgeliefert. Um weitere Stylesheets hinzuzufügen, erstellen Sie einen neuen Ordner im Seitenbaum und nennen diesen \textit{CSL Styles}. Wählen Sie anschließend diesen Ordner aus. Klicken Sie auf \enquote{Neu} und wählen \enquote{Add a custom style}.\newline \newline
Um ein neue Stylesheets hinzuzufügen gibt es drei unterschiedliche Möglichkeiten:
\begin{enumerate}
\item Geben Sie den XML-Quellcode direkt in das Textfeld ein und klicken Sie anschließend auf \enquote{Save}.
\item Geben Sie in das Textfeld die URL des CSL-Stylesheets ein und klicken Sie auf \enquote{Import}.
\item Laden Sie ein CSL-Stylesheet von Ihrem Computer hoch und klicken Sie auf anschließend auf \enquote{Upload}. 
\end{enumerate}
\begin{figure}[ht]
 \centering
 \includegraphics[scale=0.3]{puma-102}
 \caption{Neue Stylesheets hinzuzufügen}
 \label{figure1}
\end{figure}
Um eine Vorschau zu erhalten, klicken Sie auf \enquote{Show Styles}.\newline
Um eine CSL-Stylesheet zu löschen klicken Sie wie gewohnt auf das Mülleimersymbol in TYPO3. 
\begin{figure}[ht]
 \centering
 \includegraphics[scale=0.4]{puma-103}
 \caption{CSL-Stylesheet löschen}
 \label{figure1}
\end{figure}




\subsubsection{Wordpress\index{WordPress}}
Das Wordpress-Plugin ermöglicht Ihnen verschiedene PUMA-Funktionen für Ihren eigene WordPress-Blog zu nutzen.\newline\newline
\textbf{Die Funktionen:}
\begin{enumerate}
   \item Fügen Sie Publikationslisten in einen Artikel ein, indem Sie die Meta-Box-Integration nutzen
    \item Sie können alternativ die Wordpress-Shortcodes nutzen
    \item Wählen Sie Ihre Publikationen aus verschiedenen Inhaltstypen aus (z.B. user/group/viewable)
    \item Filtern Sie die Ergebnisse mit Tags oder einer Volltextsuche
    \item Wählen Sie Ihren bevorzugten Zitierstil (Citation Style Language (CSL))\footnote{\url{ http://citationstyles.org/}}  aus einer Liste aus vorinstallierten Stilen aus
    \item Sie können Ihren eigenen Zitierstil mit Hilfe der CSL\index{CSL} nutzen oder erstellen, um Ihre Literaturliste anzeigen zu lassen
    \item Sie können das Layout Ihrer Liste mit CSS bearbeiten
    \item Speichern Sie Ihre API-Einstellungen (z.B. API-Nutzer, API-Key) auf einer separaten Optionsseite für Administratoren
    \item Fügen Sie Ihrem Blog Tagwolken aus PUMA hinzu und wählen Sie zwischen drei verschiedenen Layouts aus
    \item Stellen Sie Dokumente, die an Publikationen angehängt worden sind, zum Download bereit
\end{enumerate} 
Um Publikationslisten basierend auf der Citation Style Language zu erstellen wird das WordPress-Plugin benötigt. Unter folgender Website kann das Plugin installiert werden: \url{https://wordpress.org/plugins/bibsonomy-csl/}. Eine Anleitung zur Installation finden Sie im BibSonomy Blog \footnote{\url{http://blog.bibsonomy.org/2012/12/feature-of-week-add-publication-lists.html}}
\subsubsection{Ilias}
\subsubsection{Moodle}
PUMA/BibSonomy Module (PBM) ist das PUMA/BibSonomy Plugin für Moodle. Es ermöglicht das Veröffentlichen von Publikationslisten aus PUMA heraus in einen Moodlekurs. Der Zitationsstil, in welchem die Publikationen in Moodle angezeigt werden sollen, kann mit Hilfe des CSL\footnote{\url{http://citationstyles.org/}} festgelegt werden.\newline
\textbf{ Der Administrator}\newline
Die folgenden Schritte sind wichtig für die Installation des PBM Plugins: 
%\begin{enumerate}
%	\item Laden Sie sich das Plugin als \*.zip Datei herunter. Wählen Sie auf der Seite von Moodle (\url{https://moodle.org/plugins/view.php?plugin=mod_pbm ) den Reiter \enquote{Version} aus und klicken auf Download.
	%\item Entpacken Sie die \*.zip Datei im  richtigen Ordner (z.B. /var/www/moodle/mod/pbm)
	%\item Gehen Sie auf die Moodle Webseite und loggen Sie sich als Administrator ein. Gehen Sie unter Einstellungen auf \enquote{Website-Administrationzu} wählen Sie \enquote{Plugins} aus. Klicken Sie anschließend auf\enquote{Plugin-Übersicht}.
	%\item Klicken Sie oben auf der Überblickseite der Plugins auf den Button %"Check for available updates".
	%\item Moodle informiert Sie darüber, dass das database Update durchgeführt wurde. Bestätigen Sie dies.     
	%\item Sie werden nach einer voreingestellten Serveradresse gefragt, nach Ihren OAuth Consumer- Daten und Ihrer Wahl des Zitationsstils.      
%\end{enumerate}      
 
\section{PUMA als Forschungsprojekt}
\textit{PUMA wächst und wächst. Wie die Geschichte von PUMA begann und was die Zukunft bringt.}
\newline
\newline
Die Web-Anwendung PUMA ist ein Schwester-System von BibSonomy.  PUMA wurde durch die Deutsche Forschungsgemeinschaft (DFG) von Anfang August 2009 bis Juli 2011 gefördert. Die Universitätsbibliothek Kassel und das Fachgebiet Wissensverarbeitung der Universität Kassel waren an der PUMA-Entwicklung beteiligt. Nach der ersten Projektphase wurden das System als Open-Source-Software veröffentlicht. PUMA profitiert von der Weiterentwicklung seines Schwester-Systems BibSonomy\index{BibSonomy} und wird immer wieder aktualisiert. In einer zweiten Förderphase der DFG von 2013 bis 2015 trat die DMIR-Gruppe (Data Mining \& Information Retrieval Group) von der Universität Würzburg dem PUMA-Team bei. In ihrem Rahmen wurde die Zusammenarbeit mit Fremdsystemen ausgebaut, wie zum Beispiel die Integration von PUMA im Discovery Service des HeBIS-Verbundes, dem die Open-Source-Software Vufind zugrunde liegt. Ergebnis ist ein Modul, das die Merklistenfunktion dahingehend erweitert, dass die Einträge automatisch in PUMA gespeichert und die Inhalte der Merkliste aus PUMA geholt werden. Bis zum derzeitigen Stand gibt es PUMA an insgesamt ... Universitätsbibliotheken in Deutschland.
\section{Glossar}
%\newglossaryentry{pdf}{name={ac-Konto}, description={Mitarbeiterkonto}}
%\glsaddall \printglossary 
\begin{longtable}{c p{8cm}}
 & \makecell*[tl]{}\\
APA&\makecell*[tl]{Zitationsstil der Amerikan Psychological Association}\\ 
API&  \makecell*[tl]{ Application programming interface (dt.: Programmierschnittstelle)}\\
API-Key&\makecell*[tl]{Code für die Programmierschnittstelle}\\
Citavi & \makecell*[tl]{Literaturverwaltungsprogramm}\\
CSL&\makecell*[tl]{Citation Style Language}\\
CSS&\makecell*[tl]{Cascading Style Sheets; Bildet zusammen mit HTML eine der Kernsprachen des World Wide Webs}\\
DFG&\makecell*[tl]{Deutsche Forschungsgemeinschaft}\\
dblp&\makecell*[tl]{Digital Bibliography and Libary Project, ist eione online verfügbare bibliographische Sammlung wissenschaftlicher Publikationen im Bereich Informatik}\\
DOI&  \makecell*[tl]{Digital Object Identifier}\\
Dropdownmenü&\makecell*[tl]{=Untermenü}\\
fn-Konto&\makecell*[tl]{Funktionskonto bei der Universitätsbibliothek Stuttgart}\\
HTML  & \makecell*[tl]{Hypertext Markup Language (Programmiersprache für Webseiten)}\\
ISBN & \makecell*[tl]{Internationsale Standardnummer}\\
ISSN & \makecell*[tl]{Internationale Standardnummer für fortlaufende Sammlerwerke}\\
ist &  \makecell*[tl]{Institut für Systemtheorie und Reglungstechnik an der Universität Stuttgart}\\ 
JabRef & \makecell*[tl]{Literaturverwaltungsprogramm}\\
JSON-Feed&\makecell*[tl]{}\\
OAuth & \makecell*[tl]{Ist ein offenes Protokoll, das eine standardisierte sichere API-Authentifizierung für Desktop- und Web- Anwendungen gestattet}\\
OenAccess  &\makecell*[tl]{Freier Zugang zu wissenschaftlicher Literatur und anderen Materialien im Internet}\\
OpenCMS &  \makecell*[tl]{Ist ein Content-management-System für die Gestaltung von Webseiten}\\
Open-URL & \makecell*[tl]{Ist ein Standard zur Angabe von Metadaten in einer URL, um unabhängig vom aktuellen Standort elektronische Dokumente zu verlinken}\\
OPUS&\makecell*[tl]{Dokumentenserver der Universität Stuttgart}\\
PBM&\makecell*[tl]{PUMA/ BibSonomy Module}\\
PHP&\makecell*[tl]{Personal Home Page Tools, heute: Hypertext Preprocessor (Skriptsprache)}\\
PUMA & \makecell*[tl]{Akademisches Publikationsmanagement}\\
Rest-API&\makecell*[tl]{Representational State Transfer}\\
RFC&   \makecell*[tl]{Request of Comments (dt.: Die Bitte um Kommentare)}\\
RSS&\makecell*[tl]{Really Simple Syndication, dt.: Dateiformate, die Veränderungen auf Webseiten zeigen}\\
st-Konto&  \makecell*[tl]{Studentenkonto bei der Universitätsbibliothek Stuttgart}\\
SWORD&\makecell*[tl]{}\\
SWRC&\makecell*[tl]{Semantic Web of Research Communities}\\
Tags&\makecell*[tl]{=Schlagwörter}\\
UB&\makecell*[tl]{Universitätsbibliothek}\\
URL &\makecell*[tl]{Uniform Resource Locator (dt.: Einheitlicher Ressourcenzeiger)}\\
vgl.&\makecell*[tl]{vergleiche}\\
XML&\makecell*[tl]{Extensible Markup Language; Erweiterbare Auszeichnungssprache}\\
z.B.&\makecell*[tl]{zum Beispiel}\\
Zope&\makecell*[tl]{Content-Management-System zur Gestaltung von Webseiten}\\
Zotero &\makecell*[tl]{ Literaturverwaltungsprogramm}\\
\end{longtable}




\renewcommand{\indexname}{Stichwortverzeichnis}
\addcontentsline{toc}{section}{Stichwortverzeichnis}
\printindex

\end{document}
