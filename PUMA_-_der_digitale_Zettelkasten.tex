\chapter{PUMA - der digitale Zettelkasten}
\textit{Publikationen und Lesezeichen sammeln, verwalten und teilen, mit PUMA ein Kinderspiel.}\newline
\newline
Das Akademische Publiaktionsmanagement\index{Akademische Publikationsmanagement} (PUMA) ist zu vergleichen mit einem riesigen digitalen Zettelkasten, der für alle möglichen Quellen und Medien einsetzbar ist. Es ermöglicht Struktur und Ordnung für gesammelte Publikationen. Gespeicherte Publikationen und Lesezeichen lassen sich schnell wieder finden. Gleichzeitig bietet PUMA Platz für Notizen und Anmerkungen sowie eine Zusammenarbeit mit anderen PUMA-Nutzern. 
Die Software steht lizenzfrei als Webanwendung zur Verfügung.\newline
PUMA ist ein System zum Sammeln, Verwalten, Teilen und Entdecken von Lesezeichen und Publikationen. \newline
\newline 
PUMA ist so konzipiert, dass es als alleiniges Eingabeportal für bibliografische Metadaten dienen kann. Außerdem können zu Literatureinträgen Dokumente hochgeladen werden. \newline
Durch die Vielzahl an Exportformaten und Schnittstellen zu anderen Programmen müssen die Nutzer ihre Daten nur einmal pflegen und können sie in anderen Systemen nachnutzen. Die wiederholte manuelle Eingabe von Publikationslisten entfällt. So können Forscher ihre Publikationslisten direkt aus PUMA auf ihre Homepage laden.  
\section{Für wen ist dieses Buch?} 
Dieses Buch richtet sich an die Angehörigen der Universität Stuttgart. Die im Buch erklärten Beispiele basieren auf der für die Universität  an geplanten PUMA-Installation. Diese Beispiele können in leicht abgewandelter Version von jeder Institution, die PUMA installiert hat, verwendet werden. \newline
Externe, die nicht der Universität Stuttgart angehören, können sich nicht bei dem hier vorgestellten PUMA authentifizieren. Für sie bietet sich die Nutzung des Muttersystems von PUMA, BibSonomy, an. Da PUMA und BibSonomy über fast die gleichen Funktionen und Möglichkeiten verfügen, lassen sich die Beispiele aus dem Buch mit leichten Abweichungen, auch auf BibSonomy übertragen.\newline
Besonders geeignet ist PUMA für
\begin{itemize}
\item Forscher, die ihre eigenen Publikationslisten verwalten.
\item Mitarbeiter, die Publikationslisten von Projekten, Instituten oder Fakultäten pflegen.
\item Studenten, die Material für Examensarbeiten verwalten möchten.
\item Autoren, die ihre Veröffentlichungen der Unibibliografie melden möchten.
\item Studenten und Wissenschaftler, die in Arbeitsgruppen Literatur teilen möchten.
\end{itemize}
\section{Typische Anwendungsbeispiele für PUMA}
PUMA ist für die Nutzung im akademischen Bereich entwickelt worden.
Es hilft bei Literaturrecherchen für eine Haus-, Bachelor- oder Masterarbeit, indem die recherchierte Literatur in PUMA gespeichert werden kann. Um Zeit zu sparen, können Webseiten und Publikationen mittels einer Schaltfläche (Bookmarklet) in dem eigenen Browser direkt in PUMA abspeichert werden. Am Ende der Hausarbeit hilft PUMA dabei das Literaturverzeichnis zu erstellen. Bei der Erstellung des Verzeichnisses kann aus 7.500 Zitationsstilen der passende ausgewählt werden oder auch eine individuelle Anpassung per Citation Style Language (CSL) vorgenommen werden.
\newline 
Eigene Veröffentlichungen können mit Hilfe von PUMA gepflegt und  durch den Tag \enquote{myown} gekennzeichnet werden. Dies vereinfacht das Erstellen einer Publikationsliste der eigenen Veröffentlichungen um ein Vielfaches. Mit Hilfe des OpenCMS-Plugins kann die Publikationsliste direkt auf der eigenen Homepage veröffentlicht werden und ist so für Studenten, aber auch Externe sichtbar.
\newline 
Ein weiteres typisches Anwendungsfeld bilden Institutspublikationslisten. Die Wissenschaftler eines Institutes können durch das Erstellen einer neuen Gruppe in PUMA eine gemeinsame Sammlung an Publikationen anlegen. Mit Hilfe des OpenCMS-Plugins von PUMA kann das Institut seine Institutspublikationslisten aus dieser Sammlung erzeugen. Diese kann auf der Institutshomepage veröffentlichen und den Studenten zur Verfügung gestellt werden.

   
\section{Anmelden\index{Anmeldung} bei PUMA} 
\begin{figure}[h!]
 \centering
 \fbox{\includegraphics[width=10cm]{Bilder/Kapitel1/Startseite_Puma}}
 \caption{Startseite PUMA}
 \label{figure001}
\end{figure}
\textbf{Vorab:} Sie benötigen ein st-, fn- oder ac-Konto der Universität Stuttgart.
\begin{enumerate}
    \item Rufen Sie die Anmeldeseite von PUMA auf:\newline \url{https://puma.ub.uni-stuttgart.de/}
    \item Geben Sie unter \enquote{Benutzername} Ihr ac- oder st-Konto der Universität Stuttgart ein (seltener ist das fn-Konto). 
    \item Unter  \enquote{Login} geben Sie Ihr Passwort ein. 
 \begin{figure}[h!]
 \centering
 \fbox{\includegraphics[width=9cm]{Bilder/Kapitel1/Anmeldung_bei_Puma}}
 \caption{Anmeldung bei PUMA}
 \label{figure002}
\end{figure}  
    \item Klicken Sie auf \enquote{Anmelden}.
    \item Wenn Ihre Anmeldung erfolgreich war, zeigt Ihnen PUMA Ihre persönlichen Daten an. Überprüfen Sie die vorliegenden Daten, vergeben Sie einen Benutzernamen und klicken Sie anschließend auf \enquote{Registrieren}.
\end{enumerate}
Die Registrierung bei PUMA war erfolgreich. Ab sofort können Sie sich bei PUMA mit Ihrem Benutzerkonto anmelden. \newline
Bei der erstmaligen Anmeldung ist die Vergabe eines Benutzernamens erforderlich. Nur bei öffentlich geteilten Einträgen erscheint er bei den Publiaktionseinträgen in der Form @benutzername.
\section{Der PUMA-Blog}
Im Blog der Universitätsbibliothek Stuttgart (UB) ist neben allgemeinen Informationen zu der UB auch die Kategorie \enquote{PUMA} an wählbar (\url{http://blog.ub.uni-stuttgart.de/category/puma/}). Hier können die Nutzer sich über die aktuellen PUMA-Ereignisse einen Überblick verschaffen und werden über PUMA-Updates und neue Funktionen informiert.\newline
Mit Hilfe eines RSS-Feeds können Interessierte Informationen zu Server-Updates abonnieren. Über den Link: \url{http://blog.ub.uni-stuttgart.de/category/puma/feed/} werden die aktuellen Informationen angezeigt. Durch das Klicken auf \enquote{Jetzt abonnieren} öffnet sich ein Fenster, indem das Abonnieren nochmals bestätigt werden muss. Ab sofort können die Informationen über die Lesezeichen-Leiste angezeigt werden. 
\section{BibSonomy\index{BibSonomy}}
PUMA steht nur den Mitgliedern der Universität Stuttgart zu Verfügung, die über ein st-, fn- oder ac-Konto  verfügen. Für externe Nutzer der Universitätsbibliothek Stuttgart besteht die Möglichkeit das Muttersystem von PUMA, BibSonomy, zu nutzen. Beide Systeme verfügen über fast die gleichen Funktionen und Möglichkeiten seine Publikationen und Lesezeichen zu sammeln, verwalten und teilen. \newline
Die Anmeldung bei BibSonomy erfolgt über die Homepage \newline
\url{http://www.bibsonomy.org/?lang=de}.  

%\section{BibSonomy\index{BibSonomy} vs. PUMA}
%\suppressfloats[t]
\begin{table}[h!]
\tabulinesep=1.5mm
\begin{tabu}{|X[1.4,c]|X[2.2,m]|X[2,m]|} 
\tabucline[0.5pt]-\everyrow{\tabucline[0.5pt]-} 
\rowfont\bfseries
Unterschiede & PUMA \emph{Uni Stuttgart} & BibSonomy\\ \tabucline[1pt]-
\bfseries{Anmeldung}\strut & Nur möglich mit einem st-, fn- oder ac-Konto der Universität, mit dem sich die Nutzer authentifizieren.  & Für jeden frei zugänglich, ein Benutzerkonto muss selber angelegt werden. \\ 
\bfseries{Gruppen}\index{Gruppen} & Gruppen können jederzeit und selbständig gegründet werden. & Die Gründung einer Gruppe erfordert die Freigabe des BibSonomy-Admins. \\
\bfseries{OPUS}\index{OPUS} & Für die Zukunft geplant. Ermöglicht den Nutzern ein direktes Veröffentlichen auf dem Dokumentenserver OPUS. & \\ 
\bfseries{Unibibliografie}\index{Unibibliografie}& Die Publikationsmetadaten der Unibibliografie stehen zum Beispiel für Institutspublikationslisten zur Nachnutzung zur Verfügung.&\everyrow{} \\ \tabucline[1.0pt]-
\end{tabu}
\caption{Unterschiede zwischen PUMA und BibSonomy}
\end{table}
%\normalsize