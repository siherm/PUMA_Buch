\section{Glossar}
\label{sec:glossar}
%\newglossaryentry{pdf}{name={ac-Konto}, description={Mitarbeiterkonto}}
%\glsaddall \printglossary 
%\begin{longtable}{c p{5cm}}
\renewcommand*{\arraystretch}{1.5}
\begin{longtable}{l p{8cm}}
%  & \makecell*[tl]{}\\
% APA&\makecell*[tl]{Zitationsstil der Amerikan Psychological Association}\\ 
% API&  \makecell*[tl]{Application programming interface (dt.: Programmierschnittstelle), die anderen Programmen den Zugriff auf die Funktionalitäten oder Daten von PUMA ermöglicht}\\
% API-Key&\makecell*[tl]{Text, der einen Benutzer/ein Programm gegenüber der Programmierschnittstelle identifiziert}\\
% Citavi & \makecell*[tl]{Literaturverwaltungsprogramm}\\
% CSL&\makecell*[tl]{Citation Style Language zum Definieren von Zitationsstilen}\\
% CSS&\makecell*[tl]{Cascading Style Sheets; Sprache zur Beschreibung des Aussehens von Elementen}\\
% DFG&\makecell*[tl]{Deutsche Forschungsgemeinschaft}\\
% dblp&\makecell*[tl]{Digital Bibliography and Libary Project, online verfügbare bibliographische Sammlung wissenschaftlicher Publikationen im Bereich Informatik}\\
% DOI&  \makecell*[tl]{Digital Object Identifier, ermöglicht den Zugriff auf digitale Objekte, auch wenn diese ihren realen Ort ändern}\\
% Dropdownmenü&\makecell*[tl]{Aufklappmenü}\\
% fn-Konto&\makecell*[tl]{Funktionskonto bei der Universitätsbibliothek Stuttgart}\\
% HTML  & \makecell*[tl]{Hypertext Markup Language (Auszeichnungssprache für Webseiten)}\\
% ISBN & \makecell*[tl]{Internationale Standardnummer für Bücher}\\
% ISSN & \makecell*[tl]{Internationale Standardnummer für Zeitschriften}\\
% IST &  \makecell*[tl]{Institut für Systemtheorie und Reglungstechnik an der Universität Stuttgart}\\ 
% JabRef & \makecell*[tl]{Literaturverwaltungsprogramm}\\
% JSON-Feed&\makecell*[tl]{Datenstrom im JSON-Format}\\
% OAuth & \makecell*[tl]{Offenes Protokoll, das eine standardisierte sichere API-Authentifizierung für Desktop- und Web- Anwendungen gestattet}\\
% OpenAccess  &\makecell*[tl]{Freier Zugang zu wissenschaftlicher Literatur und anderen Materialien im Internet}\\
% OpenCMS &  \makecell*[tl]{Content-Management-System für die Gestaltung der Webseiten an der Universität Stuttgart}\\
% Open-URL & \makecell*[tl]{Standard zur Angabe von Metadaten in einer URL, um unabhängig vom aktuellen Standort elektronische Dokumente zu verlinken}\\
% OPUS&\makecell*[tl]{Dokumentenserver der Universität Stuttgart}\\
% PBM&\makecell*[tl]{PUMA/BibSonomy Module}\\
% PHP&\makecell*[tl]{Personal Home Page Tools, heute: Hypertext Preprocessor (Skriptsprache, gebräuchlich in der Webprogrammierung)}\\
% PUMA & \makecell*[tl]{Akademisches Publikationsmanagement-Tool}\\
% Rest-API&\makecell*[tl]{Representational State Transfer: Einheitliche Schnittstelle auf Basis des HTTP-Protokolls, das anderen Programmen die Möglichkeit gibt, auf die Daten und Funktionalitäten von PUMA zuzugreifen}\\
% RFC&   \makecell*[tl]{Request for Comments (dt.: Bitte um Kommentare): Vorschlag für einen Standard}\\
% RSS&\makecell*[tl]{Really Simple Syndication, dt.: Protokoll, das Veränderungen auf Webseiten zurückliefert}\\
% st-Konto&  \makecell*[tl]{Studentenkonto bei der Universitätsbibliothek Stuttgart}\\
% SWORD&\makecell*[tl]{Simple Web-service Offering Repository Deposit, Standard für den Datenaustausch zwischen Repositories}\\
% SWRC&\makecell*[tl]{Semantic Web of Research Communities}\\
% Tags&\makecell*[tl]{=Schlagwörter}\\
% UB&\makecell*[tl]{Universitätsbibliothek}\\
% URL &\makecell*[tl]{Uniform Resource Locator (dt.: Einheitlicher Ressourcenzeiger) bezeichnet den Standort einer digitalen Ressource, z.B. einer Webseite}\\
% vgl.&\makecell*[tl]{vergleiche}\\
% XML&\makecell*[tl]{Extensible Markup Language; Auszeichnungssprache, um Informationen zu strukturieren}\\
% z.B.&\makecell*[tl]{zum Beispiel}\\
% Zope&\makecell*[tl]{Content-Management-System zur Gestaltung von Webseiten}\\
% Zotero &\makecell*[tl]{ Literaturverwaltungsprogramm}\\
 %& }\\
APA&Zitationsstil der Amerikan Psychological Association\\ 
API&  Application programming interface (dt.: Programmierschnittstelle), die anderen Programmen den Zugriff auf die Funktionalitäten oder Daten von PUMA ermöglicht\\
API-Key&Text, der einen Benutzer/ein Programm gegenüber der Programmierschnittstelle identifiziert\\
Citavi & Literaturverwaltungsprogramm\\
CSL&Citation Style Language zum Definieren von Zitationsstilen\\
CSS&Cascading Style Sheets; Sprache zur Beschreibung des Aussehens von Elementen\\
delicious&sozialer Lesezeichendienst\\
DFG&Deutsche Forschungsgemeinschaft\\
dblp&Digital Bibliography and Libary Project, online verfügbare bibliographische Sammlung wissenschaftlicher Publikationen im Bereich Informatik\\
DOI&  Digital Object Identifier, ermöglicht den Zugriff auf digitale Objekte, auch wenn diese ihren realen Ort ändern\\
Dropdownmenü&Aufklappmenü\\
fn-Konto&Funktionskonto bei der Universitätsbibliothek Stuttgart\\
HTML  & Hypertext Markup Language (Auszeichnungssprache für Webseiten)\\
ISBN & Internationale Standardnummer für Bücher\\
ISSN & Internationale Standardnummer für Zeitschriften\\
IST &  Institut für Systemtheorie und Reglungstechnik an der Universität Stuttgart\\ 
JabRef & Literaturverwaltungsprogramm\\
JSON-Feed&Datenstrom im JSON-Format\\
OAuth & Offenes Protokoll, das eine standardisierte sichere API-Authentifizierung für Desktop- und Web- Anwendungen gestattet\\
OpenAccess  &Freier Zugang zu wissenschaftlicher Literatur und anderen Materialien im Internet\\
OpenCMS &  Content-Management-System für die Gestaltung der Webseiten an der Universität Stuttgart\\
Open-URL & Standard zur Angabe von Metadaten in einer URL, um unabhängig vom aktuellen Standort elektronische Dokumente zu verlinken\\
OPUS&Dokumentenserver der Universität Stuttgart\\
PBM&PUMA/BibSonomy Module\\
PHP&Personal Home Page Tools, heute: Hypertext Preprocessor (Skriptsprache, gebräuchlich in der Webprogrammierung)\\
PUMA & Akademisches Publikationsmanagement-Tool\\
Rest-API&Representational State Transfer: Einheitliche Schnittstelle auf Basis des HTTP-Protokolls, das anderen Programmen die Möglichkeit gibt, auf die Daten und Funktionalitäten von PUMA zuzugreifen\\
RFC&   Request for Comments (dt.: Bitte um Kommentare): Vorschlag für einen Standard\\
RSS&Really Simple Syndication, dt.: Protokoll, das Veränderungen auf Webseiten zurückliefert\\
st-Konto&  Studentenkonto bei der Universitätsbibliothek Stuttgart\\
SWORD&Simple Web-service Offering Repository Deposit, Standard für den Datenaustausch zwischen Repositories\\
SWRC&Semantic Web of Research Communities\\
Tags&Schlagwörter\\
Tagwolke& Darstellung von Schlagwörtern in Wolkenform, wobei dei Größe eines Wortes seine Häufigkeit in einem Dokumentenkorpus wiederspiegelt.\\
UB&Universitätsbibliothek\\
URL &Uniform Resource Locator (dt.: Einheitlicher Ressourcenzeiger) bezeichnet den Standort einer digitalen Ressource, z.B. einer Webseite\\
vgl.&vergleiche\\
XML&Extensible Markup Language; Auszeichnungssprache, um Informationen zu strukturieren\\
z.B.&zum Beispiel\\
Zope&Content-Management-System zur Gestaltung von Webseiten\\
Zotero & Literaturverwaltungsprogramm\\
\end{longtable}
