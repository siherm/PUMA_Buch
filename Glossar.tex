\section{Glossar}
\label{sec:glossar}
%\newglossaryentry{pdf}{name={ac-Konto}, description={Mitarbeiterkonto}}
%\glsaddall \printglossary 
\begin{longtable}{c p{8cm}}
 & \makecell*[tl]{}\\
APA&\makecell*[tl]{Zitationsstil der Amerikan Psychological Association}\\ 
API&  \makecell*[tl]{ Application programming interface (dt.: Programmierschnittstelle)}\\
API-Key&\makecell*[tl]{Code für die Programmierschnittstelle}\\
Citavi & \makecell*[tl]{Literaturverwaltungsprogramm}\\
CSL&\makecell*[tl]{Citation Style Language}\\
CSS&\makecell*[tl]{Cascading Style Sheets; Bildet zusammen mit HTML eine der Kernsprachen des World Wide Webs}\\
DFG&\makecell*[tl]{Deutsche Forschungsgemeinschaft}\\
dblp&\makecell*[tl]{Digital Bibliography and Libary Project, ist eione online verfügbare bibliographische Sammlung wissenschaftlicher Publikationen im Bereich Informatik}\\
DOI&  \makecell*[tl]{Digital Object Identifier}\\
Dropdownmenü&\makecell*[tl]{=Untermenü}\\
fn-Konto&\makecell*[tl]{Funktionskonto bei der Universitätsbibliothek Stuttgart}\\
HTML  & \makecell*[tl]{Hypertext Markup Language (Programmiersprache für Webseiten)}\\
ISBN & \makecell*[tl]{Internationsale Standardnummer}\\
ISSN & \makecell*[tl]{Internationale Standardnummer für fortlaufende Sammlerwerke}\\
ist &  \makecell*[tl]{Institut für Systemtheorie und Reglungstechnik an der Universität Stuttgart}\\ 
JabRef & \makecell*[tl]{Literaturverwaltungsprogramm}\\
JSON-Feed&\makecell*[tl]{}\\
OAuth & \makecell*[tl]{Ist ein offenes Protokoll, das eine standardisierte sichere API-Authentifizierung für Desktop- und Web- Anwendungen gestattet}\\
OenAccess  &\makecell*[tl]{Freier Zugang zu wissenschaftlicher Literatur und anderen Materialien im Internet}\\
OpenCMS &  \makecell*[tl]{Ist ein Content-management-System für die Gestaltung von Webseiten}\\
Open-URL & \makecell*[tl]{Ist ein Standard zur Angabe von Metadaten in einer URL, um unabhängig vom aktuellen Standort elektronische Dokumente zu verlinken}\\
OPUS&\makecell*[tl]{Dokumentenserver der Universität Stuttgart}\\
PBM&\makecell*[tl]{PUMA/ BibSonomy Module}\\
PHP&\makecell*[tl]{Personal Home Page Tools, heute: Hypertext Preprocessor (Skriptsprache)}\\
PUMA & \makecell*[tl]{Akademisches Publikationsmanagement}\\
Rest-API&\makecell*[tl]{Representational State Transfer}\\
RFC&   \makecell*[tl]{Request of Comments (dt.: Die Bitte um Kommentare)}\\
RSS&\makecell*[tl]{Really Simple Syndication, dt.: Dateiformate, die Veränderungen auf Webseiten zeigen}\\
st-Konto&  \makecell*[tl]{Studentenkonto bei der Universitätsbibliothek Stuttgart}\\
SWORD&\makecell*[tl]{}\\
SWRC&\makecell*[tl]{Semantic Web of Research Communities}\\
Tags&\makecell*[tl]{=Schlagwörter}\\
UB&\makecell*[tl]{Universitätsbibliothek}\\
URL &\makecell*[tl]{Uniform Resource Locator (dt.: Einheitlicher Ressourcenzeiger)}\\
vgl.&\makecell*[tl]{vergleiche}\\
XML&\makecell*[tl]{Extensible Markup Language; Erweiterbare Auszeichnungssprache}\\
z.B.&\makecell*[tl]{zum Beispiel}\\
Zope&\makecell*[tl]{Content-Management-System zur Gestaltung von Webseiten}\\
Zotero &\makecell*[tl]{ Literaturverwaltungsprogramm}\\
\end{longtable}


